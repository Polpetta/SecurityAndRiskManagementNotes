%**************************************************************
% file contenente le impostazioni della tesi
%**************************************************************

%**************************************************************
% Frontespizio
%**************************************************************
\newcommand{\myName}{Davide Polonio, Federico Tavella, Mirko Bez, Marco Zanella}

\newcommand{\myTitle}{Security and Risk Management}
\newcommand{\myUni}{Università degli Studi di Padova}           % università
\newcommand{\myFaculty}{Corso di Laurea Magistrale in Informatica} %facoltà
\newcommand{\myDepartment}{Dipartimento di Matematica}          % dipartimento
\newcommand{\myLocation}{Padova}                                % dove
\newcommand{\myAA}{2016-2017}                                   % anno
\newcommand{\myCopyright}{CC-BY-SA-4.0}                         % copyright
\newcommand{\myVersion}{v0.16.0}                                % versione
\newcommand{\myRelease}{
https://github.com/Polpetta/SecurityAndRiskManagementNotes}
\newcommand{\myIssue}{
https://github.com/Polpetta/SecurityAndRiskManagementNotes/issues}
\newcommand{\myPullRequest}{
https://github.com/Polpetta/SecurityAndRiskManagementNotes/pulls}



%**************************************************************
% Impostazioni di impaginazione
% see: http://wwwcdf.pd.infn.it/AppuntiLinux/a2547.htm
%**************************************************************

\setlength{\parindent}{14pt}   % larghezza rientro della prima riga
\setlength{\parskip}{0pt}   % distanza tra i paragrafi


%**************************************************************
% Impostazioni di caption
%**************************************************************
\captionsetup{
    tableposition=top,
    figureposition=bottom,
    font=small,
    format=hang,
    labelfont=bf
}

%**************************************************************
% Impostazioni di glossaries
%**************************************************************
\input{res/glossary} % database di termini
\makeglossaries


%**************************************************************
% Impostazioni di graphicx
%**************************************************************
\graphicspath{{res/img/}} % cartella dove sono riposte le immagini


%**************************************************************
% Impostazioni di hyperref
%**************************************************************
\hypersetup{
    %hyperfootnotes=false,
    %pdfpagelabels,
    %draft,	% = elimina tutti i link (utile per stampe in bianco e nero)
    colorlinks=true,
    linktocpage=true,
    pdfstartpage=1,
    pdfstartview=FitV,
    % decommenta la riga seguente per avere link in nero (per esempio per la
%stampa in bianco e nero)
    %colorlinks=false, linktocpage=false, pdfborder={0 0 0}, pdfstartpage=1,
%pdfstartview=FitV,
    breaklinks=true,
    pdfpagemode=UseNone,
    pageanchor=true,
    pdfpagemode=UseOutlines,
    plainpages=false,
    bookmarksnumbered,
    bookmarksopen=true,
    bookmarksopenlevel=1,
    hypertexnames=true,
    pdfhighlight=/O,
    %nesting=true,
    %frenchlinks,
    urlcolor=webbrown,
    linkcolor=webbrown,
    citecolor=webgreen,
    %pagecolor=RoyalBlue,
    %urlcolor=Black, linkcolor=Black, citecolor=Black, %pagecolor=Black,
    pdftitle={\myTitle},
    pdfauthor={\textcopyright\ \myName, \myUni, \myFaculty},
    pdfsubject={},
    pdfkeywords={},
    pdfcreator={pdfLaTeX},
    pdfproducer={LaTeX}
}

%**************************************************************
% Impostazioni di itemize
%**************************************************************
%\renewcommand{\labelitemi}{$\ast$}

%\renewcommand{\labelitemi}{$\bullet$}
%\renewcommand{\labelitemii}{$\cdot$}
%\renewcommand{\labelitemiii}{$\diamond$}
%\renewcommand{\labelitemiv}{$\ast$}


%**************************************************************
% Impostazioni di listings
%**************************************************************
\lstset{
    language=[LaTeX]Tex,%C++,
    keywordstyle=\color{RoyalBlue}, %\bfseries,
    basicstyle=\small\ttfamily,
    %identifierstyle=\color{NavyBlue},
    commentstyle=\color{Green}\ttfamily,
    stringstyle=\rmfamily,
    numbers=none, %left,%
    numberstyle=\scriptsize, %\tiny
    stepnumber=5,
    numbersep=8pt,
    showstringspaces=false,
    breaklines=true,
    frameround=ftff,
    frame=single
}


%**************************************************************
% Impostazioni di xcolor
%**************************************************************
\definecolor{webgreen}{rgb}{0,.5,0}
\definecolor{webbrown}{rgb}{.6,0,0}
\definecolor{Pantone}{RGB}{155,0,20}


%**************************************************************
% Altro
%**************************************************************

\newcommand{\omissis}{[\dots\negthinspace]} % produce [...]

% eccezioni all'algoritmo di sillabazione
\hyphenation
{
    ma-cro-istru-zio-ne
    gi-ral-din
}

\newcommand{\sectionname}{sezione}
%\addto\captionsitalian{\renewcommand{\figurename}{Figura}
%                       \renewcommand{\tablename}{tabella}}

\newcommand{\glsfirstoccur}{\ap{{[g]}}}

\newcommand{\intro}[1]{\emph{\textsf{#1}}}

%-------------------INIZIO creazione subsubparagraph--------------------------------------
\makeatletter
\newcounter{subsubparagraph}[subparagraph]
\def\toclevel@subsubparagraph{6}
\renewcommand\thesubsubparagraph{%
  \thesubparagraph.\@arabic\c@subsubparagraph}
\newcommand\subsubparagraph{%
  \@startsection{subsubparagraph}    % counter
    {6}                              % level
    {\parindent}                     % indent
    {3.25ex \@plus 1ex \@minus .2ex} % beforeskip
    {-1em}                           % afterskip
    {\normalfont\normalsize\bfseries}}
\newcommand\l@subsubparagraph{\@dottedtocline{6}{13.5em}{5em}}
\newcommand{\subsubparagraphmark}[1]{}
\setcounter{tocdepth}{6}
\setcounter{secnumdepth}{6} % aggiunge contatore ai paragrafi
\makeatother
%-------------------FINE creazione subsubparagraph--------------------------------------

%-------------------Capitoli personalizzati------------------------------------

\titleformat{\chapter}[display]
  {\normalsize \huge  \color{black}}%
  {\flushright\normalsize \color{Pantone}%
   \MakeUppercase{\chaptertitlename}\hspace{1ex}%
   {\fontsize{60}{60}\selectfont\thechapter}}%
  {10 pt}%
  {\bfseries\huge}%

%----------------FINE Capitoli personalizzati----------------------------------
