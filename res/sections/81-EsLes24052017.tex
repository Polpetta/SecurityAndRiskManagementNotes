\section{Controlli}
\label{es:Controlli}

La teoria riguardante questi esercizi è disponibile in \ref{SicurezzaDelSoftware}

\begin{Exercise} [
  title={Associa i termini},
  label={esControlli1}
  ]

  \Question Eseguire il match delle definizioni di:
\begin{itemize}
\item Whitelist
\item Blacklist
\item Nonce
\item Jail
\item Sandbox Environment
\end{itemize}
con:
\begin{enumerate}
 \item un insieme di risorse imposte ai programmi dal kernel del sistema
operativo (per esempio la quota disco)
 \item utilizzo di timestamp per impedire il \textit{replay} attack (ad esempio
captcha)
 \item usare una lista di input accettabili
 \item un meccanismo di sicurezza per porre in quarantena programmi non fidati
 \item rigettare input sospetto
\end{enumerate}

\end{Exercise}

\begin{Answer} [
  ref={esControlli1},
  number={1}
  ]

  \Question Soluzione:
\begin{itemize}
\item Whitelist: 3
\item Nonce: 2
\item Jail: 1
\item Sandbox Environment: 4
\item Blacklist: 5

\end{itemize}

\end{Answer}



\begin{Exercise} [
  title={Quiz},
  label={esControlli2}
  ]

  \Question Una terza parte inserisce dati maliziosi in un'altra pagina html
maliziosa. Questa è conosciuta come:
\begin{enumerate}
\item \textit{cross-site scripting}
\item \textit{blacklist}
\item \textit{race condition}
\item \textit{cleartext}
\end{enumerate}

\end{Exercise}

\begin{Answer} [
  ref={esControlli2},
  number={2}
  ]

  \Question Risposta esatta: 1
\end{Answer}

\begin{Exercise} [
  title={Quiz},
  label={esControlli3}
  ]

  \Question Quale tecnica non è appropriata nell'evitare
  l'\textit{OS command injection}?
\begin{enumerate}
\item Separare le informazioni sul controllo da quelle sui dati.
\item Usare una libreria
\item Eseguire il codice in ambiente ``jail'' o sandbox
\item Usare una password \textit{hard-coded} per consentire l'accesso.
\end{enumerate}


\end{Exercise}

\begin{Answer} [
  ref={esControlli3},
  number={3}
  ]

  \Question Risposta esatta: 4
\end{Answer}


\begin{Exercise} [
  title={Quiz},
  label={esControlli4}
  ]

  \Question Quale delle seguenti risposte è vera riguardo ai web server?
  \begin{enumerate}
   \item i server non possono avere lo stato della sessione, e quindi se ne
devono occupare i client
   \item il solo miglior punto per fare input validation e autenticazione è
lato client
   \item usare il client come punto di salvataggio è sicuro se la criptazione
e i controlli di integrità sono attivi
   \item I server possono fidarsi degli input provenienti dal web se validano i
dati nella form.
  \end{enumerate}
\end{Exercise}

\begin{Answer} [
  ref={esControlli4},
  number={4}
  ]

  \Question Risposta esatta: 3
\end{Answer}

\begin{Exercise} [
  title={Quiz},
  label={esControlli5}
  ]

  \Question Il modo \textbf{migliore} per validare correttamente l'input lato
client è tramite:
  \begin{enumerate}
\item \textit{Nonce}
\item \textit{Whitelist}
\item \textit{Blacklist}
\item \textit{Integrity Checking}
\end{enumerate}


\end{Exercise}

\begin{Answer} [
  ref={esControlli5},
  number={5}
  ]

  \Question Risposta esatta: 2
\end{Answer}


\begin{Exercise} [
  title={Quiz},
  label={esControlli6}
  ]

  \Question La migliore implementazione per eseguire correttamente
\textit{access control} sarebbe:
\begin{enumerate}
 \item non dare possibilità all'attaccante di guadagnare dati sensibili
 \item usare sempre il minor numero di permessi nel codice, per la durata più
breve possibile
 \item evitare di usare cookies e file nascosti
 \item non darei mai una autorizzazione più avanzata di quella \textit{guest}
per gli utenti web
\end{enumerate}

\end{Exercise}

\begin{Answer} [
  ref={esControlli6},
  number={6}
  ]

  \Question Risposta esatta: 2
\end{Answer}


\begin{Exercise} [
  title={Quiz},
  label={esControlli7}
  ]

  \Question La protezione migliore contro il \textit{SQL injection} è:
  \begin{enumerate}
    \item \textit{Cleartext}
    \item \textit{Encryption and Integrity Checking}
    \item \textit{Sanitization}
    \item Definire UTF-8 come codifica
  \end{enumerate}
\end{Exercise}

\begin{Answer} [
  ref={esControlli7},
  number={7}
  ]

  \Question Risposta esatta: 3
\end{Answer}


\begin{Exercise} [
  title={Quiz},
  label={esControlli8}
  ]

  \Question Il metodo principale per evitare un \textit{replay attack} tra il
client e il server è:
\begin{enumerate}
\item Integrity checking
\item Whitelist
\item Blacklist
\item Nonce
\end{enumerate}

\end{Exercise}

\begin{Answer} [
  ref={esControlli8},
  number={8}
  ]

  \Question Risposta esatta: 4
\end{Answer}


\begin{Exercise} [
  title={Quiz},
  label={esControlli9}
  ]

  \Question Un attacco che potrebbe causare i problemi più gravi include:
  \begin{enumerate}
    \item \textit{hard-coded password}
    \item \textit{race condition}
    \item \textit{denial of service}
    \item \textit{chatty error message}
  \end{enumerate}

\end{Exercise}

\begin{Answer} [
  ref={esControlli9},
  number={9}
  ]

  \Question Risposta esatta: 1
\end{Answer}


\begin{Exercise} [
  title={Quiz},
  label={esControlli10}
  ]

  \Question La maniera migliore per assicurarsi che nessuna modifica a un
messaggio avvenga è tramite:
\begin{enumerate}
  \item Hashing
  \item Whitelist
  \item Blacklist
  \item Encryption
\end{enumerate}

\end{Exercise}

\begin{Answer} [
  ref={esControlli10},
  number={10}
  ]

  \Question Risposta esatta: 1
\end{Answer}


\begin{Exercise} [
  title={Quiz},
  label={esControlli11}
  ]

  \Question Tutte le risposte \textbf{eccetto una} possono causare dati
  invalidi e intrusioni. Qual è l'eccezione?
\begin{enumerate}
 \item Generatore di numeri randomici non (veramente) casuali
 \item \textit{Buffer overflow}
 \item Variabili non inizializzate risultati in messaggi di errore
 \item \textit{Race conditions}
\end{enumerate}

\end{Exercise}

\begin{Answer} [
  ref={esControlli11},
  number={11}
  ]

  \Question Risposta esatta: 4
\end{Answer}
