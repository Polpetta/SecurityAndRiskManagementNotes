\section{Incident Response vs Business Contiuity}
\label{esIRBC}

La teoria \`e disponibile in \ref{IRBC}

\subsection{Pianificazione dei processi}
\label{esIRBC:pp}

La teoria \`e disponibile in \ref{IRBC:pp}



\begin{Exercise} [
  title={Quiz},
  label={esIRBC1}
  ]

  \Question La sfida principale nel mettere insieme un IRP \`e principalmente
dovuta a:
\begin{enumerate}
 \item Avere il supporto del \textit{management} e \textit{department}
 \item Capire i requisiti per la \textit{chain of custody}
 \item Mantenere l'IRP aggiornato
 \item Assicurarsi che l'IRP sia corretto
\end{enumerate}
\end{Exercise}

\begin{Answer} [
  ref={esIRBC1},
  number={1}
  ]

  \Question Risposta esatta: 1
\end{Answer}


\begin{Exercise} [
  title={Quiz},
  label={esIRBC2}
  ]

  \Question La ragione principale per eseguire il \textit{Triage} \`e:
  \begin{enumerate}
   \item Per coordinare le risorse limitate
   \item Per disinfettare un sistema compromesso
   \item Per determinare la ragione dell'incidente
   \item Per scovare un incidente
  \end{enumerate}

\end{Exercise}

\begin{Answer} [
  ref={esIRBC2},
  number={2}
  ]

  \Question Risposta esatta: 1\footnote{La seconda non \`e in quanto il termine
corretto dovrebbe essere \textit{sinetized}, e non \textit{disinfect}}
\end{Answer}


\begin{Exercise} [
  title={Quiz},
  label={esIRBC3}
  ]

  \Question Quando un sistema \`e stato compromesso a livello amministrativo,
l'azione \textbf{pi\`u} importante da fare \`e:
\begin{enumerate}
 \item Assicurarsi che le \textit{patch} e gli antivirus siano aggiornati
 \item Cambiare la password dell'amministratore
 \item Richiedere assistenza legale per investigare sull'incidente
 \item Ricorstruire il sistema
\end{enumerate}
\end{Exercise}

\begin{Answer} [
  ref={esIRBC3},
  number={3}
  ]

  \Question Risposta esatta: 4
\end{Answer}


\begin{Exercise} [
  title={Quiz},
  label={esIRBC4}
  ]

  \Question Il metodo migliore per scovare un incidente \`e:
  \begin{enumerate}
   \item Investigare i report per scovare discrepanze
   \item Usare la tecnologia NIDS/HIDS
   \item Eseguire una scansione regolare delle vulnerabilit\`a
   \item Eseguire una rotazione dei lavori
  \end{enumerate}
\end{Exercise}

\begin{Answer} [
  ref={esIRBC4},
  number={4}
  ]

  \Question Risposta esatta: 2
\end{Answer}


\begin{Exercise} [
  title={Quiz},
  label={esIRBC5}
  ]

  \Question La persone o il gruppo che sviluppa le strategie per rispondere
agli incidenti include:
\begin{enumerate}
 \item CISO
 \item CRO
 \item IRT
 \item IMT
\end{enumerate}

\end{Exercise}

\begin{Answer} [
  ref={esIRBC5},
  number={5}
  ]

  \Question Risposta esatta: 4\footnote{REsponse team lavora, IMT crea le
strategie}
\end{Answer}


\begin{Exercise} [
  title={Quiz},
  label={esIRBC6}
  ]

  \Question La \textbf{prima} cosa che dovrebbe essere fatta quando tu scopri
che un intruso ha hackerato il tuo computer \`e quello di:
\begin{enumerate}
 \item Disconnettere il computer dalla rete e sperare di disconnettere
l'attaccante
 \item Spegnere il server per evitare una ulteriore perdita in
\textit{confidentiality} e in integrit\`a dei dati.
 \item Chiamare la polizia
 \item Seguire le direzioni dettate dall'\textit{Incident Response Plan}
\end{enumerate}
\end{Exercise}

\begin{Answer} [
  ref={esIRBC6},
  number={6}
  ]

  \Question Risposta esatta: 4
\end{Answer}
