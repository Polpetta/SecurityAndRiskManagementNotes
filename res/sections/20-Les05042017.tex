% È roba un po' confusa, è prima mattina per tutti. Se hai capito meglio quello che il prof voleva dire sei libero di mettere a posto :D
Un attaccante può tentare di modificare il messaggio ma quello che gli mancherà sarà $k$, ovvero la mia chiave con cui cifro il messaggio.
L'attaccante allora potrebbe provare a sostituire la chiave $k$, e il difensore non si accorgerebbe di nulla. Questo metodo funziona quando il file ha quindi una struttura, ed è presente un'alta entropia.


\subsubsection{Firme elettroniche}

Non si firma mai il messaggio perché è molto oneroso ($m^d$ con $m$ molto grande), quello che si fa è applicare l'hash al messaggio e fare la firma dell'hash del messaggio.

$$
m \rightarrow h(m) \rightarrow s(h(m))
$$

La firma è

$$
\langle m, h(m), n \rangle
$$

Per verificare prendo il messaggio, calcolo l'hash e lo confronto con l'hash ricevuto del messaggio. Se sono uguali la comunicazione è andata a buon fine. Questo garantisce l'autenticità del messaggio e il non ripudio.

Il non ripudio è qualcosa di fondamentale per permettere il commercio. Il non ripudio è possibile perchè si è in grado di \textbf{firmare}. La firma digitale implementa il non ripudio nel mondo elettronico.

\subsubsection{Public Key Infrastructure}

\todo{Copia disegno}

Una \textit{Certification Authority} è una struttura che gestisce la \textit{catena di trust}. In Italia sono presenti queste società che permettono di firmare in maniera digitale documenti, con valore legale. Questo avviene perchè quando si va ad aprire un account è necessario presentare a mano la propria carta d'identità.

Quindi in una CA sono presenti:
\begin{itemize}
\item I dati anagrafici
\item L'algoritmo per eseguire la criptazione
\item La propria chiave pubblica
\item La firma della CA, fatta con la chiave privata
\end{itemize}

I primi tre punti costituiscono $m$.

È anche presente un database pubblico con i certificati che sono stati revocati, questo per assicurare al venditore che prima di ogni transazione possa andare a verificare dalla CA che il certificato dell'acquirente è valido.

Esistono diverse CA.

\subsection{Network Access Server}

NAS: Network Access Server

Oggi queste modalità sono soppiantate da VPN.

\todo{Completare}

\subsection{Vulnerability Assessment}

\begin{itemize}
\item Scan servers, work stations e controllare i dispositivi per le vulnerabilità (servizi open, patching, vulnerabilità delle configurazioni). L'80\% degli attacchi avviene perchè i sistemi non sono patchati. Alcune aziende prevedono nella policy che i sistemi devono essere patchati ogni $x$ mesi in modo da evitare vulnerabilità. Senza queste policy è difficile che si facciano!
% PATCHAH PATCHAH PATCHAH PATCHAH
\item Controlli e test per il corretto funzionamento
\begin{itemize}
 \item Aderenza alle \textit{policy \& standards}
\end{itemize}
\item Penetration testing, molto utilizzato per le reti wireless, meno utile su servizi web. 
\end{itemize}

\subsubsection{Serviced Applications}

\todo{Copia tabella}

\section{Riassunto dei controlli di rete}

\todo{copia elenco}


\section{Esercizi}


A map of the network that shows where service requests entrer and are processed:

\begin{itemize}
\item Is called the path of physical access (no perché è la mappa degli accessi logici)
\item Is primarily used in developing security policies
\item Can be used to determine whether sufficient defense in depth is implemented (corretta)
\item Helps to determine where antivirus software should be installed
\end{itemize}


% ALTRO ESERCIZIO

The filter with the most extensive filtering capability is the

\begin{itemize}
\item Packet filter
\item Application-level firewall (corretta)
\item Circuit-level firewall
\item State inspection
\end{itemize}


% ALTRO ESERCIZIO

The technique which implements non-repudiation is:

\begin{itemize}
\item Hash
\item Secrey Key Encryption
\item Digital Signature (corretta)
\item IDS
\end{itemize}



% ALTRO ESERCIZIO

Anti-virus software typically implements which type of defensive software

\begin{itemize}
\item Neural Network
\item Statistical-based
\item Signature-based (corretta)
\item Packet filter 
\end{itemize}


% ALTRO ESERCIZIO

MD5 is an example of what type of software

\begin{itemize}
\item Public Key Encryption
\item Secret Key Encryption
\item Message Authentication (corretta)
\item PKI
\end{itemize}

% ALTRO ESERCIZIO

A personal firewall implemented as parte of the OS or antivirus software qualifies as a:

\begin{itemize}
\item Dual-homed firewall
\item Packet filter (corretta)
\item Screened host
\item Bastion host
\end{itemize}

\part{Risk Management}

Quanto è importante vestire nella sicurezza? La sicurezza è sempre più ``venduta'' e comunicata come il classico ROI\footnote{Ritorno d'investimento}. Inoltre c'è anche un altro fattore da tenere conto: la pubblicità. La cattiva pubblicità infatti è molto devastante per un'azienda, e può danneggiare molto facilmente la fiducia degli utenti. Si stima che ci sia un ritorno dell'investimento del 5\%.

Non tutti i controlli vanno implementati, bisogna sapere a cosa indirizzare. L'importante è convinvere i decisori a ``convincerli'' a spendere in sicurezza.



\chapter{Gestione del rischio}

La storia, il grado di maturità, la cultura, la società e la tolleranza influiscono sul rischio. Esistono anche dei fattori esterni, che sono le \textit{regulation}, ovvero le normative, le leggi. Il fattore esterno è la motivazione più forte oggigiorno, in quanto son previsti anche sanzioni di tipo penale.

L'altro fattore esterno è la \textit{industry}, ossia confrontarsi con i propri competitor.

\textbf{Risk tolerance:} come il management vive la percezione del rischio (quanto rischio è disposto ad accettare).

\section{Risk Management Process}
\todo{copia schema}
I passi della gestione del rischio si suddividono in passi:
\begin{enumerate}
\item Individuazione dei limiti e dei contesti
\item Valutazione del rischio
\item Trattamento del rischio
\item Risk assessment
	\begin{itemize}
		\item Identificazione (Asset: materiali, immateriali\footnote{Reputazione ad esempio})
		\item Analisi
		\item valutazione
	\end{itemize}
\item Risk Treatment
	\begin{itemize}
	\item evitare
	\item ridurre
	\item trasferire
	\item mantenere
	\end{itemize}
\item Accept Residual Risk
\end{enumerate}

Il rischio di può ridurre, mantenere, trasferire, evitare. Alla fine rimane il rischio residuo, dove bisogna decidere se questo rischio è sopportabile o meno, o se ci sono miglioramenti che sono possibili da intraprendere. In caso sia possibile eseguire un'ulteriore riduzione del rischio si re-itera il processo.

Una volta che il lavoro è stato fatto questo dev'essere comunicato con le varie persone incaricate. Il monitoraggio è una parte fondamentale del PDCA, che permette un miglioramento continuo






\section{Risk Appetite}

Come una persona si pone nel confronto del rischio (es. aprire mail che dentro hanno spam, dare i dati della carta di credito).

È sempre meglio in ogni caso eseguire una valutazione piuttosto che scegliere ``a scatola chiusa''.


\section{Processo di Risk Management}

Si parte con il \textit{Risk appetite}, per identificare il rischio, per poi muoversi verso la creazione di un piano della gestione del rischio, che è importante che venga implementato. Siccome i rischi si evolvono e continuano a cambiare, è importate eseguire un \textit{monitoring} continuo del rischio. È anche importante tenere in considerazione il fatto che i controlli potrebbero introdurre delle nuove vulnerabilità. 

Il processo quindi dev'essere iterativo, in quanto l'ambiente è dinamico.


\subsection{Risk Assessment}
%CROWN JUWELS (gioiello della corona) assets che valgono l'80% degli asset p.es. brevetto particolare che è il core business
Include 5 step:

\subsubsection{Assegnazione dei valori agli assets}

Quando eseguo \textit{risk assessment} non è possibile contare tutti gli oggetti all'iterno di un'azienda, ed è quindi importante individuare solamente tutto quello che veramente conta per la socità.

\paragraph*{In pratica}
% Parte aggiunta dopo

Per agire quindi bisogna valutare l'ultimo \textit{risk assessment} che è stato effettuato, e bisogna vedere se ci sono state aggiunte o rimozioni.

Una volta identificati gli \textit{assets} più importanti è necessario calcolarne il valore, che spesso non risulta essere semplice\footnote{Ad esempio il costo di un determinato Software}. Bisogna inoltre calcolare il costo di una eventuale \textit{recovery}.



\subsubsection{Determinare la perdita causata dalle vulnerabilità e dalle minacce}

Bisogna assegnare un valore a questi oggetti, soprattutto se immateriali. Spesso i beni immateriali (come la reputazione) si riesce a valutare effettivamente solamente dopo che la perdita è avvenuta.

Le perdite vengono calcolare su confidenzialità, integrità e disponiblità.

\paragraph*{In pratica}
% Parte aggiunta dopo

Esistono i costi tangibili e intangibili. Sono presenti anche il costo del rimpiazzo e quelle legate alle perdite. 
Il costo dal punto di vista intangibile è di tipo qualitativa, e non quantitativa.

\todo{Ci sarebbe una tabella qui}

\subsubsection{Stimare la probabilità di exploit}

Si valuta qual è la probabilità che un evento malevolo mi colpisca tramite survey, interviste e casi simili già accaduti.

\paragraph*{In pratica}



\subsubsection{Calcolare la perdita attesa}

La perdita è composta dal \textit{downtime}, \textit{recovery}, \textit{liability} e il rimpiazzo del prodotto. Il costo dell rimpiazzo spesso è minore di tutto quello che ci sta prima: ovvero tutti i costi che vengono a crearsi a causa della disfunzione creata.

\paragraph*{In pratica}



\subsubsection{Trattamento dei rischi}

Survey e controllo esistenti, più nuovi controlli.

Decidiamo se trattare, trasferire, mantenere o ignorare il rischio. \todo{completare}.

\paragraph*{In pratica}


