% È roba un po' confusa, è prima mattina per tutti. Se hai capito meglio quello
che il prof voleva dire sei libero di mettere a posto :D
Un attaccante può tentare di modificare il messaggio ma quello che gli mancherà
sarà $k$, ovvero la mia chiave con cui cifro il messaggio.
L'attaccante allora potrebbe provare a sostituire la chiave $k$, e il difensore
non si accorgerebbe di nulla. Questo metodo funziona quando il file ha quindi
una struttura, ed è presente un'alta entropia.

Non si firma mai il messaggio perché è molto oneroso ($m^d$ con $m$ molto
grande), quello che si fa è applicare l'hash al messaggio e fare la firma
dell'hash del messaggio.

$$
m \rightarrow h(m) \rightarrow s(h(m))
$$

La firma è

$$
\langle m, h(m), n \rangle
$$

Per verificare prendo il messaggio, calcolo l'hash e lo confronto con l'hash
ricevuto del messaggio. Se sono uguali la comunicazione è andata a buon fine.
Questo garantisce l'autenticità del messaggio e il non ripudio.

Il non ripudio è qualcosa di fondamentale per permettere il commercio. Il non
ripudio è possibile perchè si è in grado di \textbf{firmare}. La firma digitale
implementa il non ripudio nel mondo elettronico.

\paragraph{Public Key Infrastructure}

Una \textit{Certification Authority} è una struttura che gestisce la
\textit{catena di trust}. In Italia sono presenti queste società che permettono
di firmare in maniera digitale documenti, con valore legale. Questo avviene
perchè quando si va ad aprire un account è necessario presentare a mano la
propria carta d'identità.


\begin{figure}[H]
 \centering
 \includegraphics[scale=0.45]{publicKeyInfrastructure}
 \caption{Schema di funzionamento di una infrastruttura a chiave pubblica}
\end{figure}

Quindi in una CA sono presenti:
\begin{itemize}
\item I dati anagrafici
\item L'algoritmo per eseguire la criptazione
\item La propria chiave pubblica
\item La firma della CA, fatta con la chiave privata
\end{itemize}

I primi tre punti costituiscono $m$.

È anche presente un database pubblico con i certificati che sono stati revocati,
questo per assicurare al venditore che prima di ogni transazione possa andare a
verificare dalla CA che il certificato dell'acquirente è valido.

Esistono diverse CA.


\subsubsection{Network Access Server}

NAS: Network Access Server presenta differenti caratteristiche. Un NAS gestisce
l'autenticazione degli utenti, compiendo l'\textit{access control} e
l'\textit{accounting} di essi. Una volta eseguito con successo un login,
vengono inviate al client le informazioni sull'utente (come ad esempio il suo
ID).
Il problema dei NAS \`e che sono proni ad attacchi da parte di hackers, oltre a
essere suscettibili di attacchi DOS.

Oggi questa modalità \`e stata soppiantata dalle VPN.

\subsection{Vulnerability Assessment}

\begin{itemize}
\item Scan servers, work stations e controllare i dispositivi per le
vulnerabilità (servizi open, patching, vulnerabilità delle configurazioni).
L'80\% degli attacchi avviene perchè i sistemi non sono patchati. Alcune aziende
prevedono nella policy che i sistemi devono essere patchati ogni $x$ mesi in
modo da evitare vulnerabilità. Senza queste policy è difficile che si facciano!
% PATCHAH PATCHAH PATCHAH PATCHAH
\item Controlli e test per il corretto funzionamento
\begin{itemize}
 \item Aderenza alle \textit{policy \& standards}
\end{itemize}
\item Penetration testing, molto utilizzato per le reti wireless, meno utile su
servizi web.
\end{itemize}

%\subsubsection{Serviced Applications}

% Qui non mettiamo la tabella perche' e' inutile, non aggiunge nessun contenuto
utile agli appunti imho

\section{Riassunto dei controlli di rete}

Riassumento, i controlli di rete possono essere di tre tipologie:

\begin{enumerate}
 \item Tecniche di controllo di sicurezza della rete
 \begin{itemize}
  \item Criptazione
  \item VPN
  \item Hashing
  \item Firme digitali
  \item Configurazione host in modalità \textit{bastion}
  \item Certificato di autorit\`a, infrastruttura a chiave pubblica
 \end{itemize}

 \item Protezione dei \textit{device} nella rete
 \begin{itemize}
  \item Firewall
  \item Proxy server
  \item DMZ
  \item Sistemi di rivelazione di intrusione
  \item Sistemi di prevenzione di intrusione
  \item NAS (\textit{network access server})
  \item \textit{Honeypot}, \textit{honeynet}
 \end{itemize}

 \item Protocolli di sicurezza
 \begin{itemize}
  \item SSL
  \item SSH
  \item S/MIME
  \item Gestione dell'informazione sulla sicurezza: gestione dei log
 \end{itemize}
\end{enumerate}



\section{Esercizi}

Gli esercizi riepilogativi su questa parte del corso si possono trovare in
\ref{EsNetRiep}.


\part{Risk Management}

\label{riskMng}

Quanto è importante vestire nella sicurezza? La sicurezza è sempre più
``venduta'' e comunicata come il classico ROI\footnote{Ritorno d'investimento}.
Inoltre c'è anche un altro fattore da tenere conto: la pubblicità. La cattiva
pubblicità infatti è molto devastante per un'azienda, e può danneggiare molto
facilmente la fiducia degli utenti. Si stima che ci sia un ritorno
dell'investimento del 5\%.

Non tutti i controlli vanno implementati, bisogna sapere a cosa indirizzare.
L'importante è convinvere i decisori a ``convincerli'' a spendere in sicurezza.



\chapter{Gestione del rischio}

La storia, il grado di maturità, la cultura, la società e la tolleranza
influiscono sul rischio. Esistono anche dei fattori esterni, che sono le
\textit{regulation}, ovvero le normative, le leggi. Il fattore esterno è la
motivazione più forte oggigiorno, in quanto son previsti anche sanzioni di tipo
penale.

L'altro fattore esterno è la \textit{industry}, ossia confrontarsi con i propri
competitor.

\textbf{Risk tolerance:} come il management vive la percezione del rischio
(quanto rischio è disposto ad accettare).

\section{Risk Management Process}
La gestione del rischio si suddivide nei seguenti in passi:
\begin{enumerate}
\item Individuazione dei limiti e dei contesti
\item Valutazione del rischio
\item Trattamento del rischio
\item Risk assessment
\begin{itemize}
  \item Identificazione (Asset: materiali, non materiali\footnote{Reputazione ad
esempio})
  \item Analisi
  \item valutazione
\end{itemize}
\item Risk Treatment
\begin{itemize}
  \item evitare
  \item ridurre
  \item trasferire
  \item mantenere
\end{itemize}
\item Accept Residual Risk
\end{enumerate}

\begin{figure}[H]
 \centering
 \includegraphics[scale=0.5]{riskManagementProcess}
 \caption{Schema del processo di gestione del rischio}
\end{figure}


Il rischio di può ridurre, mantenere, trasferire, evitare. Alla fine rimane il
rischio residuo, dove bisogna decidere se questo rischio è sopportabile o meno,
o se ci sono miglioramenti che sono possibili da intraprendere. In caso sia
possibile eseguire un'ulteriore riduzione del rischio si re-itera il processo.

Una volta che il lavoro è stato fatto questo dev'essere comunicato con le varie
persone incaricate. Il monitoraggio è una parte fondamentale del PDCA, che
permette un miglioramento continuo






\section{Risk Appetite}

Come una persona si pone nel confronto del rischio (es. aprire mail che dentro
hanno spam, dare i dati della carta di credito).

È sempre meglio in ogni caso eseguire una valutazione piuttosto che scegliere
``a scatola chiusa''.


\section{Processo di Risk Management}

Si parte con il \textit{Risk appetite}, per identificare il rischio, per poi
muoversi verso la creazione di un piano della gestione del rischio, che è
importante che venga implementato. Siccome i rischi si evolvono e continuano a
cambiare, è importate eseguire un \textit{monitoring} continuo del rischio. È
anche importante tenere in considerazione il fatto che i controlli potrebbero
introdurre delle nuove vulnerabilità.

Il processo quindi dev'essere iterativo, in quanto l'ambiente è dinamico.


\subsection{Risk Assessment}

\`E formato da 5 step:
\begin{enumerate}
 \item Assegnazione dei valori agli \textit{assets}. \`E solitamente presente
il problema dell'identificazione dei \textit{Crown Juwels} (gioiello della
corona) ovvero gli \textit{assets} che valgono buona parte del totale, p.es.
brevetto particolare che è il core business.
 \item Determinazione della perdita dovuta e vulnerabilità nel sistema
 \item Stima della possibilit\`a di un attacco portato a termine con successo
(ogni quanto potrebbe accadere?)
 \item Calcolare l'eventuale perdita. Son sempre da tenere a mente le seguenti
formule:
 \begin{itemize}
  \item $Perdita\ =\ Downtime\ +\ Recovery\ +\ Liability\ +\ Replacement$
  \item $Esposizione\ al\ rischio\ =\ Vulnerability\ \cdot\ Loss$
 \end{itemize}

 \item Gestione del rischio
\end{enumerate}

Ogni punto verr\`a spiegato meglio nei successivi paragrafi.


\subsubsection{Assegnazione dei valori agli assets}

Quando eseguo \textit{risk assessment} non è possibile contare tutti gli oggetti
all'iterno di un'azienda, ed è quindi importante individuare solamente tutto
quello che veramente conta per la socità.

\paragraph*{In pratica}
% Parte aggiunta dopo

Per agire quindi bisogna valutare l'ultimo \textit{risk assessment} che è stato
effettuato, e bisogna vedere se ci sono state aggiunte o rimozioni.

Una volta identificati gli \textit{assets} più importanti è necessario
calcolarne il valore, che spesso non risulta essere semplice\footnote{Ad esempio
il costo di un determinato Software}. Bisogna inoltre calcolare il costo di una
eventuale \textit{recovery}.



\subsubsection{Determinare la perdita causata dalle vulnerabilità e dalle
minacce}

Bisogna assegnare un valore a questi oggetti, soprattutto se immateriali. Spesso
i beni immateriali (come la reputazione) si riesce a valutare effettivamente
solamente dopo che la perdita è avvenuta.

Le perdite vengono calcolare su confidenzialità, integrità e disponiblità.

\paragraph*{In pratica}
% Parte aggiunta dopo

Esistono i costi tangibili e intangibili. Sono presenti anche il costo del
rimpiazzo e quelle legate alle perdite.
Il costo dal punto di vista intangibile è di tipo qualitativa, e non
quantitativa.

\begin{figure}[H]
 \centering
 \includegraphics[scale=0.45]{threatAgentTypes}
 \caption{Tabella con dei possibili esempi}
\end{figure}


\subsubsection{Stimare la probabilità di exploit}

Si valuta qual è la probabilità che un evento malevolo mi colpisca tramite
survey, interviste e casi simili già accaduti.

\paragraph*{In pratica}

La passata esperienza, la definizione degli standard internazionali, i consigli
di specialisti ed esperti nel settore, ricerche di mercato e esperimenti
tramite prototipi sono diversi modi da cui \`e possibile attingere
informazione per capire quanto si \`e esposti a un possibile \textit{exploit}.



\subsubsection{Calcolare la perdita attesa}

La perdita è composta dal \textit{downtime}, \textit{recovery},
\textit{liability} e il rimpiazzo del prodotto. Il costo dell rimpiazzo spesso è
minore di tutto quello che ci sta prima: ovvero tutti i costi che vengono a
crearsi a causa della disfunzione creata.

\paragraph*{In pratica}

Per calcolare la perdita attesa esistono tre tipologie di calcolo: la stima
\textbf{qualitativa}, quella \textbf{quantitativa} e quella
\textbf{semiquantitativa}.

% Sta roba sta in 6.3.1.4 Calcolare la perdita attesa
% Copy-Pastata dalla lezione 21
\begin{itemize}
\item Qualitativa: corretta perchè riuscite a prioritizzare i rischi, quindi
capite cosa è più importante di cosa.
\item Quantitativa: la più corretta perché in qualche modo riporta in una scala
assoluta (es. soldi). Il problema è arrivarci ad una valutazione corretta.
\item Semiquantitativa: mix tra le prime due.
\end{itemize}

\subparagraph*{Analisi qualitativa}

L'analisi qualitativa è usata:
\begin{itemize}
\item Come analisi preeliminare per i rischi.
\item Con asset non tangibili come la reputazione e l'immagine.
\item Quando non sono presenti sufficienti informazioni per eseguire un'analisi quantitativa.
\end{itemize}



\subparagraph*{Analisi semiquantitativa}



\subparagraph*{Analisi quantitativa}

\begin{itemize}
\item Single Loss Expectancy (SLE): il costo per l'organizzazione se una
minaccia si verifica:
$$
\text{SLE} = \text{ASSET VALUE (\textbf{AV})} \cdot \text{ Exposure Factor (\textbf{EF})}
$$
\item Annualized Rate of Occurrence (ARO): quanto costa questo evento in un
anno?
\item Annual Loss Expectancy (ALE): la perdita attesa annuale è dato da: 
$$
\text{ALE} = \text{SLE} \cdot \text{ARO}
$$
\end{itemize}


Quanto si perde del bene se l'evento capita?

Sull'anno questo evento quanto mi costa? L'evento negativo ha sicuramente un
certo costo, ma che impatto ha all'anno? Se per esempio una alluvione avviene
ogni 30 anni, allora il suo costo è $1/30$ all'anno. Quindi ogni anno l'azienda
ha da risparmiare un certo valore per far fronte alla perdita futura.

\subsubparagraph*{Esempio}


Quantitativamente:
\begin{itemize}
\item Il costo di un incidente HIPAA con protezioni insufficienti:
\begin{itemize}
 \item SLE = 50k + (1y in prigione) \$ 100k = \$150k
 \item Più perdite in termine di reputazione
\end{itemize}
\item Stima del tempo = 10 anni o meno = 0,1
\item Stime di perdita annuale (ALE): \$150 $\cdot 0,1$ = \$15k
\end{itemize}

\todo{Ci sarebbe anche la tabella per completare l'esempio. Serve? Per me è no,
non quella gialla piuttosto quella dopo? - Ok per quella dopo allora}

% Anche questa parte e' copy-pastata dalla lezione 21

Le \textit{Security breach notification laws} sono leggi che impongono ad entità
che sono state soggette a data stealing di notificarlo ai clienti e di prendere
contromisure per cercare di rimediare alle possibili conseguenze. Esiste una
direttiva europea.


\paragraph*{Tipologie di Minacce}

Tipi di aggenti:
\begin{itemize}
\item Hackers Crackers
\item Criminali
\item Terroristi
\item Spie industriali
\item Dall'interno
\end{itemize}
%copiare tabella


\paragraph*{Come difendersi}

Per difendersi le grande società di software eseguono delle settimane di
allenamento per i programmatori definite anche come \textit{security coding}



\subsubsection{Trattamento dei rischi}

Si eseguono delle \textit{survey} e si controllano i rischi esistenti. Se non
dovesse essere abbastanza, vengono aggiunti nuovi controlli.

Con il trattamento del rischio viene deciso se evitare, trasferire, mantenere,
mitigare o pianificareil rischio.

\begin{itemize}
\item Accettazione del rischio. È importante prendere delle decisioni riguardo
all'accettazione del rischio.
\item Evitare il rischio: bloccare i comportamenti che causano esposizione al
rischio
\item Mitigazione del rischio: implementare dei controlli per minimizzare le
vulnerabilità sotto un livello accettabile.
\item Trasferimento del rischio: qualcun'altro assume il rischio per l'azienda.
Per esempio potrebbe essere ricorrere ad una assicurazione.
\item Pianificazione del rischio: implementare una serie di contromisure. Che
controlli dovrebbero essere messe in piedi?
\end{itemize}
% Il prof ha detto che la roba prima la chiede sicuramente :(

\begin{figure}[H]
 \centering
 \includegraphics[scale=0.5]{treatRisk}
 \caption{Schema di come dovrebbe essere trattato il rischio}
\end{figure}



Ad un certo punto potrebbe non essere più possibile eliminare il rischio. Il
rischio che rimane viene detto \textbf{richio residuo} e questo rischio viene
accettato.

La \textit{sensitivity analysis} indica quanto l'azienda è ``sensibile'' a certi
avvenimenti.

Il \textit{risk assessment} va fatto solo su una parte dell'azienda, non su
tutte.


\paragraph*{In pratica}

% Vedi appunti della lezione successiva.
