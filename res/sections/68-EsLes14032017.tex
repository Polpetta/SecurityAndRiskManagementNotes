\section{Business Continuity and Disaster Recovery}
\label{EsBCDR1}

Teoria disponibile al capitolo \ref{BCDR}

\begin{Exercise} [
  title={Definizioni},
  label={bcdr1}
 ]

 \Question L'ammontare di dati che si puó perdere dal un malfunzionamento di 
un computer si definisce come:
\begin{enumerate}
  \item Recovery Time Objective
  \item Recovery Point Objective
  \item Service Delivery Objective
  \item Maximum Tolerable Outage
\end{enumerate}

\end{Exercise}

\begin{Answer} [
   ref={bcdr1},
   number={1}
 ]

  \Question Risposta esatta: 2

\end{Answer}

% Altro esercizio

\begin{Exercise} [
  title={RTO Largo},
  label={bcdr2}
 ]
 
 \Question Quanto l'RTO è ``largo'', viene solitamente associato con:
\begin{enumerate}
  \item Applicazioni critiche
  \item Una strategia di recupero veloce
  \item Servizi di tipo \textit{sensitive} o \textit{nonsensitive}
  \item Un estensivo piano di ripristino
\end{enumerate}

\end{Exercise}

\begin{Answer} [
   ref={bcdr2},
   number={2}
 ]

  \Question Risposta esatta: 4

\end{Answer}

% Altro esercizio

\begin{Exercise} [
  title={RTO Stretto},
  label={bcdr3}
 ]
 
 \Question Quando l'RPO è molto breve, la soluzione migliore è:
 \begin{enumerate}
  \item Cold site
  \item Mirroring dei dati
  \item Un piano di recupero del disastro dettagliato ed efficente
  \item Un accurato piano per la continuazione del business
 \end{enumerate}

\end{Exercise}

\begin{Answer} [
  ref={bcdr3},
  number={3}
 ]

 \Question Risposta esatta: 2
\end{Answer}