\chapter{Business Continuity and Disaster Recovery}
\label{EsBCDR1}

Teoria disponibile al Capitolo \ref{BCDR}

\begin{Exercise} [
  title={Definizioni},
  label={bcdr1}
 ]

 \Question L'ammontare di dati che si puó perdere dal un malfunzionamento di
un computer si definisce come:
\begin{enumerate}
  \item Recovery Time Objective
  \item Recovery Point Objective
  \item Service Delivery Objective
  \item Maximum Tolerable Outage
\end{enumerate}

\end{Exercise}

\begin{Answer} [
   ref={bcdr1},
   number={1}
 ]

  \Question Risposta esatta: 2

\end{Answer}

% Altro esercizio

\begin{Exercise} [
  title={RTO Largo},
  label={bcdr2}
 ]

 \Question Quanto l'RTO è ``largo'', viene solitamente associato con:
\begin{enumerate}
  \item Applicazioni critiche
  \item Una strategia di recupero veloce
  \item Servizi di tipo \textit{sensitive} o \textit{nonsensitive}
  \item Un estensivo piano di ripristino
\end{enumerate}

\end{Exercise}

\begin{Answer} [
   ref={bcdr2},
   number={2}
 ]

  \Question Risposta esatta: 4

\end{Answer}

% Altro esercizio

\begin{Exercise} [
  title={RTO Stretto},
  label={bcdr3}
 ]

 \Question Quando l'RPO è molto breve, la soluzione migliore è:
 \begin{enumerate}
  \item Cold site
  \item Mirroring dei dati
  \item Un piano di recupero del disastro dettagliato ed efficente
  \item Un accurato piano per la continuazione del business
 \end{enumerate}

\end{Exercise}

\begin{Answer} [
  ref={bcdr3},
  number={3}
 ]

 \Question Risposta esatta: 2
\end{Answer}

% Altra sezione esercizi
\label{EsBCDR2}

\begin{Exercise} [
  title={Azioni da intraprendere},
  label={bcdr4}
 ]

 \Question La prima cosa che dovrebbe essere fatta quando c'è stata un
 intrusione nel tuo computer sarebbe:
 \begin{enumerate}
   \item Disconnettere tutti i computer dalla rete dell'azienda, sperando di
   disconnettere anche l'attaccante
   \item Spegnere il server per prevenire possibili perdite di confidenzialità
   e integrità dei dati
   \item Chiamare il manager
   \item Seguire le direttive del piano di risposta ad un incidente
 \end{enumerate}

\end{Exercise}

\begin{Answer} [
  ref={bcdr4},
  number={4}
 ]

 \Question Risposta esatta: 4
\end{Answer}

% Altro esercizio

\begin{Exercise} [
  title={Domanda su BCP},
  label={bcdr5}
 ]

 \Question Durante un audit del piano di continuità del business, la scoperta
 più sconcertante sarebbe:
 \begin{enumerate}
   \item L'albero delle chiamate non è stato controllato in maniera debita
   negli ultimi sei mesi
   \item L'analisi sull'impatto del business non è stata aggiornata quest'anno
   \item Un test del sistema di backup e recupero non è stato eseguito
   regolarmente
   \item Il sito di backup manca di un UPS
 \end{enumerate}

\end{Exercise}

\begin{Answer} [
  ref={bcdr5},
  number={5}
 ]

 \Question Risposta esatta: 3

\end{Answer}

% Altro esercizio

\begin{Exercise} [
  title={Quiz},
  label={bcdr6}
 ]

 \Question Il primo e più importante test per verificare il business
 continuity plan è:
 \begin{enumerate}
   \item \textit{Fully operational test}
   \item \textit{Preparedness Test}
   \item \textit{Security Test}
   \item \textit{Desk-Based paper test}
 \end{enumerate}

\end{Exercise}

\begin{Answer} [
  ref={bcdr6},
  number={6}
 ]

 \Question Risposta esatta: 4

\end{Answer}

% Altro esercizio

\begin{Exercise} [
  title={Quiz},
  label={bcdr7}
 ]

 \Question Quando accade un disastro, qual è la massima priorità:
 \begin{enumerate}
   \item Assicurarsi che tutte le persone siano in salvo
   \item Minimizzare la perdita dei dati salvando i dati importanti
   \item Recuperare i nastri di backup
   \item Chiamare un \textit{manager}
 \end{enumerate}

\end{Exercise}

\begin{Answer} [
  ref={bcdr7},
  number={7}
 ]

 \Question Risposta esatta: 1

\end{Answer}

% Altro esercizio

\begin{Exercise} [
  title={Quiz},
  label={bcdr8}
 ]

 \Question Un processo documentato dove si determina l'operazione
 IT \emph{più} cruciale dalla prospettiva del business.
 \begin{enumerate}
   \item \textit{Business Continuity Plan}
   \item \textit{Disaster Recovery Plan}
   \item \textit{Restoration Plan}
   \item \textit{Business Impact Analysis}
 \end{enumerate}

\end{Exercise}

\begin{Answer} [
  ref={bcdr8},
  number={8}
 ]

 \Question Risposta esatta: 4
\end{Answer}



% Altro esercizio

\begin{Exercise} [
  title={Quiz},
  label={bcdr9}
 ]

 \Question L'obiettivo principale di un Post-Test è:.
 \begin{enumerate}
   \item Scrivere un report a scopo di audit
   \item Ritornare al servizio normale
   \item Valutare l'efficacia del test e aggiornare il \textit{response
   plan}
   \item Fare un resoconto sul test al management
 \end{enumerate}

\end{Exercise}

\begin{Answer} [
  ref={bcdr9},
  number={9}
 ]

 \Question Risposta esatta: 3
\end{Answer}

% Altro esercizio

\begin{Exercise} [
  title={Quiz},
  label={bcdr10}
 ]

 \Question Un test che verifica che il sito alternativo può processare
 le transazioni con successo è conosciuto anche come:
 \begin{enumerate}
   \item \textit{Structured walkthrough}
   \item \textit{Parallel Test}
   \item \textit{Simulation Test}
   \item \textit{Preparedness Test}
 \end{enumerate}

\end{Exercise}

\begin{Answer} [
  ref={bcdr10},
  number={10}
 ]

 \Question Risposta esatta: 2
\end{Answer}
