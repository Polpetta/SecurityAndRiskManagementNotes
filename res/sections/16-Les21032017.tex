\chapter{Access Control Techniques}
\begin{itemize}
 \item Mandatory Access Control: General (system-determined) access control. 
 Specifica di chi può accedere a cosa.
 \item Discretionary Access Control: persone che hanno l'autorizzazione ad 
 accedere. Si trova spesso dove l'informazione risulta essere classificata
 \item Role-Based Access Controll: è la soluzione migliore per il contesto 
 organizzativi classici. Essendoci un numero fissato di ruoli per 
 l'amministratore di sistema è più semplice perché può concentrarsi sui ruoli 
 anziché sul singolo impiegato. Questa realtà è ancora in forte sviluppo.
 \item Physical Access Control: è quell'area legata alla sicurezza fisica, ed è 
 legata al CSO ultimamente. Si costituisce di lucchetti, chiavi, ecc...
\end{itemize}


\section{Role-Based Access Control}
%TODO copia tabella

Prima di mettersi a fare controlli sugli accessi bisogna prima comoprendere per 
bene che ruoli sono presenti all'interno dell'azienda (analisi interna).


\section{Military Security Policy}

Si applica una classificazione dell'informazione, che può essere:
\begin{itemize}
 \item Top secret
 \item Secret
 \item Confidential
 \item Non-classified
 
\end{itemize}

Per evitarel'overclassification si utilizzano i domini: si cercano di mettere 
delle persone con determinate competenze nel contesto giusto.
Ad esempio: se un dominio è il nucleare e io non ho competenze a riguardo non 
posso accedere a quel dominio.

Un soggetto può accedere ad un oggetto solo se domina la classificazione 
dell'oggetto.

\section{Modello Bell-La Padula}

Si è dimostrato che adottando questo modello possiamo avere buone proprietà di 
confinamento dell'informazione.

Utilizza il principio di confidenzialità: si può avere accesso in lettura a 
tutto ciò che dominiamo. Per la scrittura invece vale il contrario: posso 
accedere in scrittura a tutto ciò che mi domina ma non posso scrivere a ciò che 
domino. Il tutto viene riassunto in \textit{no read up} e \textit{no write 
down}.

Principio di tranquillità: le classi degli oggetti non possono cambiare.


La declassificazione risolve il problema del galleggiamento verso l'alto. Avere 
troppi documenti top-secret può causare problema. Eseguire una 
declassificazione è una procedura costosa, ed è quando il livello di segretezza 
di un certo documento viene abbassato.

\section{System Access Control}

\begin{itemize}
 \item eseguire il log degli eventi
 \item Report all'amministrazione di sistema dei tentativi falliti di accesso, 
 che possono essere anche accidentiali.
\end{itemize}

\section{Application-Level Access Control}

\begin{itemize}
 \item creare/cambiare file o struttura del database
 \item autorizzare azioni al livello di:
  \begin{itemize}
   \item applicazione
   \item file
   \item transazione
   \item campo
  \end{itemize}
\item log di accessi alla rete e ai dati
\end{itemize}

\subsection{Password}

\subsection{Tutto quello che avreste voluto sapere ma non avete mai osato 
chiedere sulle password}

Cos'è una password? Una sequenza arbitraria di caratteri segreta e difficile da 
indovinare.

Ancora al tempo di Venezia era presente addirittura una sezione di cifratura 
apposita. Oggi, l'NSA è l'istituzione che ha più matematici al mondo.

\subsubsection{Gestione delle password}

Le password servono a discriminare chi ha diritto o no ad accedere al servizio.
Le password nei sistemi vengono solitamente cifrate e salvate, in questa 
maniera non è presente la password in chiaro. Per eseguire ciò di solito si 
eseguono delle tecniche di \textit{hashing}.

\subsubsection{Difficoltà delle password}

Con una password da N caratteri, in cui ogni carattere può assumere M valori, 
abbiamo $N^M$ possibili password.

\subsubsection{Come creare una buona password}

Le password scelte dagli utenti non sono solitamente forti, ma tendono ad 
essere deboli e ripetitive. La bontà di una password dipende anche dalla sua 
\textit{entropia}.
Le password solitamente sono simili a quelle di nomi di persone, date di 
nascite o numeri di telefono.

\paragraph*{Base words}
Le \textit{base-words} passwords sono password significative dal punto di vista 
memnoniche che risultano essere difficile da indovinare da un attaccante perchè 
per un estraneo risultano come se fossero una combinazione casuale di lettere. 
È un metodo molto valido per costruire una forte password.

\paragraph*{Facciamo i conti}

Il dizionario inglese ha 170 mila parole, vuol dire che se combiniamo 2 parole 
abbiamo 29 miliardi di tentativi per indovinare per psw. Il problema è che le 
GPU moderne possono provare fino a 40 miliardi di psw al giorno.
È possibile attaccare oggi come oggi sistemi molto difesi tramite attacchi di 
forza bruta.
$2^80$ è il limite teorico oltre il quale un sistema è considerato 
inattaccabile. Una psw di 10 caratteri con 46 possibili caratteri ha circa 
$2^55$ possibilità.

\paragraph*{Attacco del dizionario}
Le password sono solitamente deboli all'attacco del dizionario: ovvero sono 
password che hanno al loro interno pattern di persone/cose (come ad esempio il 
nome di una serie TV o simili). I dizionari al giorno d'oggi sono addirittura 
comprabili, e presentano tutte le entry del determinato campo semantico.
È meglio quindi evitare l'uso di password che abbiano pattern al loro interno. 
È meglio la generazione di una password totalmente casuale.

\paragraph{Autenticazione a due fattori}

L'autentificazione viene detta forte quando è ad almeno due valori. 
Autentificazione forte vuol dire usare almeno due canali differenti.

Ad esempio al giorno d'oggi durante un'autenticazione bancaria si richiedono 
solitamente un codice e un altro codice proveniente da una chiavetta/sms/token 
hardware.

Il tutto si basa su questi tre puntiavetta) e qualcosa che siamo (biometrico. 
per eseguire una corretta autenticazione:
\begin{itemize}
 \item Qualcosa che siamo
 \item Qualcosa che abbiamo
 \item Qualcosa che sappiamo
\end{itemize}

Per avere un'autenticazione forte devono essere presente almeno due di queste 
tecniche. Questo funziona bene perchè almeno due canali devono essere 
compromessi.

Una nuova tecnica è anche quella della autorizzazione tramite 
geolocalizzazione. Sempre nell'ambito bancario, pagamenti eseguiti a breve 
tempo da diverse parti del mondo non vengono autorizzati.

\subsection{Single Sign On}

Vantaggi:
\begin{itemize}
\item una buona password rimpiazza molte password
\item ID consisente tra i sistemi
\item riduce il lavoro degli admin nel setup e per le password dimenticate
\item accesso rapido ai sistemi
\end{itemize}


Il Single Sign On serve per poter accedere a una serie di servizi offerti dal 
provider. Con il SSO è il segmento con cui ci si autentica per primi che 
fornisce l'autorizzazione anche per le altre applicazioni.
La sicurezza che il SSO offre non è la stessa che la segmentazione dei servizi 
offre.
Svantaggi:
\begin{itemize}
\item %TODO finiscimi
\end{itemize}

\subsection{Multi-factor authenticator}

\subsection{Biometria}

È molto utile per eseguire autenticazioni, ma funziona ancora male in contesti 
quali per esempio le riprese di sicurezza.

Chi sei e cosa fai, è suscettibile ad errori.

\begin{itemize}
\item False Rejection Rate: falso negativo, un reject che non dovrebbe esserci 
stato
\item False Acceptance Rate: falso positivo, un accept quando non avrebbe 
dovuto esserci
\item Failure to Enroll Rate: numero di utenti che non sono riusciti a 
registrarsi correttamente
\end{itemize}

\subsubsection{Metodi migliori per l'autenticazione}

% TODO copiare la tabella.

\subsubsection{Biometric Info Management \& Security (BIMS) Policy}

\begin{itemize}
 \item Procedure di identificazione e autenticazione
 \item Backup delle autenticazioni (es. non chiedere l'autenticazione ogni 
 volta)
 \item Trasmissione/conservazione sicura dei dati biometrici
 \item Sicurezza fisica dell'hardware
 \item Test di valutazione
\end{itemize}

I dispositivi per la cattura del viso per esempio stanno in aree aperte, e se 
un attaccante è in grado di manomettere il dispositivo ha la possibilità di 
inserire un dispositivo di acquisizione personale, per poter registrare i dati 
e ottenerli in maniera non autorizzata. In questa maniera l'attaccante è in 
grado di eseguire un furto d'identità. Questi strumenti dovrebbero essere posti 
in aree controllate, cosa che spesso non succede. È importante che l'integrità 
di questi strumenti venga manetenuta.

\subsection{Verifiche di un IS Auditor}
\begin{itemize}
 \item Verificare che ci siano delle procedure
 \item Che siano implementate
 \item che i processi seguano il pattern del \textit{need to know}
 \item security awareness
 \item L'owner e il custode devono sapere di essere responsabili di quello che 
 fanno e 
 \item Security Adimistrator deve forni
 \item I meccanismi di sicurezza devono essere aggiornati e consistenti con la 
 realtà
 \item I servizi amministrativi devono essere consistenti, e devono provvedere 
 sicurezza fisica e logica
\end{itemize}

% ESERCIZI
\section{Esercizi}

A form of biometrics that is considered invasive by users is:
\begin{itemize}
 \item Retina (risposta esatta)
 \item Iris
 \item 3D hand
 \item Signature
\end{itemize}

La forma di biometria che è meno invasivo

\begin{itemize}
\item 
\item
\item
\item 
\end{itemize}

Julie is a Data Owner. She configures permissions in the database to enable 
users to access the forms she thinks they should be able to access. This 
technique is known as

\begin{itemize}
\item 
\item
\item
\item Discretionary (corretta)
\end{itemize}

John has a secuirty clearance of (Engineering, Confidential). Using Bell and La 
Padula Model, John can write to

\begin{itemize}
\item Confidential
\item Top secret, secret and Confidential (corretta)
\item Confidential and Unclassified
\item Unclassified
\end{itemize}