\subsubsection{RSA}

Si basa sull'esponenziazione, e presenta dei parametri pubblici e dei parametri 
privati.


La chiave pubblica solitamente si indirizza con $e$, mentre quella privata con 
$P$(\todo{d???}). È necessario eseguire:

\[ C = P^e\ mod\ n \]

Per decriptare:

\[ P = C^d\ mod\ n \]

dove $n = p \cdot q$ con $|p| \sim 1$Kb.

Questo avviene grazie alle proprietà dell'RSA.

La modalità del RSA che permette dalla composizione di messaggi non è sicura 
perché

\[ c_1 \cdot c_2 = {p^e}_1 \cdot {p^e}_2 = (p_1 \cdot p_2)^e\]


Si basa su una proprietà molto forte che viene definita comee 
\textit{Malleability}, molto utile soprattutto nelle firme, perchè garantisce 
l'autenticità di chi manda i messaggi.


\paragraph{Funzionamento}

Si calcolano due primi grandi, si prende $\Phi$ dove rappresenta il numero 
coprimi con $n$.
Esiste la proprietà per cui $\Phi(n) = (p-1) \cdot (q-1)$.

La chiave privata è l'inverso della chiave pubblica $mod\ n$ ($d = e^{-1}  
\text{ mod } \Phi(n)$).

Assunzioni per l'RSA:
\begin{itemize}
	\item \todo{Dopo sistemo} $\Phi(n) = (p-1)(q-1)$ senza $p$ e $q$ è 
oneroso calcolare $\Phi(n)$ e quindi se si riuscisse a fattorizzare $n$ si 
potrebbe trovare $p$ e $q$ e quindi si basa sul fatto che è difficile fare la 
fattorizzazione.
    \item \todo{Qui? Non ha più detto la seconda assunzione dell'RSA - 
FeelsBadMan:(}
\end{itemize}


Come si sceglie $e$? La $e$ può essere scelta a piacere e di solito viene scelto 
$3$, in quanto è primo e siamo ragionevolmente certi che può essere reversibile. 
Purtroppo è stavo verificato che è debole e quindi il numero è stato cambiato.

\subsubsection{Conclusioni}

Vantaggi:
\begin{itemize}
\item Risolve il problema della gestione delle chiavi
\item Basato su problemi matematici
\item Relativamente recente
\end{itemize}

Svantaggi:
\begin{itemize}
\item Insicuro contro il quantum computing
\item Computazionalmente pesante
\end{itemize}


\section{Network defense}

\subsection{Difesa in profondità}

Le difese in profondità comprendono diversi strati di protezione, per prevenire 
attacchi dall'esterno ma che potrebbero avvenire anche dall'interno.

\subsubsection{Bastion Host}

Pur essendo un'architttettura vecchia, viene ancora utilizzata.
Computer fortificato contro gli attaccanti:
\begin{itemize}
\item applicazioni disattivate
\item sistema operativo patchato
\item Configurazioni di sicurezza strette
\end{itemize}

\subsection{Attaccando la rete}\footnote{\textit{Black Hat} è anche un termine 
per distinguere gli attaccanti.}

Quali sono i punti d'accesso in una rete? Inizialmente, potrebbe essere da una 
chiavetta USB (vedi Stuxnet).

\textbf{Tutto ciò che ha un contatto con l'esterno può infettare la rete.}

Come ci si può quindi difendere da ciò?
\paragraph*{DMZ} Innanzitutto, è consigliabile usare una DMZ, 
\textit{De-Militarized Zone}, per separare la rete commerciale da quella 
interna.



\paragraph*{Filtri}

Il filtro serve a filtrare le connessioni. Livello 0 della sicurezza

\begin{itemize}
\item Route filter: sorgente e destinatario
\item Packet filter: pacchetto
\item Content filter: contenuto
\end{itemize}

\subparagraph*{Packet Filter Firewall}

%todo
\todo{Qui c'è un TODO ma non son sicuro riguardo a cosa}

La modalità di configurazione di un firewall comporta vantaggi (capacità 
inspettiva) e svantaggi (lentezza della rete). È un bilanciamento tra lentezza 
della porta e capacità ispettiva.

Esistono diverse configurazioni per il \textit{firewalling}:
\begin{itemize}
 \item Router Packet Filtering: analisi sul singolo pacchetto, non c'è memoria 
dei precedenti
 \item Stateful Inspection: non guardiamo il singolo pacchetto ma la trama dei 
vari pacchetti. Causa un po' di overhead
 \item Circuit-Level Firewall: viene utilizzato un \textit{Proxy Server} che 
funge da intermediario e porta la risposta. Funge da mediatore, con spesso la 
capacità di \textit{screening}.
 \item Application-Level Firewall: tutte le applicazioni che accedono ad 
internet vengono gestite dal proxy
\end{itemize}

\paragraph*{Multi-Homed Firewall}

Vuol dire che ci sono due uscite: una che va verso la DMZ e una verso la rete 
privata.

VPN significa \textit{Virtual Private Network}, che permettono alle macchine di 
essere virtualmente sulla stessa rete. Tutte le grandi società hanno una propria 
VPN.

\paragraph*{Regole di sicurezza per gli accessi}

\todo{Copia slide - Writing Rules}

Le regole derivano innanzitutto dalle policy di sicurezza. L'altro input per 
dare le regole è il mondo del reale (capacità di filtering). Dopo che le regole 
sono state scritte sicuramente possiamo fare PDCA\footnote{Detto anche 
\textit{Ciclo di Deming}} per migliorare le regole. Le policy vanno aggiornate, 
in base alla posizione aziendale (più o meno aggressiva) e ai breakthrough 
tecnologici (es. introduzione del cloud computing).



\subsection{Canali di accesso logico}

Sono i canali tramite i quali è possibile accedere a una rete.

Esistono diversi canali di accessi, che sono per esempio:
\begin{itemize}
\item Wlan (la seconda più pericolosa perché vi si può accedere da fuori 
l'azienda) \todo{Dipende dove è messa. O ho capito male?}
\item Usb (la più pericolosa)
\item Internet (non così pericolosa perché c'è il firewall)
\end{itemize}

È importante quindi che anche queste entrate vengano protette.

\todo{Ha messo una tabella ma è per fare un esempio}
\paragraph*{Esempio di applicazioni di una scuola}
\todo{Copiare tabella}

\subsection{Intrusion Detection Systems \& Intrusion Prevention Systems}

Detti anche IDS e IPS sono dei sistemi che aiutano la protezione del sistema. 
Gli HIDS sono detti anche \textit{Host IDS} e sono IDS installati nella 
macchina. Questi sistemi purtroppo non hanno una mappa completa del traffico di 
rete, e non riescono a proteggere in maniera efficace il computer: c'è un 
problema di falsi positivi: il sistema che si sta facendo qualcosa di scorretto 
perché il comportamento devia dallo standard, ma non è necessariamente un 
comportamento negativo.

\subsubsection{IDS Intelligence Systems}

Basati sulle firme: pattern specifici vengono riconosciuti come attacchi. 
Problema dei falsi positivi.

Basati sulla statistica: viene capito il comportamento atteso del sistema. Se 
c'è una variazione, potrebbe esserci un attacco.

Basati su reti neurali: base statistica con auto-apprendimento.

\subsection{Honeypot \& Honeynet}

Sostanzialmente sono delle ``trappole''. In un \textit{honeypot} si vuole 
indurre l'attaccante ad entrare in un certo sistema, mentre l'\textit{honeynet} 
è un modo per poter studiare l'attaccante tramite la simulazione di una rete che 
possa rappresentare del valore per l'attaccante e hanno un buon 
funzionamento\footnote{Vengono spesso utilizzate dalle università}.

\subsubsection{Data privacy}

\begin{itemize}
\item Confidenzialità
\item Autenticità
\item Integrità
\item Nonrepudation: non c'è la possibilità di dire "\textit{non l'ho inviato io 
quel messaggio}" (firma digitale). Le chiavi simmetriche non garantiscono questa 
proprietà e quindi non si possono impelementare per fare pagamenti online..
\end{itemize}

\subsubsection{Remote Access Security}

Le VPN vengono spesso implementate con IPSec \todo{cos'è IPSec?}

\subsubsection{Secure Hash Functions}

L' hash è una funzione matematica che prende in input una stringa di idmensione 
aribtraria e restituisce una stringa di dimensione fissa. La sicurezza di oggi è 
basata sugli \textit{hash} Presenta diverse proprietà:
\begin{itemize}
\item La non invertibilità (\emph{one way hash})
\item Collision resistance: è difficile trovare una coppia $(x_1,x_2)$ tale che 
$h(x_1) = h(x_2)$ con $h$ funzione di hash\footnote{Grazie a questa proprietà si 
possono eseguire le firme digitali}.
\end{itemize}

\paragraph*{Utilizzo}

Per utilizzare gli hash si può prendere il messaggio $M$, eseguire l'hash di $M$ 
($h(M)$) e appendere questo risultato alla fine, avendo quindi $payload = M + 
h(M)$. A questo punto, è possibile inviare il messaggio. Il destinatario dovrà 
solamente eseguire l'hash del messaggio per verificare che il messaggio sia 
stato correttamente inviato.

\paragraph*{HMAC\footnote{\url{https://it.wikipedia.org/wiki/HMAC}}}

È un RFC specifico, riconosciuto come standard e utilizzato nell'invio dei 
messaggi.


\subsubsection{Firma elettronica}

Come si fa a firmare un messaggio? Si prende il messaggio e lo si esponenzia 
alla chiave privata. Chi lo riceve utilizza la chiave pubblica per decriptare il 
messaggio. In questo modo abbiamo la non-repudiabilità.

Mandiamo una coppia $ \langle m, Sig(h(m)) \rangle $ quando mi arriva un 
messaggio devo controllare che
$h(m) = h(m)$ in modo da verificare che il messaggio sia integro.
