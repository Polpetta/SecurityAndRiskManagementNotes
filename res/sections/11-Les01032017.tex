\subsection{Vocabolario dei termini}

Una lista di vocaboli utili:
\begin{itemize}
  \item \textbf{Skimming:} si rubano i fondi prima che vengano registrati;
  \item \textbf{Cash Larceny:} atto di rubare soldi ad una compagnia 
  dopo che sono stati registrati;
  \item \textbf{Lapping:} quando si è in debito a causa di un prestito,
  si ricorre ad un altro prestito per pagare quello precedente. Questo 
  ``gioco'' non è possibile che sia portato avanti all'infinito, in 
  quanto le banche hanno un tasso di interesse;
  \item \textbf{Check Tampering:} falsificazione di carte per aumentare il proprio 
  guadagno (es. falsificazione degli assegni);
  \item \textbf{Shell Company:}\footnote{Fenomeno presente soprattutto nel nord-est 
  Italia} quando un'attività acquista qualcosa, è possibile detrarre le tasse. Quindi è
  possibile creare un'azienda vera ma inesistente, il cui scopo è di vendere
  beni ad un'altra azienda. Viene detto in italiano cartiera in quanto il suo 
  solo scopo è di fare ``carte'';
  \item \textbf{Payroll Manipulation} (Manipolazione del libro paga):
  Impiegati fantasma, ore di lavoro falsificate, tempo di vacanza
  o di assenza diminuito artificialmente sul libro paga;
  \item \textbf{Fraudulent Write-off}: Asset utili vengono marchiati
  come scarti;
  \item \textbf{Collusione}: Due o più dipendenti o dipendenti e 
  fornitori si mettono d'accordo per frodare l'azienda insieme;
  \item \textbf{False Shipping Orders} o \textbf{Missing/Defective 
  Receiving Record}: Furto dall'inventario.
\end{itemize}

\subsection{Considerazione legale}

Ovviamente l'azione di frode non è etica e dev'essere presente il dolo. Produce 
un danno alla controparte.

\subsection{Elementi chiave della frode}

Per bloccare una frode è sufficiente bloccare uno di questi tre elementi:
\begin{itemize}
  \item Motivazione;
  \item Opportunità;
  \item Giustificazione.
\end{itemize}

\subsubsection{Motivazione}

La frode ha sempre una motivazione dietro: può essere necessaria (per esempio
si muore di fame) o percepita (una persona ricca vuole un Ferrari al posto di
una BMW). Questi vengono detti \textbf{enabler} e sono le condizioni messe in
``and'' per far scattare la frode.

\subsection{Come una frode viene scoperta}

Il 35\% delle frodi vengono scoperte grazie alle \textbf{soffiate}. In ordine,
le percentuali sono:
\begin{enumerate}
  \item Soffiata (35\%) (es. dei dipendenti notano che c'è un aumento di ricchezza 
  improvviso da parte di un collega);
  \item Per caso (25\%): dato dalla casualità;
  \item Audit interni eseguiti dal CISA\footnote{Internal Audit} (20\%);
  \item Controlli interni (18\%);
  \item Audit esterni (12\%): un esterno esegue controlli per verificare se 
  sono presenti frodi che circuiscono i controlli interni;
  \item Notifica dalla polizia (4\%).
\end{enumerate}
Va notato che la somma delle percentuali non da 100, ma un numero maggiore
perchè alcune frodi vengono riportate attraverso più metodi di reporting. 

Ultimamente anche i fornitori costituiscono una fonte per scoprire l'avvenire 
di una frode.

\subsection{Azioni da intraprendere dopo che una frode viene scoperta}

Alcune azioni possono essere: azioni disciplinari, licenziamento e/o azioni 
civili/legali.

Le frodi spesso non vengono riportate per i seguenti motivi:
\begin{itemize}
  \item A causa di una brutta pubblicità (42\%): questo purtroppo è un grosso 
  incentivo alla frode, in quanto il malfattore
  saprà già che non verranno presi provvedimenti legali;
  \item I provvedimenti disciplinari interni sono spesso sufficienti (32\%);
  \item Accordi privati (30\%);
  \item Costi troppo alti per perseguire l'atto (21\%): a volte l'atto è troppo 
  costoso per essere perseguito da un avvocato o un
  gruppo di avvocati. Nelle grandi aziende è presente un ufficio di avvocati
  sempre a disposizione, per poter agire all'avvenire di queste situazioni.
\end{itemize}

\subsection{Profilo del fraudolento}

Le frodi più costose vengono eseguite solitamente da persone che sono presenti
ai piani alti dell'azienda (ad esempio da un \textit{Executive}\footnote{Dirigenti 
di fascia alta}). Questi commettono frodi che sono in media una decina di volte
più ingenti di quelle effettuate da un impiegato normale. Uomini e donne presentano 
lo stesso livello di fraudolenza, ma le frodi più ``grosse'' vengono commesse dagli 
uomini: questo perché non è presente un'adeguata rappresentazione femminile ad 
alti livelli.\\
\newline
Media per sesso: 250 \euro uomini contro 120 \euro donne.\\
\newline
Le frodi, specialmente quelle più ingenti, sono eseguite da persone che hanno la fedina
penale pulita, quindi cercare il colpevole di una frode nelle persone che hanno commesso
 crimini in precedenza è sconsigliato. Anche perché questi ultimi molto spesso ricoprono 
un ruolo di grado medio/basso nella gerarchia aziendale (probabilmente a causa della 
loro istruzione povera).

Le veri frodi, quelle che sono costate molto anche a livello di immagine, sono
quelle che hanno un alto grado di \textbf{collusione}, ovvero un gran numero di
persone si mettono in combutta assieme per eseguire la frode. Un esempio 
reale può essere il caso Parmalat oppure il timbro di cartellini lavorativi 
illeciti. La collusione di personale sullo stesso livello non è troppo grave, 
molto critica è invece un livello di collusione che spazia in diversi gradi 
dell'azienda.

\subsection{Esercizi}

Gli esercizi sono disponibili alla sezione \ref{EsFrodi}.

\section{Come scoprire una frode}

\subsection{Audit}

L'audit verifica che le cose vadano com'è stato dichiarato, quindi non è
possibile eseguire \textit{detection} di frodi tramite questo tipo di controllo.

\subsection{Red flags di una frode}
Sono presenti segnali importanti quando un utente sta commettendo una frode,
come per esempio un aumento improvviso della sua ricchezza, manifestato
in un cambio drastico dello stile di vita.

È importante quindi analizzare il profilo di una persona per scoprire se può
essere soggetta a commettere un crimine di tale genere. Elementi che possono 
indicare che una persona commetterà o sta commettendo una frode sono:
difficoltà finanziare, problemi legali, se ha già commesso questo atto in 
passato (eseguire un \textit{background check}), ego - voglia di ``rompere'' 
le regole (ad esempio ``bucare'' un firewall, entrare in sistemi non 
autorizzati, ecc) o problemi in famiglia (es. divorzio).

\subsection{Habits di un fraudolento}

Questi sono i punti chiave che accomunano tutti i soggetti fraudolenti:
\begin{itemize}
  \item Giustificazione di povere abitudini lavorative;
  \item Tentativi disperati di raggiungere gli obiettivi lavorativi;
  \item Atteggiamento super-protettivo per certi tipi di documenti (o il 
  soggetto non ha voglia di condividere o rilasciare la documentazione);
  \item Rifiuto di cambiare doveri lavorativi;
  \item Orari di lavoro costantemente non standard (il soggetto inizia a 
  lavorare tanto presto o finisce molto tardi).
\end{itemize}

Anche le \textbf{transazioni} finanziarie possono essere dei segnali da non 
sottovalutare. In particolare bisogna prestare attenzione a:
\begin{itemize}
  \item Transazioni troppo frequenti o infrequenti, o fuori orario;
  \item Ammontare non usuale: troppo o troppo poco (detta anche
  \textbf{tecnica del salame});
  \item Partecipazioni insolite: include sconosciuti o molte parti molto vicine;
  \item Le consegne ricevute non vengono consegnate;
  \item Supervisione insufficiente;
  \item Piccole correzioni agli acconti;
  \item Indirizzi diversi per lo stesso venditore, o utilizzo di venditori con 
  nomi simili.
\end{itemize}
