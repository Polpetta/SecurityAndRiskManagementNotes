\chapter{Application Controls}
\label{cs:ac}

Esame: descrizione di un caso + poche domande

\todo{Era la descrizione dell'immagine con un sacco di freccette}
Abuse case (invece di use case) sono i tipici casi in cui il sistema può essere
flesso in chiara violazione delle policy di sicurezza.

Code review consiste anche nell'analisi statica.

Test sul software: penetration test, stress test

\paragraph{Transaction Validation}

\paragraph{Batch Processing} Elaborazione sequenziale di blocchi di transazioni
dopo che sono state aggregate. \todo{Insert Image}

\paragraph{Transaction Authorization}
Manuale

Automatico

\paragraph{Error Handling Alternatives}
Rigettare le transazioni con errori e continuare a processare il batch;
Stoppare tutto il batch



\subparagraph{Data Processing}

\todo{copia schema}

Il processing comprende controllo dei parametri di sistema (es. specificare
minimo e massimo), Standing data (file permanenti), report delle eccezioni
(errori nelle transazioni), log delle transazioni, file delle transazioni
(giornaliero).

\paragraph{Processing controls}

Per-Transaction Basis \todo{completare}
\begin{itemize}
\item Editing
\item Controlli sugli ammontare calcolati
\item Controlli programmati
\item Report delle eccezioni
\end{itemize}

Per-Batch Basis

\begin{itemize}
\item Batch Register:
\item Run-To-Run Basis
\item
\end{itemize}

\paragraph{Controllo sui dati}

\todo{copiare l'elenco}

Source document retention: la gestione dei documenti è un'attività molto onerosa
perché richiede che ci sia una organizzazione in piedi.

Internal \& External Labeling: c'è il backup, bisogna etichettarli in modo da
trovare quello giusto.

\section{Esercizi}

Gli esercizi sono disponibili in \ref{esCs:ac}

\chapter{Application Audit}
\label{cs:aa}

I task degli auditor delle applicazioni. Bisogna andare a testare le
applicazioni che sono significative per l'azienda. Tutte è impossibile.
Testare i controlli. Analisi del risultato devo estendere il controllo è
inutile, è oneroso e così via.

\todo{Insert image: Integrated testing facilities}


\todo{Insert image parallel operation}
si fa quando si vuole aggiornare i sistemi in produzione.

\todo{Insert image Embedded}
Embedded Audit Modules (EAM): il suo grande svantaggio È quello di rallentare il
sistema. Qualora vada tutto bene sono soldi spesi inutilmente.
Systems Control Audit Review File (SCARF):
Sample Audit Review File (SARF) scelta a campione non è più la scelta migliore.

%%%%%VIETATO PRANZARE PRIMA DI SRM
\paragraph{Testing Application Techniques}
\todo{copia elenco}
Validate Systems $\Leftarrow$ Importanti Parallel Simulation e Parallel
Operation



\paragraph{Tecniche di Auditing Onine}
Audit Hooks:all'interno del codice ci sono delle condizioni particolari che se
sono soddisfatte triggerano delle operazioni di auditing, è meno pesante delle
EAM infatti viene attivato il controllo SOLO quando si raggiungono certi path
del CFG.
\todo{Copia  tabella}
\todo{Copia disegnino di Hook}

\section{Esercizi}

Gli esercizi sono disponibili in \ref{esCs:aa}

\part{Incident process process forensics}

Un incidente relativo alla sicurezza può capitare per molti problemi. Un
incidente è catalogato come tale quando c'è una violazione della \textit{policy}
di sicurezza. Un incidente può essere di due tipologie:
\begin{itemize}
\item Malizioso
\item Casuale (non malizioso)
\end{itemize}

\chapter{Incident Response vs Business Contiuity}
\label{IRBC}

Avere un piano di risposta agli incidenti fa parte della \textit{business
continuity plan}.

\section{Business Continuity Plan}

\subsection{Terminologia della \textit{recovery}}
\todo{Add refernce to old image --> RECOVERY TERMS}
Esiste una determinata tipologia di termini:
\begin{itemize}
\item Interruption Window
\item Service Delivery Objective
\item ?? \todo{Aggiungimi :(}
\end{itemize}

In particolare, esistono diversi ruoli che vedremo in dettaglio.

\paragraph*{IMT} L'IMT, che significa \textbf{Incident Management Team}, fa
parte della \textit{steering commitee} e sono membri dell'IRT. % va troppo
veloce




\paragraph*{IRT} L'IRT, che significa \textbf{Incident Response Team}, ed è il
team che gestisce l'incidente, ha una conoscenza tecnica dei sistemi e di
``hacking'' (è il classico informatico)

\paragraph*{Investigatore} Presenti in aziende grandi, è una persona di fiducia
che ha accesso a tutte le risorse aziendali e si occupa di cercare le cause
dell'incidente che a volte possono anche essere dolose.



\subsection{Incident Response Plan}

Questo piano si suddivide in diversi \textit{step}:
\begin{enumerate}
\item Preparazione
\item Identificazione
\item Contenimento
\item Analisi e eradicazione
\item Recupero
\item Imparare
\end{enumerate}

Vediamo brevemente il significato di ogni passo, che sarà analizzato in
profondità più avanti.

\paragraph*{Preparazione} Questa fase è la preparazione dell'incidente, ovvero è
una fase che è l'\textbf{analisi dello scenario}.

\paragraph*{Identificazione} Questo \textit{step} succede quando si viene
avvisati dell'anomalia e si comincia a cercare la causa del problema. Più e
ricco e sistema il complesso informativo più è difficile trovare la causa del
problema.

\paragraph*{Contenimento} In questa fase si cerca di ``limitare i danni'', con
l'obiettivo di mantenere l'operatività. Questo a volte consiste in sacrificare
elementi della produttività.

\paragraph*{Analisi e eradicazione} Si cerca di capire cos'è successo e perchè è
successo, e si cerca di risolvere la causa.

\paragraph*{Recupero} Si ritorna con i servizi di nuovo funzionanti, per
ritornare alla fase di preparazione

\paragraph*{Imparare} Questa parte è molto importante, in quanto imparare la
lezione permette di evitare futuri attacchi e di rendere il sistema più sicuro.
