% STEP 3 6.3.1.5 Trattamento dei rischi

Ci si basa per cose sull'esperienza passata per eventi per esempio atmosferici.
Per eventi che invece non è possibile basarsi sulle esperienze precedenti ci si
basa su delle \textit{guidelines}.

\paragraph*{Fonti di perdite}

Archetipi:
\begin{itemize}
\item Utente malintenzionato (o malizioso): vuole causare un danno all'azienda
\item Utente non intenzionale: causa un danno all'azienda ma senza volerlo
\item Utente che piega le regole: in qualche modo è a metà tra i due profili
precedenti. Le sue azioni sono intenzionali, ma non maliziose.
\end{itemize}

Il 51\% pensa sia accettabile prendere dati dell'azienda perchè pensano che sia
normale farlo in assenza di \textit{policies}. Molti dati vengono passati
nell'azienda tramite \textit{Dropbox} o su \textit{Gmail}. Molti anche eseguono
download illegali sul computer aziendale. Essenzialmente, c'è poco controllo sui
dati, facendo vedere come un utente può avere un atteggiamento malizioso senza
troppi problemi.


% Questa parte È NUOVA, il resto VA INTEGRATO con gli appunti dei giorni scorsi
\subsection{Tipi di controlli}

La vulnerabilità è ciò che una minaccia sfrutta per portare un attacco.


% Questa parte dove va? Cos'è?
La tolleranza zero è una policy adottata all'interno dell'azienda, su cui
l'azienda di solito non transige ed esegue il licenziamento. Alcuni esempi sono
per esempio su discriminazioni raziali o basati sulla religione.

\begin{figure}[H]
 \centering
 \includegraphics[scale=0.5]{controlTypes}
 \caption{Schema dei diversi tipi di controlli}
\end{figure}



\paragraph*{Controlli e contromisure}

Il costo non dovrebbe mai superare la perdita attesa.
Una contromisura la si pu\`o vedere come un controllo mirato (specifico per un
particolare tipo di minaccia o vulnerabilità)

\todo{Mancano pro e contro - Ho cercato nelle slide e non li ho trovati}

\paragraph*{Monitoraggio del rischio}

Praticamente consiste nel PDCA\footnote{Ciclo di Deming}.
C'è l'IOS\footnote{Office of Internal IOS Service è l'ufficio che se ne occupa.}
che fa il planning dell'audit per i prossimi $n$ anni.


\paragraph*{Security control baselines and metrics}

Una \textit{baseline} è un parametro di riferimento ed è molto importante,
perchè ci permette di prendere le misure sulle cose per vedere se stiamo
migliorando o meno!
Le misure sono collezionate durante gli anni. La consistenza delle misurazioni
sono molto importanti.

\section{Risk management}
È un qualcosa che dev'essere allineato con la strategia aziendale.
È business driven (non technology driven).


Lo \textit{steering cometee} definisce le proprietà della gestione del rischio.
Ed è preferibile metterci gente che conosce a fondo il business.

\subsection{I ruoli del risk-management}

I ruoli del \textit{risk management} sono:
\begin{itemize}
\item CIO e CISO\footnote{La tendenza è che i ruoli si stanno fondendo}
\item Security Trainers: sono sostanzialmente dei consulenti, e sono incaricati
di sviluppare materiale d'insegnamento adeguato per educare gli utenti finali
\item System/Info Owners: responsabili di assicurarsi che i controlli siano
funzionanti
\item Governance \& Security Management: allocano risorse, valutano e usano i
risultati della valutazione del rishio
\item Business Managers: prendono decisioni difficili riguardo alla priorit\`a
di raggiungere \textit{business goals}
\item IT Security Practitioners: implementano la sicurezza nei sistemi IT
\end{itemize}

\subsection{Due Diligence}

È quella cosa che va messa in piedi per non essere perseguiti in maniera civile,
ovvero far le cose con ``buon senso''. Oggi come oggi si esegue il \textit{risk
assessment} per prevenire guai dal punto di vista civile. La \textbf{due care}
si esegue per ridurre il rischio a livelli accettabili. Nell'ordinamento
giuridico italiano il termine è diligenza del buon padre di famiglia (articolo
1176).


\section{Esercizi}

Gli esercizi relativi a questa parte si possono trovare in \ref{esGestRisk}
