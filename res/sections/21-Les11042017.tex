% Sta roba sta in 6.3.1.4 Calcolare la perdita attesa

\textit{Security breach notification laws} sono leggi che impongono ad entità 
che sono state soggette a data stealing di notificarlo ai clienti e di prendere 
contromisure per cercare di rimediare alle possibili conseguenze. Esiste una 
direttiva europea.


\paragraph*{Tipologie di Minacce}

Tipi di aggenti:
\begin{itemize}
\item Hackers Crackers
\item Criminali
\item Terroristi
\item Spie industriali
\item Dall'interno
\end{itemize}
%copiare tabella


\paragraph*{Come difendersi}

Per difendersi le grande società di software eseguono delle settimane di 
allenamento per i programmatori definite anche come \textit{security coding}



% STEP 3 6.3.1.5 Trattamento dei rischi

Ci si basa per cose sull'esperienza passata per eventi per esempio atmosferici. 
Per eventi che invece non è possibile basarsi sulle esperienze precedenti ci si 
basa su delle \textit{guidelines}.

\paragraph*{Fonti di perdite}

Archetipi:
\begin{itemize}
\item Utente malintenzionato (o malizioso): vuole causare un danno all'azienda
\item Utente non intenzionale: causa un danno all'azienda ma senza volerlo
\item Utente che piega le regole: in qualche modo è a metà tra i due profili 
precedenti. Le sue azioni sono intenzionali, ma non maliziose.
\end{itemize}

Il 51\% pensa sia accettabile prendere dati dell'azienda perchè pensano che sia 
normale farlo in assenza di \textit{policies}. Molti dati vengono passati 
nell'azienda tramite \textit{Dropbox} o su \textit{Gmail}. Molti anche eseguono 
download illegali sul computer aziendale. Essenzialmente, c'è poco controllo sui 
dati, facendo vedere come un utente può avere un atteggiamento malizioso senza 
troppi problemi.


% Questa parte È NUOVA, il resto VA INTEGRATO con gli appunti dei giorni scorsi
\paragraph*{Step 4: calcolare la perdita attesa}

Tre tipi di valutazione
\begin{itemize}
\item Qualitativa: corretta perchè riuscite a prioritizzare i rischi, quindi 
capite cosa è più importante di cosa.
\item Quantitativa: la più corretta perché in qualche modo riporta in una scala 
assoluta (es. soldi). Il problema è arrivarci ad una valutazione corretta.
\item Semiquantitativa: 
\end{itemize}

\subparagraph*{Analisi qualitativa}

\todo{Va troppo veloce}


\subparagraph*{Analisi semiquantitativa}

\todo{vedi sopra}

.

\subparagraph*{Analisi quantitativa}

\begin{itemize}
\item Single Loss Expectancy (SLE): il costo per l'organizzazione se una 
minaccia si verifica\todo{aggiungere formule}
\item Annualized Rate of Occurrence (ARO): quanto costa questo evento in un 
anno?
\item Annual Loss Expectancy (ALE): la perdita attesa annuale è dato da: $ALE = 
SLE \cdot ARO$
\end{itemize}


Quanto si perde del bene se l'evento capita? 

Sull'anno questo evento quanto mi costa? L'evento negativo ha sicuramente un 
certo costo, ma che impatto ha all'anno? Se per esempio una alluvione avviene 
ogni 30 anni, allora il suo costo è $1/30$ all'anno. Quindi ogni anno l'azienda 
ha da risparmiare un certo valore per far frontee alla perdita futura.

\subsubparagraph*{Esempio}


Quantitativamente:
\begin{itemize}
\item Il costo di un incidepnte HIPAA con protezioni insufficenti:
\begin{itemize}
 \item SLE = 50k + (1y in prigione) \$ 100k = \$150k
 \item Più perdite in termine di reputazione
\end{itemize}
\item Stima del tempo = 10 anni o meno = 0,1
\item Stime di perdita annuale (ALE): \$150 $\cdot 0,1$ = \$15k
\end{itemize}

\todo{Ci sarebbe anche la tabella per completare l'esempio. Serve? Per me è no, 
non quella gialla piuttosto quella dopo? - Ok per quella dopo allora}

\paragraph*{Step 5: trattamento del rischio}



\begin{itemize}
\item Accettazione del rischio. È importante prendere delle decisioni riguardo 
all'accettazione del rischio.
\item Evitare il rischio: bloccare i comportamenti che causano esposizione al 
rischio
\item Mitigazione del rischio: implementare dei controlli per minimizzare le 
vulnerabilità sotto un livello accettabile.
\item Trasferimento del rischio: qualcun'altro assume il rischio per l'azienda. 
Per esempio potrebbe essere ricorrere ad una assicurazione.
\item Pianificazione del rischio: implementare una serie di contromisure. Che 
controlli dovrebbero essere messe in piedi?
\end{itemize}
\todo{Ha detto che lo chiede sicuramente :(}

\todo{copiare flowchart?}
Ad un certo punto potrebbe non essere più possibile eliminare il rischio. Il 
rischio che rimane viene detto \textbf{richio residuo} e questo rischio viene 
accettato.

La \textit{sensitivity analysis} indica quanto l'azienda è ``sensibile'' a certi 
avvenimenti.

Il risk assessment va fatto solo su una parte dell'azienda, non su tutte.



\subsection{Tipi di controlli}

\todo{Add control types l'immagine}

La vulnerabilità è ciò che una minaccia sfrutta per portare un attacco.


% Questa parte dove va? Cos'è?
La tolleranza zero è una policy adottata all'interno dell'azienda, su cui 
l'azienda di solito non transige ed esegue il licenziamento. Alcuni esempi sono 
per esempio su discriminazioni raziali o basati sulla religione. 


\paragraph*{Controlli e contromisure}

Il costo non dovrebbe mai superare la perdita attesa.
Contromisura = Controllo mirato (specifico per un particolare tipo di minaccia o 
vulnerabilità)
\todo{Mancano pro e contro}

\paragraph*{Step 6: monitoraggio del rischio}

Praticamente consiste nel PDCA.
C'è l'IOS\footnote{Office of Internal IOS Service è l'ufficio che se ne occupa.} 
che fa il planning dell'audit per i prossimi $n$ anni.




\paragraph*{Security control baselines and metrics}

Una \textit{baseline} è un parametro di riferimento ed è molto importante, 
perchè ci permette di prendere le misure sulle cose per vedere se stiamo 
migliorando o meno!
Le misure sono collezionate durante gli anni. La consistenza delle misurazioni 
sono molto importanti.

\section{Risk management}
È un qualcosa che dev'essere allineato con la strategia aziendale.
È business driven (non technology driven).


Lo \textit{steering cometee} definisce le proprietà della gestione del rischio. 
Ed è preferibile metterci gente che conosce a fondo il business.

\subsection{I ruoli del risk-management}
\begin{itemize}
\item CIO e CISO la tendenza è che i ruoli si stanno fondendo
\item Security Trainers che sono consuletnti
\item System/Info Owners
\item \todo{continua la lista}
\end{itemize}

\subsection{Due Diligence}

È quella cosa che va messa in piedi per non essere perseguiti in maniera civile, 
ovvero far le cose con ``buon senso''. Oggi come oggi si esegue il \textit{risk 
assessment} per prevenire guai dal punto di vista civile. La \textbf{due care} 
si esegue per ridurre il rischio a livelli accettabili. Nell'ordinamento 
giuridico italiano il termine è diligenza del buon padre di famiglia (articolo 
1176).


\section{Esercizi}
%ESERCIZI

Il Risk Assessment includes:
\begin{itemize}
\item The steps: risk analysis, risk treatment, risk acceptance, and risk 
monitoring
\item Answers the question: What risk are we prone to, and what is the financial 
costs of these risks?
\item Assesses controls after implementation
\item The identification, financial analysis, and prioritization of risks, anon 
of controls (risposta esatta)
\end{itemize}


% Altro esercizio
Risk management includes:
\begin{itemize}
\item The steps: risk analysis, risk treatment, risk acceptance, and risk 
monitoring (risposta esatta)
\item Answers the question: What risk are we prone to, and what is the financial 
costs of these risks?
\item Assesses controls after implementation
\item The identification, financial analysis, and prioritization of risks, anon 
of controls
\end{itemize}

% Altro esercizio
The FIRST step in Security Risk assessment is:
\begin{itemize}
\item determine threats and vulnerabilities
\item determine values of key assets (corretta)
\item Analyze existing controls
\item 
\end{itemize}



% Altro

\begin{itemize}
\item The probability that an attack will occur in one year
\item The duration of time where a loss is expected to occur
\item The cost (risposta esatta)
\end{itemize}


The role(s) responsible for deciding whether risk should be accepted, 
transferred, or mitigated is:
\begin{itemize}
\item The Chied information officer
\item The chief risk officer
\item The chif information security officier
\item Enterprise fovernance and senior business management (risposta esatta)
\end{itemize}

% Altro esercizio

Which of these risk is best measured using a qualitative process?
\begin{itemize}
\item Temporart power outae in an office building
\item loss of consumer confidence due to a malfunctioning website (corretta)
\item Theft of an mployee's laptop while traveling
\item Disruption of supply deliveries due to flooding
\end{itemize}

% Altro esercizio
The risk that is assumed after implementing controls is known as:

\begin{itemize}
\item accepted risk
\item ALE
\item Quantitative risk
\item residual risk (corretta)
\end{itemize}

% Altro esercizio
The primary purpose of risk management is to:
\begin{itemize}
\item Eliminate all risk
\item Find the most cost-effective controls
\item Reduce risk o an acceptable level (corretta)
\item Determine budget for residual risk
\end{itemize}

% Altro esercizio
Due diligence ensures that

\begin{itemize}
\item An organization has exercised the best possible security practices 
according to best practices
\item An organization has exercised acceptably reasonable security practices 
addressing all major security areas
\item An organization has implemented risk management and established the 
necessary controls
\item An organization has allocated a CISO who is responsible for securing the 
organization's information assets
\end{itemize}

% Altro esercizio

\begin{itemize}
\item The average cost of loss of this asset, for a single incident
\item An estimate using quantitative risk management of the frequency of asset 
loss due to a threat
\item an estimate using qualitative risk management of the priority of the 
vulnerability
\item ALE = SLE x ARO (corretta)
\end{itemize}
