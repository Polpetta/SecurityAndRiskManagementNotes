\part{Sicurezza fisica e del personale}
\label{SFDP}
\chapter{Sicurezza fisica}

La difesa in profondità serve per proteggersi quando un livello di sicurezza
viene superato. Questo tipo di difesa è possibile immaginarsela come gli strati
di una cipolla.

È importante attrezzarsi anche per difendersi da eventi che possono accadere
all'interno dell'azienda (es. incendio).

\section{Tipologie di controlli}

Le diverse tipologie di controlli sono concentriche, a difficoltà incrementale e
indipendenti. Questi tipi di controlli possono essere \textbf{preventivi},
\textbf{reattivi} e \textbf{correttivi}.

Una lista di controlli (a partire dalla parte centrale per muoversi
verso quella più esterna) sono:

\begin{itemize}
\item Postazione di lavoro chiusa;
\item Videocamere e sistema di allarme;
\item Restrizione del personale d'accesso;
\item Guardie di sicurezza, log manuale e badge con foto identificative;
\item Singolo punto di accesso controllato e finestre sbarrate;
\item Non pubblicizzare l'ubicazione di strutture (\textit{facilities})
sensibili.
\end{itemize}

Un buon consiglio è avere un numero di accessi al locale ridotto. Questo però
non è sempre possibile, in quanto in una \textit{facility} grossa potrebbero
esserci norme di sicurezza da seguire.

Anche dal punto di vista del riconoscimento una buona pratica è
far indossare il badge al personale.

Ognuna delle misure esposte prima hanno un costo che di solito è grande, che
può portare a problemi (es. le registrazioni dovrebbero essere
salvate anche in postazioni sicure in quanto potrebbero essere soggette a furti).

Quanto spenderci e quanto investire è una questione a cui è difficile
rispondere.

\section{Sicurezza fisica}
\label{SFDP:fisica}
\subsection{Protezione dalla corrente}

Classificazione in funzione del fenomeno negativo che si verifica:
\begin{itemize}
\item \textbf{Surge, spikes and sags:} qualcuno ha collegato un dispositivo alla rete che
ha causato un sovraccarico, causando picchi (positivi e negativi);
\item \textbf{Blackout};
\item \textbf{Brownout:} i livelli di corrente dichiarati non sono quelli effettivi sulla
rete;
\item \textbf{EMI (Electromagnetic Interference):} dispositivi elettrici che generano
effetti parassitari su dispositivi terzi.
\end{itemize}

I dispositivi di protezione dipendono dalla durata del fenomeno: per la breve
durata bastano i surge protector, per durata minore di 30 min si utilizza un UPS
(universal power supply); per periodi più lunghi si usa un generatore
alternativo di corrente. I generatori possono arrivare anche a costare milioni
di euro.

\subsection{Equipaggiamento della computer room}

\begin{itemize}
\item Rilevatore d'acqua (a livello del pavimento);
\item Estintori;
\item Allarme antincendio manuale;
\item Rilevatori di fumo (in alto e in basso);
\item Interruttori di spegnimento d'emergenza.
\end{itemize}

\subsection{Ambiente IPF}

Una sala di calcolo non va messa nei piani alti in quanto potrebbe essere
soggetta a fulmini e a intemperie, e neanche al piano base in quanto è soggetta
ad allagamenti. Un buon posizionamento è a metà della costruzione. È
importante che ci siano anche delle porte taglia fuoco.

\subsubsection{Sistema per la soppressione dei fuochi}

Questi sistemi possono essere:
\begin{itemize}
\item Acqua;
\begin{itemize}
\item Caricato: ovvero gli estintori sono caricati e hanno l'attivazione
immediata. Lo svantaggio è che essendo sempre caricati potrebbero costituire un
problema;
\item Secchi.
\end{itemize}
\item Gas;
\begin{itemize}
\item Pericolosi (che possono essere Halon e Anidride Carbonica);
\item Amici dell'ambiente, che non spengono veramente il fuoco ma riducono le
capacità ignifughe (FM-200 e Argonite).
\end{itemize}
\end{itemize}

\subsection{Sistemi di chiusura}

Esistono diverse modalità di chiusura dei sistemi, che per essere sbloccati
possono richiedere dalla autenticazione biometrica fino ad una semplice
porta elettronica.

Molti di questi sistemi sono elettrici ed è quindi importante che i
meccanismi siano \textit{fail-safe}: ovvero che in mancanza della corrente
elettrica la porta debba rimanere aperta. Proprio per questo gli attaccanti
potrebbero sfruttare ciò per guadagnare l'accesso.

\subsubsection{Porta dell'uomo morto}
Questa tecnica prevede di avere due porte e può essere aperta solamente una alla volta. Oltre a ciò permette solo una persona nell'area tra le due porte, riducendo il rischio di \textit{piggybacking}, cioè che una persona non autorizzata segua una autorizzata. È un sistema utilizzato, per esempio, dalle banche.


\subsection{Computer nei luoghi pubblici}

Protezioni logiche: Imaged computers, antivirus e antispyware, filtri web,
login/password e protezione firewall dal resto dell'organizzazione.

Protezioni fisiche: lucchetto sul case, dispositivo della kensington per i
laptop. Anche le etichette con il codice è meglio inciderle nello chassis del
laptop piuttosto che usare un'etichetta rimovibile.

\subsection{Mobile computing}

\begin{itemize}
\item Incisione del numero seriale sul laptop;
\item Backup dei dati critici e sensibili;
\item Criptare i dati sensibili;
\item Proteggere i file importanti con password individuali;
\item Individuare il responsabile del furto e segnalarlo.
\end{itemize}


\subsection{Sicurezza dei dispositivi}

Le porte USB e le entrate Flash devono essere bandite o disabilitate dal
computer. In alternativa è necessario eseguire una crittografia di tutti i dati.


\subsection{Tabella delle criticità}

\begin{figure}[H]
 \centering
 \includegraphics[width=0.8\textwidth]{criticality-table}
\end{figure}

\subsection{Mappa della sicurezza fisica}
\begin{figure}[H]
 \centering
 \includegraphics[width=0.8\textwidth]{physical-security-map}
\end{figure}

\subsection{Esercizi}

Gli esercizi sono disponibili nella Sezione \ref{esSFDP:fisica}.

\section{Sicurezza del personale}
\label{SFDP:DP}

\subsection{Consapevolezza della sicurezza \& Training}

Il training copre quello che ci si aspetta dagli impiegati?

Il training può essere implementato come: orientamento dei nuovi impiegati e
newsletter interna alla compagnia. Inoltre è possibile determinare l'efficacia
intervistando gli impiegati.


\subsubsection{Awareness Function: tipi di addestramento sulla sicurezza}

\textbf{Awareness:} crea nel personale una sensibilità nei confronti di problemi di
sicurezza.\\
\newline
\textbf{Training:} skill necessarie per una posizione particolare.\\
\newline
\textbf{Education:} skills di alto livello.\\
