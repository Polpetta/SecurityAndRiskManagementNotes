\chapter{Gestione Del Rischio}

La teoria riguardante questa serie di esercizi si trova nella
parte \ref{riskMng}.

\section{Gestione del rischio}

\label{esGestRisk}

\begin{Exercise} [
  title={Quiz},
  label={gestRisk1}
  ]

  \Question Il \textit{Risk Assessment} include:

\begin{enumerate}
 \item I passi: analisi dei rischio, trattamento del rischio, accettazione del
rischio e monitoraggio del rischio
 \item Risponde alla domanda: che cosa i rischi sono proni a, a che cosa sono i
costi finanziari per questi rischi?
 \item Stima il controllo dopo l'implementazione
 \item L'identificazione, l'analisi finanziaria, la prioritizzazione dei rischi
e la valutazione dei controlli
\end{enumerate}

\end{Exercise}


\begin{Answer} [
  ref={gestRisk1},
  number={1}
  ]

  \Question Risposta esatta: 4

\end{Answer}

% Altro esercizio

\begin{Exercise} [
  title={Quiz},
  label={gestRisk2}
  ]

  \Question Il \textit{Risk Management} include:
\begin{enumerate}
 \item I passi: analisi dei rischio, trattamento del rischio, accettazione del
rischio e monitoraggio del rischio
 \item Risponde alla domanda: che cosa i rischi sono proni a, a che cosa sono i
costi finanziari per questi rischi?
 \item Stima il controllo dopo l'implementazione
 \item L'identificazione, l'analisi finanziaria, la prioritizzazione dei rischi
e la valutazione dei controlli
\end{enumerate}
\end{Exercise}


\begin{Answer} [
  ref={gestRisk2},
  number={2}
  ]

  \Question Risposta esatta: 1

\end{Answer}


% Altro esercizio

\begin{Exercise} [
  title={Quiz},
  label={gestRisk3}
  ]

  \Question Il primo passo nella \textit{Security Risk Assessment} \`e:
\begin{enumerate}
 \item Determinare i pericoli e le vulnerabilit\`a
 \item Determinare il valori degli \textit{assets} pi\`u importanti
 \item Stimare la probabilit\`a di un \textit{exploit}
 \item Analizzare controlli esistenti
\end{enumerate}

\end{Exercise}


\begin{Answer} [
  ref={gestRisk3},
  number={3}
  ]

  \Question Risposta esatta: 2

\end{Answer}


% Altro esercizio

\begin{Exercise} [
  title={Quiz},
  label={gestRisk4}
  ]

  \Question La \textit{Single Loss Expectancy} si riferisce:
\begin{enumerate}
 \item Alla probabilit\`a che un attacco accadr\`a in un anno
 \item Alla durata di tempo dove ci si aspetta che una perdita avverr\`a (ad
esempio un mese, un anno, una decade)
 \item Il costo di perdere un \textit{asset} una volta
 \item Il costo medio di quell'\textit{asset} per anno
\end{enumerate}

\end{Exercise}


\begin{Answer} [
  ref={gestRisk4},
  number={4}
  ]

  \Question Risposta esatta: 3

\end{Answer}


% Altro esercizio

\begin{Exercise} [
  title={Quiz},
  label={gestRisk5}
  ]

  \Question Il ruolo responsabile per decidere quando il rischio dovrebbe
essere accettato, trasferito o mitigato \`e:
\begin{enumerate}
 \item Il \textit{Chief Information Officer}
 \item Il \textit{Chief Risk Officer}
 \item Il \textit{Chief Information Security Officer}
 \item L'\textit{Enterprise governance} e \textit{Senior business management}
\end{enumerate}

\end{Exercise}


\begin{Answer} [
  ref={gestRisk5},
  number={5}
  ]

  \Question Risposta esatta: 4

\end{Answer}


% Altro esercizio

\begin{Exercise} [
  title={Quiz},
  label={gestRisk6}
  ]

  \Question Quali di questi rischi sono misurati meglio usando un processo
qualitativo?
\begin{enumerate}
 \item Mancanza temporanea di corrente in un ufficio
 \item Perdita della fiducia dei clienti a causa di un malfunzionamento del
sito web
 \item Furto del portatile di un impiegato mentre esso era in viaggio
 \item Mancanta consegna di prodotti dovuta ad una inondazione
\end{enumerate}

\end{Exercise}


\begin{Answer} [
  ref={gestRisk6},
  number={6}
  ]

  \Question Risposta esatta: 2

\end{Answer}


% Altro esercizio

\begin{Exercise} [
  title={Quiz},
  label={gestRisk7}
  ]

  \Question Il rischio che \`e assunto dopo l'implementazione dei controlli \`e
conosciuto come:
\begin{enumerate}
 \item Rischio accettato
 \item Perdita aspettata annuale
 \item Rischio quantitativo
 \item Rischio residuo
\end{enumerate}

\end{Exercise}


\begin{Answer} [
  ref={gestRisk7},
  number={7}
  ]

  \Question Risposta esatta: 4

\end{Answer}


% Altro esercizio

\begin{Exercise} [
  title={Quiz},
  label={gestRisk8}
  ]

  \Question Lo scopo primario della gestione del rischio \`e di:
\begin{enumerate}
 \item Eliminare tutto il rischio
 \item Trovare il miglior rapposto costo/efficacia per i controlli
 \item Ridurre il rischio ad un livello accettabile
 \item Determinare il \textit{budget} per il rischio residuo
\end{enumerate}

\end{Exercise}


\begin{Answer} [
  ref={gestRisk8},
  number={8}
  ]

  \Question Risposta esatta: 3

\end{Answer}


% Altro esercizio

\begin{Exercise} [
  title={Quiz},
  label={gestRisk9}
  ]

  \Question La diligenza dovuta (\textit{due diligence}) si assicura che:
\begin{enumerate}
 \item Una organizzazione abbia esercitato le migliori pratiche di sicurezza
possibili in accordo alle migliori pratiche
 \item Una organizzazione abbia esercitato in maniera accettabile le pratiche
di sicurezza riguardo a tutte le aree di sicurezza
 \item Una organizzazione abbia implementato la gestione del rischio e abbia
messo in pratica tutti i controlli necessari
 \item Una organizzazione abbia nominato un \textit{Chief Information Security
Officier} che sia responsabile della sicurezza degli \textit{asset} informativi
\end{enumerate}

\end{Exercise}


\begin{Answer} [
  ref={gestRisk9},
  number={9}
  ]

  \Question Risposta esatta: 4

\end{Answer}


% Altro esercizio

\begin{Exercise} [
  title={Quiz},
  label={gestRisk10}
  ]

  \Question L'\textit{ALE} \`e:
\begin{enumerate}
 \item Il costo medio della perdita di quell \textit{asset}, per un sinfolo
incidente
 \item Usando la gestione del rischio quantitativa una stima della frequenza
della perdita di un \textit{asset} a causa di un furto
 \item Usando la gestione del rischio quanlitativa una stima della priorit\`a
della vulnerabilit\`a
 \item $ALE\ =\ SLE\ \cdot\ ARO$
\end{enumerate}

\end{Exercise}


\begin{Answer} [
  ref={gestRisk10},
  number={10}
  ]

  \Question Risposta esatta: 4

\end{Answer}
