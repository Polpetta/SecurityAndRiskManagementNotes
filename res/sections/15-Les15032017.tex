\chapter{Information security}

Obiettivi:
\begin{itemize}
\item principi della sicurezza dell'informazione
\item posizioni all'interno degli organigrammi all'interno della gestione della sicurezza
\item tecniche di accesso controllato
\item tecniche di autenticazione
\item tecniche di combinazione di combinazione
\item tecniche biometriche
\item elementi del Bell-LaPadula
\item elementi di sicurezza miliare
\item backup
\item schemi di classificazione dell'informazione
\end{itemize}

\section{Goal della sicurezza dell'informazione}

L'IS è una triade: integrità, disponibilità e confidenzialità. È molto
importante anche la conformità rispetto alla legge e rispetto ai requisiti di 
privacy.

L'Italia è leader nella privacy e le sue normative sono molto avanzate.


\section{Principi dell'information security}

Questi principi non invecchiano mai e sono fondamentali:
\begin{itemize}
  \item Need-to-know \\
  Le persone dovrebbero avere la abilità di accedere solamente ai dati che gli 
  servono, nulla di più. Questo principio è difficile da realizzare, ma vengono 
  di molto ridotti i rischi.
  \item Least Privilege \\
  È la trasposizione nel senso del processo del need-to-know. Ogni persona 
  dovrebbe avere le abilità sufficienti per eseguire i propri task, ma non 
  altro.
  \item Segregation of Duties \\
  Si occupa della separazione dei ruoli tra diverse persone. Di solito si 
  divide sempre in: chi origina, chi autorizza e chi verifica.
  \item Privacy \\
  Questo concetto è molto grande, e dice che il trattamento dei dati che 
  vengono raccolti devono essere proporzionali alle attività che vengono 
  compiute sul soggetto. Un'altra cosa importante è la \textit{data retention}, 
  ovvero il limite massimo in cui è possibile conservare i dati, dopo la quale 
  la cancellazione è obbligatoria. I dati possono essere:
  \begin{itemize}
    \item Personali
    \item Sensibili (ad esempio dati medici): sono presenti qui norme 
    ristrettissime
  \end{itemize}
  Se non vengono trattati in maniera corretta è possibile che ci siano sanzioni 
  anche di tipo penali.
\end{itemize}

\section{Gestione della sicurezza della informazione}

Componenti di un programma di information security di successo:

\begin{itemize}
\item  Senior Mgmt commitment: capire se i manager sono realmente interessati alla sicurezza dell'azienda.
\item Politiche (direttive di alto livello per raggiungere delle obiettive) e Procedure (declinano le politiche su aspetti concreti).
\item Allocazione delle responsabilità: chi fa cosa.
\item Educazione e Consapevolezza sulla sicurezza: l'educazione e la consapevolezza sono il modo migliore per prevenire le frodi e gli incidenti.
\item Audit e Compliance (azioni che devono essere attuate per aderire alla legge): se l'azienda non è compliance ci sono delle azioni correttive/punitive.
\item Gestione e risposta agli incidenti.
\end{itemize}

Dovrebbe essere presente un programma per la sicurezza della informazione. La 
prima cosa da fare di solito è capire se c'è il 
\textit{commitment}\footnote{L'impegno} da parte dei ``pezzi grossi'', ovvero 
se ci tengono. Le politiche\footnote{Direttive di alto livello. Di solito sono 
pubbliche ma sono molto astratte.} e le procedure\footnote{Declinano le 
politiche su qualcosa di più concreto.} vanno messe in atto. Come già visto 
durante il \textit{disaster recovery plan}, è importante gestire le 
responsabilità in maniera precisa, soprattutto per evitare possibili ``scarichi 
di responsabilità''. Non è che bisogna mettere il nome e cognome dell'addetto, 
ma assegnare la responsabilità ad un certo ruolo è una buona pratica. La 
\textit{security and awareness education} è uno strumento potentissimo, che 
applica la prevenzione per evitare incidenti. Si dice \textit{awareness} quando 
si sa che c'è qualcosa che non va, mentre \textit{education} è quando si è 
consci di una situazione sensibile e si sa come agire.
La \textit{compliance} sono le normative che di solito bisogna seguire per 
essere il regola con la legge. Di solito in caso di violazione si può andare 
incontro ad:
\begin{itemize}
  \item Processi interni
  \item Penali
  \item Denuncia (anche penale)
\end{itemize}
Bisogna infine anche avere un piano di risposta per l'incidente.

Quando il programma viene finito bisogna che venga approvato, in cui vengono 
eseguiti dei confronti e vengono analizzati eventuali dubbi.

\section{Titoli comuni e responsabilità}

% TODO: inserire schema

\begin{itemize}
  \item CEO/President \\
  \item CISO: in principio non esisteva, adesso sta prendendo il sopravvento. Copre anche il ruolo del CSO.
  \begin{itemize}
	\item Security Specialist
	\item Security Administrator \\
	Alloca le risorse per gli impiegati basandosi sulla documentazione scritta, 
  oltre a monitorare l'accesso ai terminali e alle applicazioni e a preparare 
  report sulla sicurezza.
  \end{itemize}
  \item CSO: c'è da sempre
  \item Business Executive
  \begin{itemize}
  \item Information/Data Owner \\
  Questo ruolo si occupa di:
  \begin{itemize}
    \item Decidere che è responsabile all'accesso al dato. Molto importante è 
    la gestione dell'\textit{authorization creep}, ovvero quando si hanno delle 
    autorizzazioni che inizialmente erano utili ma che ora non servono più.
  \end{itemize}
  \item Process Owner \\
  Responsabile del processo (algoritmo che opera sul dato)
  \item IS Auditor \\
\end{itemize}
  \item Information Owner o Data Owner \\
  
  \item Data custodian \\
  Si preoccupa della protezione dei dati e potrebbe essere un analista di 
  sistema o un amministrazione di sistema.
  
\end{itemize}

\section{Classificazione della sicurezza}

Anche la sicurezza ha una propria classificazione, che è:
\begin{itemize}
  \item Critical
  \item Vital
  \item Sensitive
  \item Nonsensitive
\end{itemize}

\subsection{Classificazione della sensitività}

\begin{itemize}
\item Public: può accedervi chiunque.
\item Internal: può accedervi solo chi è interno all'azienda.
\item Private: possono accedervi solo per persone autorizzate.
\item Confidential: può accedervi sono chi è in una determinata lista.
\end{itemize}

\subsection{Classificazione del dato}

Diversi dati possono essere classificati, ma come?

\begin{itemize}
\item Come marchiamo le informazioni classificate? Nell'informazione cartacea, 
se correttamente classificata riporta un apposito timbro per esempio.
\item Chi decide che classificazione dare ai dati? Il classico problema è la
overclassification: dò una classificazione più alta di quella necessaria al
documento, il che implica spreco di risorse.
\item Come vengono trasportate/conservate/gestite/archiviate le informazioni classificate?
\item Cosa dice la legge sulla gestione di questa informazione?
\item Chi ha l'autorità per determinare chi ha l'accesso e per cosa si ha
l'accesso? Tipicamente il CISO.
\end{itemize}

\subsection{Gestione dei dati sensibili}

%copia tabella

Le informazioni priviledged hanno un trattamento di tipo privilegiale e di solito hanno carattere legale.

\subsubsection{Conservazione e distruzion di dati confidenziali}

%da completare
Disposing of Media
Repair: quando viene mandato fuori un materiale per la riparazione (es. le fotocopiatrici hanno una memoria interna che deve essere cancellata prima di essere mandata in riparazione).
Storage

\section{Esercizi}

Gli esercizi si trovano nella sezione \ref{EsBCDR3}

\section{Tipologie di sicurezza}

\subsection{Sicurezza in profondità}

Per ridurre il rischio di un attacco in maniera facile, viene evitato di 
mettere tutte le difese in un unico punto, ma vengono stratificate le difese in 
diversi punti, usando tecnologie diverse.
Per arrivare al kernel, devono essere bucati prima tutti gli strati intorno.