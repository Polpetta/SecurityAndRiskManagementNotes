\part{Information security}
Obiettivi di questa parte:
principi della sicurezza dell'informazione, posizioni all'interno
degli organigrammi della gestione della sicurezza, tecniche di accesso
controllato, tecniche di autenticazione, tecniche di combinazione,
tecniche biometriche, elementi del Bell-LaPadula, elementi di sicurezza
militare, backup e schemi di classificazione dell'informazione.

\chapter{Information security}
\label{cap:infoSec}
\section{Goal della sicurezza dell'informazione}

L'IS è una triade: \textbf{integrità}, \textbf{disponibilità} e
\textbf{confidenzialità}. È molto importante anche la conformità rispetto alla
legge e rispetto ai requisiti di privacy. L'Italia è leader nella privacy e le
sue normative sono molto avanzate.


\section{Principi dell'information security}

Questi principi non invecchiano mai e sono fondamentali:
\begin{itemize}
  \item \textbf{Need-to-know:} le persone dovrebbero avere la abilità di
  accedere solamente ai dati che gli servono, nulla di più. Questo principio è
  difficile da realizzare, ma, se messo in pratica, vengono ridotti di molto i
  rischi.
  \item \textbf{Least Privilege:} è la trasposizione nel senso del processo
  del need-to-know. Ogni persona dovrebbe avere le abilità sufficienti per
  eseguire i propri task, ma non altro.
  \item \textbf{Segregation of Duties:} si occupa della separazione dei ruoli
  tra diverse persone. Di solito si divide sempre in: chi origina, chi
  autorizza e chi verifica.
  \item \textbf{Privacy:} questo concetto è molto grande, dice che il
  trattamento dei dati che vengono raccolti devono essere proporzionali alle
  attività che vengono compiute sul soggetto. Un'altra cosa importante è la
  \textit{data retention}, ovvero il limite massimo di tempo per il quale è
  possibile conservare i dati. Trascorso questo intervallo di tempo la
  cancellazione dei dati è obbligatoria. I dati possono essere:
  \begin{itemize}
    \item Personali;
    \item Sensibili (ad esempio dati medici): sono presenti qui norme
    ristrettissime;
  \end{itemize}
  Se non vengono trattati in maniera corretta è possibile che ci siano sanzioni
  anche di tipo penali.
\end{itemize}

\section{Gestione della sicurezza della informazione}

In questa sezione vengono descritte le componenti di un programma di
information security di successo.

Dovrebbe essere sempre presente un programma per la sicurezza della
informazione. La prima cosa da fare di solito è capire se c'è il cosiddetto
\textbf{senior management commitment}, ovvero l'impegno/interesse da parte dei
\emph{pezzi grossi} alla sicurezza dell'azienda.

Le \textbf{politiche} e le \textbf{procedure} vanno messe in atto.
Le prime sono direttive di alto livello che di solito sono pubbliche ma anche
molto astratte, mentre le seconde declinano le politiche su aspetti concreti.

Come già visto durante il \textit{disaster recovery plan}, è importante una
puntuale \textbf{allocazione delle responsabilità}, soprattutto per evitare
possibili \emph{scarichi di responsabilità}. Non è che bisogna mettere il nome
e cognome dell'addetto, ma assegnare la responsabilità ad un certo ruolo.

La \textbf{security and awareness education} è uno strumento potentissimo, che
aiuta a prevenire incidenti e frodi. Si dice \textit{awareness} quando si sa
che c'è qualcosa che non va, mentre \textit{education} è quando si è consci di
una situazione sensibile e si sa come agire.

La \textit{compliance} sono le azioni che devono essere attuate per
essere in regola con la legge. In caso di violazione si può incorrere in:
\begin{itemize}
  \item Processi interni;
  \item Penali;
  \item Denuncia (anche penale).
\end{itemize}
Inoltre è essenziale la \textbf{gestione e risposta agli incidenti},
ovvero avere un piano di risposta per gli incidenti.

Quando il programma viene definito deve essere approvato, in cui vengono
eseguiti dei confronti e vengono analizzati eventuali dubbi.

\section{Titoli comuni e responsabilità}

\begin{enumerate}
  \item \textbf{CEO/President};
  \item \textbf{CISO:} in principio non esisteva, adesso sta prendendo il
  sopravvento. Ricopre anche il ruolo del CSO;
  \begin{itemize}
        \item \textbf{Security Specialist}: Progetta, implementa,
        controlla le politiche e le procedure;
        \item \textbf{Security Administrator:} alloca le risorse per gli
        impiegati basandosi sulla documentazione scritta, oltre a monitorare
        l'accesso ai terminali e alle applicazioni e a preparare report sulla
        sicurezza. Sostanzialmente amministra i computer e si occupa
        della network security.
  \end{itemize}
  \item \textbf{CSO} (Chief Security Officier): Si occupa della
  sicurezza fisica;
  \item \textbf{Business Executive};
  \begin{itemize}
  \item \textbf{Information/Data Owner}:
  è il responsabile della sicurezza del dato e
  \begin{enumerate*}[label=\alph*)]
  \item decide chi ha accesso ai dati direttamente \textbf{o}
  \item da un permesso scritto per l'accesso
  agli amministratori della sicurezza, per prevenire
  manipolazioni e modifiche non volute.
  \end{enumerate*}


  Il data owner deve riguardare periodicamente le autorizzazioni del dato
  per restringere l'\textit{authorization creep}, ovvero quando si hanno
  delle autorizzazioni che inizialmente erano utili ma che ora non servono più.
  \item \textbf{Process Owner:} responsabile del processo (algoritmo che opera
  sul dato);
  \item \textbf{IS Auditor}: accertamento \emph{indipendente} degli
  obiettivi di sicurezza e dei controlli;
  \end{itemize}
  \item \textbf{CPO} (Chief Privacy Officier): si occupa di proteggere
  i diritti dei clienti e dei dipendenti;
  \item \textbf{Data custodian:} si preoccupa della protezione dei dati e
  potrebbe essere un analista di sistema o un amministratore di sistema.
\end{enumerate}

\section{Classificazione della criticità}

Anche la criticità ha una propria classificazione, che è:
\begin{enumerate}
  \item \textbf{Critical};
  \item \textbf{Vital};
  \item \textbf{Sensitive};
  \item \textbf{Nonsensitive}.
\end{enumerate}
(Per una descrizione dei vari livelli si rimanda alla Sezione~
\ref{sec:classificazioneServizi})

\subsection{Classificazione della sensitività}

\begin{enumerate}
\item \textbf{Confidential:} può accedervi solo chi è in una determinata lista;
\item \textbf{Private:} possono accedervi solo per persone autorizzate;
\item \textbf{Internal:} può accedervi solo chi è interno all'azienda;
\item \textbf{Public:} può accedervi chiunque.
\end{enumerate}


\subsection{Classificazione del dato}

Diversi dati possono essere classificati, ma come?

\begin{itemize}
\item Come marchiamo le informazioni classificate? Nell'informazione cartacea,
se correttamente classificata riporta un apposito timbro per esempio.
\item Chi decide che classificazione dare ai dati? Il classico problema è la
\textit{overclassification}: do una classificazione più alta di quella
necessaria al documento, il che implica spreco di risorse.
\item Come vengono trasportate/conservate/gestite/archiviate le informazioni
classificate?
\item Cosa dice la legge sulla gestione di questa informazione?
\item Chi ha l'autorità per determinare chi ha l'accesso e per cosa si ha
l'accesso? Tipicamente il CISO.
\end{itemize}

\subsection{Gestione dei dati sensibili}

Le informazioni privileged hanno un trattamento di tipo privilegiale e di
solito hanno carattere legale.

\subsubsection{Conservazione e distruzione di dati confidenziali}

\textbf{Smaltimento dei media:} usare tool sicuri per l'eliminazione dei dati.
Se i dati sono altamente sensibili si può ricorrere al \textit{Degauss} o alla
distruzione fisica.\\
\newline

\textbf{Riparazione:} se un materiale (es. le fotocopiatrici hanno una memoria
interna) viene inviato per la riparazione, prima bisogna cancellare la memoria.
\\
\newline
\textbf{Storage:} criptare i dati e mantenere i dispositivi in sicurezza.

\section{Esercizi}

Gli esercizi si trovano nella Sezione \ref{EsInfoSec1}.

\section{Tipologie di sicurezza}

\subsection{Sicurezza in profondità}

Per ridurre il rischio di un attacco in maniera facile, viene evitato di
mettere tutte le difese in un unico punto, ma vengono stratificate le difese in
diversi punti, usando tecnologie diverse.
Per arrivare al kernel, devono essere bucati prima tutti gli strati intorno.
