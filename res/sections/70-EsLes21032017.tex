\chapter{Access Control Techniques}

Teoria disponibile al Capitolo \ref{AccessControlTechniques}

\section{Password}
\label{EsPass}

Teoria disponibile alla Sezione \ref{Password}


\begin{Exercise} [
  title={Domanda},
  label={pass1}
 ]

 \Question Una forma di autenticazione biometrica che viene considerata
invasiva dagli utenti è:
\begin{enumerate}
 \item Retina
 \item Iride
 \item Scanner 3D della mano
 \item Firma
\end{enumerate}

\end{Exercise}


\begin{Answer} [
  ref={pass1},
  number={1}
  ]

  \Question Risposta esatta: 1

\end{Answer}

% Altro esercizio

\begin{Exercise} [
  title={Domanda},
  label={pass2}
 ]

 \Question La forma di biometria che è meno prona agli errori è:
\begin{enumerate}
\item Retina
\item Voce
\item Impronta digitale
\item Firma
\end{enumerate}


\end{Exercise}


\begin{Answer} [
  ref={pass2},
  number={2}
  ]

  \Question Risposta esatta: 1

\end{Answer}

% Altro esercizio

\begin{Exercise} [
  title={Domanda},
  label={pass3}
 ]

 \Question Julie è un \textit{Data Owner}. Ha configurato i permessi nel 
database in modo tale da permettere agli utenti di accedere ai campi che lei 
pensa che essi debbano avere accesso. Questa tecnica è conosciuta come:
\begin{enumerate}
 \item Modello Bell-LaPadula
 \item Mandatory access control
 \item Accesso basato sul ruolo
 \item Discretionary access control
\end{enumerate}


\end{Exercise}


\begin{Answer} [
  ref={pass3},
  number={3}
  ]

  \Question Risposta esatta: 4

\end{Answer}

% Altro esercizio

\begin{Exercise} [
  title={Domanda},
  label={pass4}
 ]

 \Question John ha un livello di sicurezza pari a (Engineering, Confidential).
Utilizzando il modello Bell-LaPadula, John può scrivere nei livelli:
\begin{enumerate}
\item Confidential
\item Top secret, secret and Confidential
\item Confidential and Unclassified
\item Unclassified
\end{enumerate}


\end{Exercise}

\begin{Answer} [
  ref={pass4},
  number={4}
  ]

  \Question Risposta esatta: 2

\end{Answer}
