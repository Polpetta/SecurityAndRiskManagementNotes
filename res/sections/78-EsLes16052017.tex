\chapter{Security Management}
\label{esSM}

Teoria disponibile in \ref{SM}

\section{COBIT}
\label{esSM:COBIT}

Teoria disponibile in \ref{COBIT}

\begin{Exercise} [
  title={Quiz},
  label={esSM1}
  ]

  \Question La differenza tra quanto un'organizzazione rende e quanto 
  vorrebbe rendere è anche conosciuta come:
\begin{enumerate}
 \item Analisi del Gap
 \item Controllo della qualità
 \item Misura della performance
 \item \textit{Benchmarking}
\end{enumerate}

\end{Exercise}

\begin{Answer} [
  ref={esSM1},
  number={1}
  ]

  \Question Risposta esatta: 1
\end{Answer}


\begin{Exercise} [
  title={Quiz},
  label={esSM2}
  ]

  \Question ``Le password devono essere lunghe almeno 14 caratteri, 
richiedere una combinazione di almeno 3 lettere minuscole, maiuscole, numeri o 
caratteri''. Questo è un esempio di:
\begin{enumerate}
 \item Standard
 \item \textit{Policy}
 \item Procedura
 \item \textit{Guideline}
\end{enumerate}

\end{Exercise}

\begin{Answer} [
  ref={esSM2},
  number={2}
  ]

  \Question Risposta esatta: 1\footnote{La risposta corretta è la 1 perché
  gli standard specificano un livello minimo di conformità obbligatoria.
  Mentre le linee guida sono una lista di raccomandazioni da seguire in
  assenza di uno standard esistente.}
\end{Answer}


\begin{Exercise} [
  title={Quiz},
  label={esSM3}
  ]

  \Question Il punto principale del COBIT o del CMM Livello 4 è:
  \begin{enumerate}
   \item \textit{Security Documentations}
   \item Metriche
   \item Rischio
   \item \textit{Business Continuity}
  \end{enumerate}
  
\end{Exercise}

\begin{Answer} [
  ref={esSM3},
  number={3}
  ]

  \Question Risposta esatta: 2
\end{Answer}


\begin{Exercise} [
  title={Quiz},
  label={esSM4}
  ]

  \Question \textit{Product testing} è per la maggior parte associato con il 
settore:
\begin{enumerate}
 \item Audit
 \item \textit{Quality Assurance}
 \item Controllo di qualità
 \item \textit{Compliance}
\end{enumerate}
  
\end{Exercise}

\begin{Answer} [
  ref={esSM4},
  number={4}
  ]

  \Question Risposta esatta: 3\footnote{La 2 non è perché è più ad alto 
livello.}
\end{Answer}


\begin{Exercise} [
  title={Quiz},
  label={esSM5}
  ]

  \Question ``Gli impiegati non devono mai aprire gli allegati e-mail, tranne 
se l'allegato è legato al business''. Questo è un esempio di un:
\begin{enumerate}
 \item \textit{Policy}
 \item Procedure
 \item \textit{Guideline}
 \item Standard
\end{enumerate}
  
\end{Exercise}

\begin{Answer} [
  ref={esSM5},
  number={5}
  ]

  \Question Risposta esatta: 4
\end{Answer}


\begin{Exercise} [
  title={Quiz},
  label={esSM6}
  ]

  \Question Le metriche \textbf{più} importanti quando si misura la 
\textit{compliance} includono:
\begin{enumerate}
 \item Metriche che possono essere facilmente automatizzate
 \item Metriche riguardanti l'\textit{intrusion detection}
 \item Quelle raccomandate dalle \textit{best practices}
 \item Metriche che misurano la conformità alla \textit{policy}
\end{enumerate}
  
\end{Exercise}

\begin{Answer} [
  ref={esSM6},
  number={6}
  ]

  \Question Risposta esatta: 4
\end{Answer}
