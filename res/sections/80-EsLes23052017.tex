\section{Security Administrator}
\label{es:SA}

La teoria riguardante questi esercizi è disponibile in \ref{SA}

\subsection{Security Operations}
\label{esSA:SO}

La teoria riguardante questi esercizi è disponibile in \ref{SA:SO}

\begin{Exercise} [
  title={Quiz},
  label={esSA1}
  ]

  \Question Chi può contribuire di più per determinare le priorità e l'impatto 
del rischio per le informazioni dell'organizzazione?
\begin{enumerate}
 \item \textit{Chief Risk Officier}
 \item \textit{Business Process Owners}
 \item \textit{Security Manager}
 \item \textit{Auditor}
\end{enumerate}

\end{Exercise}

\begin{Answer} [
  ref={esSA1},
  number={1}
  ]

  \Question Risposta esatta: 2\footnote{Gli owner sono quelli che hanno 
l'informazione, quindi è lui quello che ha tutti i dati!}
\end{Answer}


\begin{Exercise} [
  title={Quiz},
  label={esSA2}
  ]

  \Question Un documento che descrive come i permessi sono definiti e allocati 
è il:
\begin{enumerate}
 \item \textit{Data Classification}
 \item \textit{Acceptable Usage Policy}
 \item \textit{End-User Computing Policy}
 \item \textit{Access Control Policies}
\end{enumerate}
  
\end{Exercise}

\begin{Answer} [
  ref={esSA2},
  number={2}
  ]

  \Question Risposta esatta: 4
\end{Answer}


\begin{Exercise} [
  title={Quiz},
  label={esSA3}
  ]

  \Question Il ruolo di un \textit{Information Security Manager} in relazione 
alla \textit{security strategy} è:
\begin{enumerate}
 \item Autore primario con input del business
 \item Comunicatore agli altri dipartimenti
 \item \textit{Reviewer}
 \item Approva la strategia
\end{enumerate}
  
\end{Exercise}

\begin{Answer} [
  ref={esSA3},
  number={3}
  ]

  \Question Risposta esatta: 1
\end{Answer}



\begin{Exercise} [
  title={Quiz},
  label={esSA4}
  ]

  \Question Il ruolo con più probabilità di testare un controllo è:
  \begin{enumerate}
   \item l'Amministratore di sicurezza
   \item l'Architetto della sicurezza
   \item il \textit{Quality Control Analyst}
   \item il \textit{Security Steering Committee}
  \end{enumerate}

\end{Exercise}

\begin{Answer} [
  ref={esSA4},
  number={4}
  ]

  \Question Risposta esatta: 1\footnote{Nella domanda 3, quel "quality control" 
non è il controllo della qualità in senso di sicurezza, ma è in senso più 
generico}
\end{Answer}


\begin{Exercise} [
  title={Quiz},
  label={esSA5}
  ]

  \Question Il ruolo responsabile di definire gli obiettivi di sicurezza e 
instituire un'organizzazione di sicurezza è:
\begin{enumerate}
 \item \textit{Chief Security Officer}
 \item \textit{Executive Management}
 \item \textit{Board of Directors}
 \item \textit{Chief Information Security Officer}
\end{enumerate}

\end{Exercise}

\begin{Answer} [
  ref={esSA5},
  number={5}
  ]

  \Question Risposta esatta: 2
\end{Answer}


\begin{Exercise} [
  title={Quiz},
  label={esSA6}
  ]

  \Question Implementando un controllo, la guida \textbf{primaria} per 
l'implementazione aderisce a:
\begin{enumerate}
 \item \textit{Organization policy}
 \item Framework di sicurezza come il COBIT, NIST, IOS/IEC
 \item \textit{Prevention Detection, Correction}
 \item Una difesa a strati
\end{enumerate}
  
\end{Exercise}

\begin{Answer} [
  ref={esSA6},
  number={6}
  ]

  \Question Risposta esatta: 1
\end{Answer}


\begin{Exercise} [
  title={Quiz},
  label={esSA7}
  ]

  \Question Le persone nel \textit{Security Steering Committee} che possono
fornire le informazioni più utili per assicurare il successo 
del \textit{Information Security} sono:
\begin{enumerate}
 \item \textit{Chief Information Security Officier}
 \item \textit{Business Process Owners}
 \item \textit{Executive Management}
 \item \textit{Chief Information Officer}
\end{enumerate}
  
\end{Exercise}

\begin{Answer} [
  ref={esSA7},
  number={7}
  ]

  \Question Risposta esatta: 2
\end{Answer}
