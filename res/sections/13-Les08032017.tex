\section{Classificazione dell'evento danneggiante}

È importante classificare i danni subiti secondo un costo.
Gli eventi danneggianti possono essere classificati. Sono riportati in ordine 
crescente diverse categorie:

\begin{itemize}
\item Negligible \\
Costo non significativo
\item Minor \\
Non ignorabile ma con costo senza impatto
\item Major \\
Impatto su uno o più dipartimenti e può impattare sui clienti esterni
\item Crisis \\
Evento che ci costa la permanenza sul mercato
\end{itemize}

Questa classificazione non è semplice, anche perché a ogni categoria ha una 
classe di costi e contromisure diverse. Si può avere una sovrastima e una 
sottostima dei costi, che corrispettivamente portano ad un amento dei costi 
eccessivi o a una sicurezza non adeguata.
Minor, major e crisis dovrebbero essere documentati e tracciati per poter 
rimediare.

\subsection{Disastro e impatto}

Supponiamo di essere in una Università.
Potrebbero accedere i seguenti disastri:
\begin{itemize}
 \item Incendio \\
 Questo impatta sulle classi e il dipartimento. Come lo 
classifichiamo? Crisi, potrebbe essere major se limitato ad una parte del 
dipartimento. Attenzione: c'è la vita umana in rischio e quindi un forte 
rischio da tenere in considerazione.

 \item Attacco hacker \\
 Attaccano per esempio il sistema di registrazione dei voti. Può essere un 
evento major che ha implicazioni legali (non \`e possibile registrare voti).

 \item Indisponibilità dalla rete \\
 Livello di disastro: crisi. I dipartimenti dipendono dalla rete perché 
ci lavorano e questo blocca completamente il dipartimento. Oggi una rete viene 
considerata come il sistema nervoso di una organizzazione. Senza quello, 
l'organizzazione \`e costretta a fermarsi (blocco di registrazioni online ad 
esempio con impossibilità degli studenti di seguire).

 \item Social engineering \\
 Può esserci un attacco al sistema di registrazione magari tramite una frode 
(mi prendono password e token).

 \item Server failure \\
 Collasso dei server. Si hanno le stesse conseguenze di network unavailable. 
Classificato come major (crisi in alcuni casi, ad esempio durante una sessione 
d'esami). Al giorno d'oggi \`e possibile contenere il disastro, e queste 
contromisure sono standard e pure a basso costo, in quanto sono hardware.
\end{itemize}

\subsection{Recupero del servizio}

Per attuare un recupero è possibile che siano necessarie anche due settimane. 
Questo perché sono involti processi che vanno capiti e contestualizzati.

% Aggiungere grafico appena le slide sono disponibili

Abbiamo un servizio regolare, si verifica un'interruzione e ci troviamo 
nell'\textbf{interruption window} (finestra di tempo per cui il servizio non è 
disponibile). Se è stato fatto un \textit{disaster recovery plan} è 
possiamo entrare in \textbf{alternate mode} (servizio più scarno\footnote{Per 
esempio se era possibile servire 100 utenti contemporaneamente magari adesso 
\`e possibile servirne solamente 50.} ma disponibile).

Il \textbf{Service Delivery Objective} è il livello di servizio che possiamo 
fornire in \textit{alternate mode} (ad esempio una banca vuole che sia 
garantito almeno il 50\% delle transazioni durante la alternate mode, quindi 
l'SDO è 50\%).

La modalità alternativa ovviamente non perdura costantemente, ed ad un certo 
punto raggiunge una fine in cui il comportamento normale del sistema viene 
ripristinato. Il tempo passato tra l'inizio della modalità alternativa e la 
sua conclusione con il ripristino delle normali operazioni\footnote{Anche 
detto \textit{restoration}} viene definito come \textbf{Maximum Tolerable 
Outage}.

È possibile che non sia possibile ritornare subito al 100\% dell'operatività. 
È quindi necessario avere un \textbf{restoration plan} che permetta di passare 
dall'\textit{alternate mode} al \textit{regular service mode} in minor tempo 
possibile.

\subsubsection{Definizioni}

Di seguito viene riportata una lista di definizioni appena viste:

\begin{itemize}
 \item \textbf{Business Continuity} \\
 Riuscire ad operare anche se con una modalità degradata del servizio.
 \item \textbf{Disaster recovery} \\
 Permette di sopravvivere anche se il servizio IT è non disponbile
 \item \textbf{Alternate Process Mode} 
 Modalità alternata.
 \item \textbf{Disaster Recovery mode} \\
 Piano che permette di passare in modalità alternata.
 \item \textbf{Restoration plan} \\
 Permette di passare dalla modalità alternata a quella normale.
\end{itemize}

\section{Classificazione dei servizi}

Il servizio ha un costo associato, di conseguenza anche la mancanza erogazione 
del servizio lo ha. In ordine di importanza:
\begin{enumerate}
 \item[\textbf{Critical}] Non posso sostituirlo con un processo non IT, non si 
deve fermare. Costo molto elevato. \\
Esempio: telemedicina, un medico sta al MIT 
e tramite telemedicina opera qualcuno a Ginevra, chiaramente il servizio non 
può essere interrotto. Questo servizio implica la \textit{safety}, ogni 
servizio che implica la safety è critica per definizione.

 \item[\textbf{Vital}] Andare avanti a mano per un tempo molto ridotto. Costo 
elevato.

 \item[\textbf{Sensitive}] Posso sostituire all'IT una procedura manuale per un 
periodo limitato di tempo ma mi costa di più. 

 \item[\textbf{Nonsensitive}] Posso andare avanti manualmente per un periodo 
esteso di tempo. 

\end{enumerate}

Nelle aziende di solito è sempre presente un processo critico. È strano se non 
ce n'è neanche uno, ma dipende molto dalla realtà.

\subsection{Esempio: determinazione di criticità nei processi di business}

\begin{figure}[H]
 \centering
 \includegraphics[scale=0.42]{dcobp}
 \caption{Una grande compagnia con le seguenti sotto-componenti}
\end{figure}

Il servizio più importante è \textit{sales}, perché non si può spedire senza 
vendere. Per tutte le aziende le vendite sono critiche perché altrimenti non 
fatturano.

Sotto alle sales abbiamo \textit{web service} and \textit{sales calls}. Tra i 
due web service è critico perché non può essere sostituito manualmente, mentre 
le sales calls possono essere segnate a mano.

%Nelle sales è più importante orders che inventory perché... % TODO Finire

Gli ingegneri sono a bassa priorità perché anche se non lavorano per una 
settimana sullo sviluppo di un prodotto semplicemente viene ritardato di una 
settimana. Ovviamente bisogna tenere in considerazione il fatto che se c'è una 
\textit{deadline} vicina il rischio sale.

\subsection{RPO and RTO}

\begin{itemize}
 \item \textbf{Recovery Point Objective}: ammontare dei dati che possiamo 
perdere quando recuperiamo dal backup.

 \item \textbf{Recovery Time Objective}: tempo che serve per recuperare le 
applicazioni e ripartire in modo da poter ripristinare il servizio.

 \item \textbf{Orphan data}: dati che vanno persi e non sono recuperabili.
\end{itemize}

Se voglio avere \textit{RPO} basso devo avere una spesa maggiore (es. se lo 
voglio minore vuol dire che devo effettuare backup più spesso, che significa 
rendere non disponibile il dato su cui sto effettuando backup, ossia chi deve 
lavorarci non può farlo).
\textit{RTO} è ovvio invece: più velocemente vogliamo essere operativi più 
questo costa.

\subsection{Business Impact Analysis Summary}

Per fare un riassunto della unità viene qui proposto un esempio. In una 
università, potremmo avere la seguente \textit{Business Impact Analysis 
Summary}\footnote{Per comodità esiste anche l'abbreviazione BIA.}

\begin{table}[H]
\centering
\resizebox{\textwidth}{!}{%
\begin{tabular}{|l|l|l|l|p{5cm}|}
\hline
\textbf{Servizio}      & \textbf{RPO (ore)} & \textbf{RTO (ore)} & 
\textbf{Risorse critiche}           & \textbf{Note speciali}                     
                           \\
\hline
Registrazione & 0 ore     & 4 ore     & SOLAR, rete, Registratore  & Alta 
priorità durante Nov - Gennaio, Maggio - Giugno, Agosto \\
\hline
Personale     & 2 ore     & 8 ore     & PeopleSoft                 & Potrebbe 
operare manualmente per alcune ore                  \\
\hline
Insegnamenti  & 1 giorno  & 1 ora     & D2L, rete, file di facoltà & Durante il 
semestre scolastico: alta priorità               \\
\hline
\end{tabular}%
}
\caption{Un esempio di BIA per una Università}
\end{table}


\section{Tipologie di \textit{disaster recovery}}

\subsection{RAID}

Sistema di ridondanza per i dati. Sono sei con diverse modalità.
\begin{itemize}
  \item RAID 0: i dischi anziché essere memorizzati su un disco sono 
  memorizzati su due dischi. Possibile accesso parallelo ma niente backup 
  (striping).
  
  \item RAID 1: backup su un altro disco (mirroring).
  
  \item Altri raid
  
  \item RAID 5: si fa striping e redundancy. Ho più dischi e faccio striping 
  con in aggiunta dei bit di parità.
  
\end{itemize}

\subsection{Network Disaster recovery}

Per recuperare un problema di rete la cosa più semplice è diversificare (avere 
ridondanza, una rete alternativa). Due tipi di ridondanza:
\begin{itemize}
  \item \textbf{alternative routing}: faccio un contratto con una società che 
  mi fornisce un altro mezzo fisico che esce dalla mia azienda. Bisogna essere 
sicuri che il mezzo che fornisce questa seconda società non sia lo stesso mezzo 
fisico, altrimenti è inutile;
  \begin{itemize}
    \item last-mile circuit protection: se mi tranciano l'ultimo miglio tra 
l'azienda e la centralina posso mantenere comunicazioni tramite 
\textit{microwave communication}\footnote{Comunicazione a Microonde.}, faccio 
comunicazione punto punto. Questa tecnologia viene usata abbastanza spesso in 
quanto non richiede una particolare installazione ed è abbastanza economica.
    \item long-haul network diversity
  \end{itemize}
  \item \textbf{diverse routing}: si fa con la stessa società ma questa società 
garantisce un instradamento diverso.
\end{itemize}

\subsection{Rottura vs Costi di ripristino}

Questi due obiettivi sono ovviamente conflittuali.

\begin{figure}[H]
 \centering
 \includegraphics[scale=0.45]{dvsrc}
 \caption{Relazione tra una rottura e i costi di ripristino del servizio}
\end{figure}

Asse x: tempo di disservizio del servizio

Asse y: costo

Esistono diverse tecnologie di backup:
\begin{itemize}
  \item \textbf{Hot site}: il centro di \textit{backup} è allineato con il 
  centro operativo (es. gestione dei satelliti, ho due siti, uno hot e l'altro 
  operativo). Il tempo di recupero da \textit{downtime} è molto basso ma il 
  costo è molto alto. Il costo è molto alto perché bisogna duplicare 
  l'infrastruttura e il costo dell'\textit{information processes}.
  
  \item \textbf{Warm site}: c'è un altro sito che ha tutti gli assets pronti 
  per lavorare ma non è ancora operativo.
  
  \item \textbf{Cold site}: abbiamo un sito predisposto per l'emergenza ma che 
  di fatto è vuoto. Ad esempio gli attacchi per il computer sono pronti ma 
  mancano i server e le configurazioni. Questa installazione è quella che 
  richiede più tempo per diventare operativo, anche se il suo costo è molto 
  basso.
  
  \item Le \textbf{alternative recovery strategies} che hanno costi molto bassi 
  di recupero ma si trovano in una fascia operativa media. Due società nello 
  stesso segmento di mercato ma non troppo in competizione si potrebbero 
  accordare per condividere i siti caldi. Questo rientra nella fase di 
  \textbf{reciprocal agreeement}.
  
  
  \item \textbf{Mobile site}: un mezzo mobile viene con un'antenna sul sito e 
  garantisce la connettività.
  
\end{itemize}


\subsection{Cloud computing}

Il \textit{cloud computing} è caratterizzato da elasticità e scalabilità. Posso 
scalare le risorse che mi servono, sia in positivo che in negativo perchè non 
sono regolari. Permette un'azienda di liberarsi del proprio reparto macchine 
(no server fisici in loco).

\paragraph*{Vantaggi} Agilità, efficienza del costo, \textit{deployment} veloce 
e alta scalabilità.

\paragraph*{Svantaggi} Costi customizzati, usabilità, connettività (mia e di 
chi fornisce il \textit{cloud computing}), sicurezza.


\begin{figure}[H]
 \centering
 \includegraphics[scale=0.45]{itc}
 \caption{Una visione generale di come funziona un servizio \textit{cloud}}
\end{figure}

\subsubsection{Modello di dispiegamento}

Esistono diversi metodi di dispiegamento:
\begin{itemize}
  \item Pubblico: pago e accedo al cloud (ad esempio Amazon AWS).
  
  \item Privato: creo un mio \textit{datacenter} e lo utilizzo per i miei 
  servizi (ad esempio il datacenter di una banca).
  
  \item Ibrido: via di mezzo tra public e private.
  
  \item Comunitario: raggruppa organizzazioni che hanno lo stesso tipo di 
  interessi o problematiche. Privato ma ``allargato''.
  
\end{itemize}

Da tenere sempre d'occhio è la sicurezza di questi servizi: quando si usa il 
\textit{cloud} do i miei dati in mano a terzi. Un altro problema è la 
\textit{multi tenancy}, ovvero quando vengono eseguiti più servizi di entità 
diverse sulla stessa macchina grazie alla virtualizzazione. In caso di 
fallimento di una macchina potrebbero fallire più servizi di diversi utenti, 
per non parlare del fatto che un utente malintenzionato potrebbe tentare di 
accedere al servizio di un altro utente risiedente sulla stessa istanza.

\subsubsection{Modello di servizio}

Esistono diversi modelli di servizio:
\begin{itemize}
  \item Software (SaaS): si fa girare applicazioni di terze parti sul cloud.
  \item Platform (PaaS): viene fornito il sistema e l'ambiente di sviluppo, ci 
  faccio girare le mie applicazioni. 
  \item Infrastructure (IaaS): viene fornita la macchina e l'azienda ci mette 
  sopra quello di cui ha bisogno.
  
\end{itemize}

\subsubsection{Caratteristiche essenziali}

Qui di seguito sono elencate le caratteristiche essenziali che un servizio 
cloud offre:
\begin{itemize}
  \item Resource pooling
  \item Broad network access
  \item Rapid elasticity
  \item Measured service
  \item On-demand self-service
\end{itemize}

\subsubsection{Major Areas of Security Concerns}

\begin{itemize}
 \item \textbf{Service Level Agreement (SLA)}: definisce performance, 
sicurezza, policy, disponibilità e molte altre cose. Esempio: contratto di 
fornitura di internet, il SLA di solito è la banda minima garantita.

 \item Si ha \textbf{Multi-tenancy} quando l'applicazione è sullo stesso server 
usato da altre organizzazioni. È quindi necessario avere un'adeguata 
segmentazione e isolazione. Un set di \textit{policy} sono anche necessarie.

 \item \textbf{Your coverage}: la sicurezza dipende dall'infrastruttura 
dell'azienda e da quella del cloud provider.
\end{itemize}

Non è possibile trasferire la responsabilità dei servizi nei confronti dei 
propri clienti. Es: dichiarazione dei redditi con il commercialista, il 
commercialista fa la dichiarazione ma io la firmo e di conseguenza sono il 
responsabile legale.

\subsubsection{Accordo reciproco}

\paragraph*{Vantaggi} Costo basso. 

\paragraph*{Svantaggi}La sicurezza. Processi veramente critici vanno posti a 
distanza geografica perché se capita una catastrofe è molto probabile che venga 
colpito anche chi è vicino a me (suscettibilità allo stesso disastro).