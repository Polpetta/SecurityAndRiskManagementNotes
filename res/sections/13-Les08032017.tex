\section{Classificazione dell'evento danneggiante}

È importante classificare i danni subiti secondo un costo.
Gli eventi danneggianti possono essere classificati. Sono riportati in ordine 
crescente diverse categorie:

\begin{itemize}
\item Negligible \\
Costo non significativo
\item Minor \\
Non ignorabile ma con costo senza impatto
\item Major \\
Impatto su uno o più dipartimenti e può impattare sui clienti esterni
\item Crisis \\
Evento che ci costa la permanenza sul mercato
\end{itemize}

Questa classificazione non è semplice, anche perché a ogni categoria ha una 
classe di costi e contromisure diverse. Si può avere una sovrastima e una 
sottostima dei costi, che corrispettivamente portano ad un amento dei costi 
eccessivi o a una sicurezza non adeguata.
Minor, major e crisis dovrebbero essere documentati e tracciati per poter 
rimediare.

\subsection{Disastro e impatto}

%%TODO da trasformare in tabella
Supponiamo di essere in una Università.
Potrebbero accedere i seguenti disastri:
\begin{itemize}
 \item Incendio \\
 Questo impatta sulle classi e il dipartimento. Come lo 
classifichiamo? Crisi, potrebbe essere major se limitato ad una parte del 
dipartimento. Attenzione: c'è la vita umana in rischio e quindi un forte 
rischio da tenere in considerazione.

 \item Attacco hacker \\
 Attaccano per esempio il sistema di registrazione dei voti. Può essere un 
evento major che ha implicazioni legali (non \`e possibile registrare voti).

 \item Indisponibilità dalla rete \\
 Livello di disastro: crisi. I dipartimenti dipendono dalla rete perché 
ci lavorano e questo blocca completamente il dipartimento. Oggi una rete viene 
considerata come il sistema nervoso di una organizzazione. Senza quello, 
l'organizzazione \`e costretta a fermarsi (blocco di registrazioni online ad 
esempio con impossibilità degli studenti di seguire).

 \item Social engineering \\
 Può esserci un attacco al sistema di registrazione magari tramite una frode 
(mi prendono password e token).

 \item Server failure \\
 Collasso dei server. Si hanno le stesse conseguenze di network unavailable. 
Classificato come major (crisi in alcuni casi, ad esempio durante una sessione 
d'esami). Al giorno d'oggi \`e possibile contenere il disastro, e queste 
contromisure sono standard e pure a basso costo, in quanto sono hardware.
\end{itemize}

\subsection{Recupero del servizio}

Per attuare un recupero è possibile che siano necessarie anche due settimane. 
Questo perché sono involti processi che vanno capiti e contestualizzati.

% Aggiungere grafico appena le slide sono disponibili

Abbiamo un servizio regolare, si verifica un'interruzione e ci troviamo 
nell'\textbf{interruption window} (finestra di tempo per cui il servizio non è 
disponibile). Se è stato fatto un \textit{disaster recovery plan} è 
possiamo entrare in \textbf{alternate mode} (servizio più scarno\footnote{Per 
esempio se era possibile servire 100 utenti contemporaneamente magari adesso 
\`e possibile servirne solamente 50.} ma disponibile).

Il \textbf{Service Delivery Objective} è il livello di servizio che possiamo 
fornire in \textit{alternate mode} (ad esempio una banca vuole che sia 
garantito almeno il 50\% delle transazioni durante la alternate mode, quindi 
l'SDO è 50\%).

La modalità alternativa ovviamente non perdura costantemente, ed ad un certo 
punto raggiunge una fine in cui il comportamento normale del sistema viene 
ripristinato. Il tempo passato tra l'inizio della modalità alternativa e la 
sua conclusione con il ripristino delle normali operazioni\footnote{Anche 
detto \textit{restoration}} viene definito come \textbf{Maximum Tolerable 
Outage}.

È possibile che non sia possibile ritornare subito al 100\% dell'operatività. 
È quindi necessario avere un \textbf{restoration plan} che permetta di passare 
dall'\textit{alternate mode} al \textit{regular service mode} in minor tempo 
possibile.

\subsubsection{Definizioni}

Di seguito viene riportata una lista di definizioni appena viste:

\begin{itemize}
 \item \textbf{Business Continuity} \\
 Riuscire ad operare anche se con una modalità degradata del servizio.
 \item \textbf{Disaster recovery} \\
 Permette di sopravvivere anche se il servizio IT è non disponbile
 \item \textbf{Alternate Process Mode} 
 Modalità alternata.
 \item \textbf{Disaster Recovery mode} \\
 Piano che permette di passare in modalità alternata.
 \item \textbf{Restoration plan} \\
 Permette di passare dalla modalità alternata a quella normale.
\end{itemize}

\section{Classificazione dei servizi}

Il servizio ha un costo associato, di conseguenza anche la mancanza erogazione 
del servizio lo ha. In ordine di importanza:
\begin{enumerate}
 \item[\textbf{Critical}] Non posso sostituirlo con un processo non IT, non si 
deve fermare. Costo molto elevato. Esempio: telemedicina, un medico sta al MIT e 
tramite telemedicina opera qualcuno a Ginevra, chiaramente il servizio non può 
essere interrotto. Questo servizio implica la \textit{safety}, ogni servizio che 
implica la safety è critica per definizione.

 \item[\textbf{Vital}] Andare avanti a mano per un tempo molto ridotto. Costo 
elevato.

 \item[\textbf{Sensitive}] Posso sostituire all'IT una procedura manuale per un 
periodo limitato di tempo ma mi costa di più.

 \item[\textbf{Nonsensitive}] Posso andare avanti manualmente per un periodo 
esteso di tempo.


\end{enumerate}
\subsection{Esempio: determine criticality of business processes}

Abbiamo una grande compagnia con le seguenti sotto-componenti (vedi immagine con 
albero che ha per radice corporate).

Il servizio più importante è \textit{sales}, perché non si può spedire senza 
vendere. Per tutte le aziende le vendite sono critiche perché altrimenti non 
fatturano.

Sotto alle sales abbiamo \textit{web service} and \textit{sales calls}. Tra i 
due web service è critico perché non può essere sostituito manualmente, mentre 
le sales calls possono essere segnate a mano.

Nelle sales cales è più importante orders che inventory perché... %%TODO Finire

Gli ingegneri sono a bassa priorità perché anche se non lavorano per una 
settimana sullo sviluppo di un prodotto semplicemente viene ritardato di una 
settimana.

\subsection{RPO and RTO}

\begin{itemize}
 \item \textbf{Recovery Point Objective:} ammontare dei dati che possiamo 
perdere quando recuperiamo dal backup.

 \item \textbf{Recovery Time Objective:} tempo che serve per recuperare le 
applicazioni e ripartire in modo da poter ripristinare il servizio.

 \item \textbf{Orphan data:} dati che vanno persi e non sono recuperabili.
\end{itemize}

Se voglio avere RPO/RTO bassi devo avere una spesa maggiore (es. se voglio RPO 
minore vuol dire che devo effettuare backup più spesso, che significa rendere 
non disponibile il dato su cui sto effettuando backup, ossia chi deve lavorarci 
non può farlo).

\subsection{Business Impact Analysis Summary}

%%TODO Copiare la tabella

\section{RAID}

Sistema di ridondanza per i dati.

RAID 0: i dischi anziché essere memorizzati su un disco sono memorizzati su due 
dischi. Possibile accesso parallelo ma niente backup (striping).

RAID 1: backup su un altro disco (mirroring).

...

RAID 5: si fa striping e redundancy. Ho più dischi e faccio striping con in 
aggiunta dei bit di parità.

\section{Network Disaster recovery}

Per recuperare un problema di rete la cosa più semplice è diversificare (avere 
ridondanza, una rete alternativa). Due tipi di ridondanza
\begin{itemize}
\item alternative routing: faccio un contratto con una società che mi fornisce 
un altro mezzo fisico che esce dalla mia azienda. Bisogna essere sicuri che il 
mezzo che fornisce questa seconda società non sia lo stesso mezzo fisico, 
altrimenti è inutile;
\begin{itemize}
\item last-mile circuit protection: se mi tranciano l'ultimo miglio tra 
l'azienda e la centralina posso mantenere comunicazioni tramite microwave 
communication, faccio comunicazione punto punto.
\item long-haul network diversity
\end{itemize}
\item diverse routing: si fa con la stessa società ma questa società garantisce 
un instradamento diverso.
\end{itemize}

\section{Disruption vs Recovery costs}

%%TODO copiare il grafico

Asse x: tempo di disservizio del servizio
Asse y: costo

\textbf{Hot site:} il centro di backup è allineato con il centro operativo (es. 
gestione dei satelliti, ho due siti, uno hot e l'altro operativo). Il tempo di 
recupero da downtime è molto basso ma il costo è molto alto.

\textbf{Warm site:} c'è un altro sito che ha tutti gli assets pronti per 
lavorare ma non c'è nessuno nel sito di backup.

\textbf{Cold site:} abbiamo un sito predisposto per l'emergenza ma che di fatto 
è vuoto. Es: gli attacchi sono pronti ma mancano i server e le configurazioni.

Ci sono le \textbf{alternative recovery strategies} che hanno costi molto bassi 
di recupero. Es: due società si accordano per condividere i siti caldi. Questo 
rientra nella fase di \textbf{reciprocal agreeement}.

\textbf{Mobile site:} un mezzo mobile viene con un'antenna sul sito e 
garantisce la connettività.

\section{Cloud computing}

Il cloud computing è caratterizzato da elasticità e (?). Posso scalare le 
risorse che mi servono, sia in positivo che in negativo. Permette un'azienda di 
liberarsi del proprio reparto macchine (no server fisici in loco).

Vantaggi: agilità, efficienza del costo, deployment veloce e alta scalabilità.

Svantaggi: costi customizzati, usabilità, connettività (mia e di chi fornisce 
il cloud computing), sicurezza.

\subsection{Modello di dispiegamento}

Public: pago e accedo al cloud.

Private: creo un mio datacenter e lo utilizzo per i miei servizi.

Hybrid: via di mezzo tra public e private.

Community: raggruppa organizzazioni che hanno lo stesso tipo di interessi o 
problematiche. Privato ma "allargato".

\subsection{Modello di servizio}

Software(SaaS): si fa girare applicazioni di terze parti sul cloud.

Platform(PaaS): viene fornito il sistema e l'ambiente di sviluppo, ci faccio 
girare le mie applicazioni. 

Infrastructure(IaaS): viene fornita la macchina e l'azienda ci mette sopra 
quello di cui ha bisogno.

\subsection{Caratteristiche essenziali}

Resource pooling

Broad network access

Rapid elasticity

altro... vedi schema sul cloud.

\subsection{Major Areas of Security Concerns}

Service Level Agreement (SLA): definisce performance, sicurezza, policy, 
disponibilità e molte altre cose. Esempio: contratto di fornitura di internet, 
il SLA di solito è la banda minima garantita.

Multi-tenancy: la mia applicazione è sullo stesso server usato da altre 
organizzazione. Servono segmentazione, isolazione e delle policy.

Your coverage: la sicurezza dipende dall'infrastruttura dell'azienda e da 
quella del cloud provider.

Non è possibile trasferire la responsabilità dei servizi nei confronti dei 
propri clienti. Es: dichiarazione dei redditi con il commercialista, il 
commercialista fa la dichiarazione ma io la firmo e di conseguenza sono il 
responsabile legale.

\subsection{Accordo reciproco}

Vantaggio: costo basso. Ha anche diversi svantaggi, tra cui la sicurezza.
Processi veramente critici vanno posti a distanza geografica perché se capita 
una catastrofe è molto probabile che venga colpito anche chi è vicino a me 
(suscettibilità allo stesso disastro).