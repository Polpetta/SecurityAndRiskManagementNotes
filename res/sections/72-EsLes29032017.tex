\chapter{Networking}
\label{EsNet}

La teoria è disponibile alla sezione \ref{net}

\section{Networking security}

\subsection{Hacking networks}

\label{EsHacknet}

\begin{Exercise} [
  title={Quiz},
  label={net1}
  ]

  \Question Un attacco dove molteplici computer mandano pacchetti di
connessione a un server in maniera simultanea per rallentare il
\textit{firewall} si chiama:
\begin{enumerate}
 \item DDoS
 \item \todo{Finire l'esercizio, le altre opzioni sono mancanti}
\end{enumerate}

\end{Exercise}


\begin{Answer} [
  ref={net1},
  number={1}
  ]

  \Question Risposta esatta: 1

\end{Answer}

% Altro esercizio


\begin{Exercise} [
  title={Quiz},
  label={net2}
  ]

  \Question Un attacco di tipo \textit{man-in-the-middle} è implementato con
quale metodo addizionale di attacco?
\begin{enumerate}
 \item Spoofing
 \item DoS
 \item Phishing
 \item Pharming
\end{enumerate}

\end{Exercise}


\begin{Answer} [
  ref={net2},
  number={2}
  ]

  \Question Risposta esatta: 1

\end{Answer}
