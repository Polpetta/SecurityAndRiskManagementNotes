\subsection{Segregation of duties}

Questo concetto è molto importante.

La \textit{conformità} è più importante della \textit{sicurezza} perché bisogna 
essere conformi a degli standard.

È importante, come già detto in orecedenza, separare i vari 
poteri tra le persone, in maniera tale da assicurarsi che non avvengano ``abusi 
di potere'' (es. l'amministratore di rete concede i diritti, ma qualcuno deve autorizzare che il 
ruolo $x$ necessiti dei diritti. l'amministratore implementa la concessione). 

È importante anche che gli \textit{asset} vengano controllati. Solitamente gli 
\textit{asset} sono fisici, ma nel settore IT possono essere immateriali (come 
un file contente un disegno di un progetto per una architettura di rete). Quindi 
è importante documentare le responsabilità. Si è responsabili quando occorrono 
\textit{due condizioni:} quando si è nominati ufficialmente (tramite una delega formale) o
quando si hanno i poteri per implementare correttamente quanto richiesto.

\subsubsection{Personale}

Dal punto di vista della sicurezza il personale è sempre l'anello debole. 
Il \textit{background check} dipende sempre in base alla mansione che 
dev'essere coperta. Questi controlli sono abbastanza sensibili.
Se viene maneggiato denaro o \textit{asset} sicuramente è necessario eseguire 
un \textit{background check} per assicurarsi la \textit{professionalità} e 
l'\textit{affidabilità} della persona (anche essere coinvolti in una stupida 
rissa potrebbe compromettere l'affidabilità di una persona). Di conseguenza, i propri 
profili social dovrebbero essere sempre controllati in maniera 
tale da non contenere contenuto non appropriato.

Il \textit{monitoring} delle email è un problema ovunque, tranne negli USA: 
lì le email non sono protette. L'azienda fa lo scan dei PC dei 
dipendenti. In Italia, dopo un caso eclatante, la giurisprudenza permette di 
leggere le email di lavoro.

\paragraph{Assunzione degli impiegati}

Bisogna fare attenzione quando si eseguono assunzioni: è 
importante anche controllare che chi viene assunto non esegua manomissioni dei 
documenti ufficiali\footnote{La falsificazione di documenti nella pubblica 
amministrazione comporta una reclusione dai 2 ai 5 anni.} (es. dichiarate di 
essere laureato senza una laurea).

Il \textbf{mansionario} specifica cosa un impiegato deve fare.

\textbf{Confidentiality agreement}: contratto in cui il lavoratore dichiara di 
non divulgare nulla di quello su cui lavora l'azienda e indica anche un limite di tempo 
per cui l'impiegato non può svolgere un lavoro nello stesso campo.


\paragraph{Orientamento dei nuovi impiegati}

Quando un nuovo impiegato firma un documento ha letto e accettato le policy di 
sicurezza. Inoltre si impegna a:
\begin{itemize}
\item non divulgare ID e password per il login;
\item creare password di qualità;
\item bloccare il terminale quando non è presente;
\item riportare violazioni della sicurezza sospette;
\item mantenere una buona sicurezza fisica.
\end{itemize}

\paragraph{Employee termination}

Quando si arriva al termine del periodo lavorativo si deve ritornare 
l'equipaggiamento aziendale e revocare l'accesso all'individuo. Badge e 
cartellini devono essere riconsegnati; le autorizzazioni che l'individuo aveva 
all'interno dell'azienda devono essere revocate.

\subsubsection{Accordi tra terze parti}

Definiscono le \textit{policy} di sicurezza relative alle informazioni e le 
procedure per implementare le \textit{policy}; forniscono controlli per 
proteggersi contro software malevolo. Pubblicano restrizioni per la copia e la 
distribuzione delle informazioni, implementano procedure per determinare se gli 
\textit{asset} sono stati compromessi e assicurano il ritorno o la distruzione 
dei dati al termine del rapporto lavorativo.

\subsubsection{Riassunto dei controlli sul personale}

I punti importanti del controllo sul personale sono:
\begin{itemize}
\item Segregation of Duties;
\item Vacanze obbligatorie e rotazione delle mansioni;
\item Addestramento delle policy e procedure;
\item Controlli del background;
\item Need to know/Least privilege;
\item Meccanismo di segnalazione di frodi.
\end{itemize}

\section{Esercizi}

Gli esercizi sono disponibili nella parte \ref{esSFDP:generali}

\part{Security Management}
\label{SM}

\paragraph{Obiettivi}

Gli obiettivi di questa parte del corso sono:
\begin{itemize}
\item Quality assurance;
\item Capire i ruoli di CISO, CIO, CSO Board of directors, Executive 
Management, ecc;
\item Definire delle baseline sulla sicurezza ;
\item Descrivere COBIT, CMM (vecchio), livelli 1-5.
\end{itemize}

\chapter{COBIT, CMM}

\textbf{CMM} sta per \textit{Capability Maturity Model}. Si è capito che lo 
sviluppo del software è cruciale nelle società che lo sviluppano, che vengono 
mappate in 5 livelli dove ogni livello identifica la maturità di un'azienda.
I cinque livelli sono:
\begin{enumerate}
 \item Initial: si ``vive alla giornata''. Si è estremamente reattivi agli 
eventi che accadono e non è in atto alcun tipo di pianificazione/processo
 \item Managed: si ha ancora un approccio di tipo reattivo agli eventi che 
accadono, ma si comincia ad definire processi ed ad avere pianificazioni per i 
progetti
 \item Defined: si cominciano a definire processi anche per l'organizzazione in 
toto e si comincia ad avere un approccio proattivo ai problemi
 \item Qualitatively Managed: vengono applicate misurazioni e metriche ai 
processi per poter controllarli e migliorarli continuamente
 \item Optimized: una volta che è in atto il miglioramento continuo dei 
processi che vengono a loro volta misurati tramite metriche l'azienda può 
puntare ad ottimizzarli il più possibile.
\end{enumerate}
Il CMM è stato superato dal CMMI \textit{Capability Maturity Model Integration}


Il \textbf{COBIT} è uno standard diverso.

\section{COBIT \& COSO}

\textbf{COSO} è un modello strategico per la governance dell'azienda. Purtroppo 
è troppo complesso per un'azienda di dimensioni medio grandi e quindi è caduto 
in disuso.\\
\newline
\textbf{COBIT} è un insieme di raccomandazioni finalizzate alla \textit{governance} 
del sistema IT a livello operativo.
COBIT deriva da COSO ed è allineato con esso. In particolare il grafico in Figura
\ref{fig:cobit:coso:relazione} mostra la relazione tra COSO e COBIT\footnote{Esistono manuali gratuiti 
riguardo COBIT, che spaziano dal livello \textit{entry} fino a quelli più 
avanzati. Questi manuali sono consigliati a chi vuole approcciarsi a questo ambiente 
lavorativo. Al seguente link per esempio è possibile trovare un manuale riguardo 
COBIT4:  \url{https://www.isaca.org/Italian/Documents/CobiT-41-Italian.pdf}}.
\begin{figure}[h!]
        \begin{center}
                \includegraphics[scale=2.0]{res/img/cobit_coso_cube}
        \end{center}
        \caption{Grafico che mostra la relazione tra COBIT e COSO.}
        \label{fig:cobit:coso:relazione}
\end{figure}

\subsection{COBIT}
\label{COBIT}

\subsubsection{Pianificazione e organizzazione}

In un'azienda IT è importante definire un piano strategico per l'IT. Si è passati dall'
avere un insieme di applicazioni centralizzate ad architetture distribuite 
\textit{cloud}-centriche. Chi ha previsto queste innovazioni ieri, oggi sta 
guadagnando molto.

L'\textit{information architecture} si focalizza su come viene distribuita 
l'informazione, su come viene distribuito il dato e su come aggiunge 
informazione al dato.

La tendenza tecnologica è quella di andare verso un IT decentralizzato. La 
manutenzione è importante ed è quindi altrettanto importante capire che una rivisitazione 
dei costi IT non è possibile che venga eseguita a costo 0. I processi messi in piedi sono 
molti importanti. La comunicazione è fondamentale ed è un difetto di 
informatici e ingegneri. La non-comunicazione fa appassire le 
idee\footnote{Perla del professore: \textbf{una buona idea comunicata bene 
ottiene molto di più di un'idea eccellente comunicata male.}} e questo viene 
aggravato dall'inerzia che ha una grande azienda, in quanto un gran numero di 
persone si muovono lentamente tra loro.

Anche le risorse IT sono importanti: non serve più solo prendere ``quello 
bravo'', ma è importante avere un piano di \textit{retention}, ovvero capire 
come mantenere i dipendenti (soprattutto le persone chiave) all'interno 
dell'azienda. Questo andrebbe fatto anche nelle aziende piccole (20-30 
dipendenti)\footnote{Il professore consiglia di andare a vedere il PMI: 
\textit{Project Management Institute}, che spiega come gestire vari progetti.}.

\begin{itemize}
\item PO1: definire un piano IT strategico;
\item PO2: definire l'architettura dell'informazione;
\item PO3: determinare la direzione tecnologica;
\item PO4: definire i processi IT, l'organizzazione e le relazioni;
\item PO5: gestire gli investimenti nell'IT;
\item PO6: comunicare al management direzione e obiettivi;
\item PO7: gestire le risorse umane dell'IT;
\item PO8: gestire la qualità;
\item PO9: valutare e gestire i rischi IT; 
\item PO10: gestire i progetti.

\end{itemize}

\subsubsection{Acquisizione e implementazione}

\begin{itemize}
\item AI1: identificare soluzioni automatizzate;
\item AI2: acquisire e mantenere le applicazioni software;
\item AI3: acquisire e mantenere l'infrastruttura tecnologica;
\item AI4: abilitare operazioni e utilizzo;
\item AI5: procurare le risorse IT;
\item AI6: gestire i cambiamenti;
\item AI7: installare e accreditare soluzioni e cambiamenti.
\end{itemize}

\subsubsection{Consegna e supporto}

Quando si hanno servizi di diverse parti gli SLI\todo{cosa sono?} sono fondamentali.
Gestione e capacità sono concetti diversi. Quando si ha settato 
un'organizzazione il limite estremo è la capacità ottimale (è quando tutti 
lavorano al 100\%, che è impossibile). Da qui bisogna capire dove si è e quanto 
ce n'è, e questo viene detto \textit{capacity}. Siccome è molto difficile per 
aziende grosse si guardano gli indicatori finanziari.
