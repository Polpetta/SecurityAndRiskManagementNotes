\subsubsection{Fase 4: Exploit \& mantenimento dell'accesso}

È quando l'attaccante prendere il controllo del computer della vittima, o lo 
rende parte di una \textit{botnet}\footnote{Rete di computer controllati senza 
autorizzazione tipicamente da remoto da un'entità maliziosa}.

Troiano vi da un servizio (per questo motivo lo si installa) ma 
contemporaneamente fa qualcosa di malizioso in background.


Una backdoor è una porta di accesso non nota. Di solito i sistemi sono 
progettati per Windows ma ultimamente anche per i sistemi Unix. Questo da 
l'accesso a una \textit{shell} remota, che permette di fatto l'accesso al 
computer.


Gli \textit{spyware} mirano al furto di informazioni dell'utente, per esempio 
come i \textit{keyloggers}\footnote{Vengono presi tutti gli input da tastiera}. 
Gli \textit{adware} invece puntato a far visualizzare pubblicità, spesso non 
voluta, all'utente finale.

\subsection{Botnotes}

In una architettura top-down, si ha il \textit{bot-master} che è colui che 
comanda i \textit{bot-managers} che mandano i comandi finali all'insieme finale 
di computer infettati, definiti come \textit{zombie army}.

Che cosa ci si fa con gli \textit{zombie}? Le \textit{botnet} vengono creati sia 
per scopi governativi o per più importante, soldi. I servizi che vengono 
``venduti'' solitamente sono:
\begin{itemize}
\item Spam (tipicamente per pubblicizzare medicinali anche contraffatti) 
\item DoS. Gli attacchi ora si sono evoluti a un nuovo livello. \textit{Devices} 
come \textit{IoT} sono stati usati nel Novembre del 2016 per creare uno dei più 
grossi attacchi DoS.
\end{itemize}


%esercizi

An attack where multiple computers send connection packets to a server 
simultaneously to slow the firewall (ddos) \todo{Finire esercizio!}

% altro

A man in the middle attack is implementi which additional type of attack:
\begin{itemize}

\item Spoofing (risposta esatta)
\item DoS
\item Phishing
\item Pharming

\end{itemize}



\section{Network Encryption}

\subsection{Symmetric Cryptography}\todo{Sarebbe carino aggiungere le immagini 
mancanti}

\subsubsection{Componenti crittografiche}

\begin{itemize}
\item Sender
\item Receiver
\item Encryption
\item Decryption
\item Cypher text
\end{itemize}

\subsection{Categorie di crittografia}

Chiave simmetrica (chiave segreta) e chiave asimmetrica (chiave pubblica).

\subsubsection{Crittografia asimmetrica}

Servono due chiavi, una per cifrare e una per decifrare. Per cifrare il mittente 
usa la chiave pubblica del destinatario. Il destinatario usa la sua chiave 
privata per decifrare.

\todo{aggiungere gli schemini}

Le chiavi asimmetriche pubbliche e private hanno una relazione di tipo 
matematico. Il fatto è che dalla chiave pubblica dal punto di vista matematico 
non siamo in grado di recuperare la chiave privata.

\subsubsection{Chiave simmetrica}

Per cifrare e decrifrare il messaggio occorre avere la stessa chiave.

\paragraph{Cifrari tradizionali} Esistono diversi tipi di cifrari, che possono 
essere:
\begin{itemize}
\item A sostituzione
\begin{itemize}
 \item Monoalfabetico - Quando si ha come riferimento un solo alfabeto
 Vedi ad esempio cifrario di Cesare, perchè in corrispondenza di una doppia c'è 
la stessa lettera trasposta. Ovvero le occorrenze vengono rispettate
 \item Polialfabetico - Quando si hanno come riferimenti multipli alfabeti
 In una corrispondenza di una doppia occorrenza si ottengono due caratteri 
diversi
\end{itemize}
\item A trasposizione. Questo tipo di cifrario esegue una trasposizione del 
testo a disposizione, dove la chiave sarà il vero segreto. La funzione di 
trasposizione è sempre bigettiva.
\end{itemize}



\subparagraph{Monoalfabetico}

Usa un unico alfabeto tramite trasposizione e sostituzione (es. cifrario di 
Cesare). Si capisce che è monoalfabetico dal fatto che le doppie mappano sulle 
stesse lettere.

Al contrario, se le doppie non mappano sulla stessa lettera posso dire che il 
cifrario è non monoalfabetico (che non vuol dire polialfabetico).

\subparagraph{Cifrario a trasposizione}

Fa una permutazione del testo. La funzione di permutazione deve essere 
bigettiva.

\subparagraph{Cifrari XOR}

È una delle soluzioni migliori, e funziona molto bene. Dato un input, si esegue 
uno XOR esclusivo con una chiave $K$, ottenenedo il \textit{cyphertext}. Per 
decriptare il testo la soluzione è prendere il testo cifrato e metterlo di nuovo 
in XOR esclusivo con la chiave.

Crittaggio
$$
m \oplus k = c
$$
\indent Decrittaggio
$$
m \oplus k = (m \oplus k) \oplus k = m \oplus (k \oplus k) = m \oplus 0 = m
$$

Questo cifrario viene definito come ``cifrario perfetto'', ovvero la soluzione è 
sicura dal punto di vista dell'\textit{information theory}. La sicurezza a cui 
siamo legati oggi è basata su problemi computazionali, in cui l'attaccante si 
considera battutto se il numero di operazioni da eseguire è $\ge 2^{80}$.

%% Nonostante le sue caratteristiche non viene utilizzato perchè devo avere 
sempre chiavi nuove. Altrimenti se intercetto due messaggi codificati con la 
stessa chiave:
%% $$
%% c_0 \oplus c_1 = m_0 \oplus m_1
%% $$

\subsubparagraph{One-time pad (OTP)} In questo caso, anche se l'attaccante ha 
una capacità di calcolo infinita non sarà in grado di recuperare il 
\textit{plain text}, e tutti questi testi sarebbero equiprobabili. La chiave, 
tuttavia, è utilizzabile solamente una volta ed dev'essere completamente 
casuale.
Un problema però è la frequenza delle lettere che si susseguono in un messaggio, 
e quindi un attaccante ha la possibilità di eseguire un'analisi statistica del 
testo.

\subparagraph{Cifrario a rotazione}

\todo{Copiare l'immagine}

\subparagraph{S-boxes}

\todo{Copiare l'immagine}

\subparagraph{P-boxes}

\todo{Copiare l'immagine}

\subparagraph{DES}

\todo{Copiare l'immagine}

La grandezza della chiave dichiarata è 64 bit, ma in verità non è 64 bit, ma è 
minore (56 bit, eseguita in maniera deterministica), a causa dell'NSA che voleva 
leggere le comunicazioni tra gli utenti.

\subparagraph{Triple DES}

Concatenazione del DES in cascata.

\subparagraph{AES}

Utilizza una chiave a 128 bit in input. La prima operazione che viene eseguita è 
una sostituzione di byte, per poi eseguire una trasposizione e poi viene 
eseguito uno XOR con la sotto chiave d'ingresso.

\todo{Copiare spiegazione da wikipedia}

AES ha tre differenti configurazioni in base alle chiavi e a... \todo{completare 
la frase}

\subsubparagraph{Struttura di ogni round}

La struttura di ogni round di AES è:
\begin{itemize}
 \item sostiturazione
 \item permutazione
 \item operazione complessa
 \item \todo{completare}
\end{itemize}

\paragraph{Modalità di operazione per i cifrari a blocchi}

\todo{copia schema}

\subparagraph{ECB mode}

\todo{Spiegazione da wiki + schema}

L'ECB è tuttiavia deprecato in quanto \todo{Eh non si è capito alla fine!!}

Non funziona molto bene per le immagini.

\subparagraph{CBC mode}

L'idea di questo algoritmo è quello di concatenare i vari blocchi tra loro, 
stabilendo quindi una dipendenza tra i vari blocchi. Viene eseguito uno XOR tra 
il \textit{plain text} da cifrare e il blocco precedente del testo già cifrato. 
In questa maniera ricorsiva la cifratura di un blocco dipende dalla cifratura di 
tutti i blocchi precedenti. Per il blocco iniziale viene utilizzato un 
\textit{inizialization vector}, che è un vettore casuale che serve per generare 
la necessaria entropia anche sul blocco iniziale.

%%Gli altri metodi non li abbiamo fatti!

\subsubsection{Conclusioni}

La chiave simmetrica è molto sicura contro i computer quantistici, è molto 
efficiente.

Svantaggi: tutti questi algoritmi non hanno una prova matematica della loro 
sicurezza (ad eccezione dell'OTP). Un altro svantaggio è la gestione delle 
chiavi.

Per ora l'unico motivo per cui possiamo dire che questi algoritmi sono sicuri è 
il fatto che non siamo ancora riusciti a romperli.

\subsection{Asymmetric Cryptography}

