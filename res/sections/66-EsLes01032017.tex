\chapter{Frodi}
\label{EsFrodi}

Teoria disponibile al Capitolo \ref{Frodi}

\begin{Exercise} [
  title={Frodi di un amministratore di sistema},
  label={fr1}
 ]

 \Question Che tipo di frodi potrebbe commettere un amministratore di sistema?

\end{Exercise}

\begin{Answer} [
   ref={fr1},
   number={1}
 ]

  \Question Queste sono le possibili frodi che un amministratore di sistema
potrebbe commettere:
\begin{itemize}
  \item Rubare dati agli utenti \\
  Per risolvere questo problema l'amministratore non dovrebbe avere accesso
alle password o avere la possibilità di generarne una nuova.
  \begin{itemize}
   \item Utilizzo di Multi-Factor-Authenticator.
   \item Controllo a campione
  \end{itemize}

  \item Eseguire favoritismi per un certo gruppo di
persone\footnote{Detto \textit{Snooping}}.
  Per risolvere questo problema si dovrebbe applicare la separazione dei
doveri, ovvero: l'amministratore può creare nuovi account, ma solo l'ufficio
vendite può attivare l'account.
\end{itemize}

\end{Answer}

% Altro esercizio

\begin{Exercise} [
  title={Frode su obiettivi},
  label={fr2}
 ]

 Dunlap di Sunbeam ha un'alta aspettativa che l'impiegato necessita di
raggiungere altrimenti rischia di essere licenziato. Per raggiungere i suoi
alti standards, è necessario giocare il gioco, e la frode è stata accettata.
  \Question Come risolvere?

\end{Exercise}

\begin{Answer} [
  ref={fr2},
  number={2}
 ]

 \Question Anche qui la \textit{separation of duties} è la soluzione al
problema, e quindi serve un'altra entità che certifica che le vendite sono
state eseguite effettivamente.

\end{Answer}

% Altro es

\begin{Exercise} [
  title={Caso di Lepping},
  label={fr3}
 ]

 Una manager prende soldi da un account, e quando il pagamento è in scadenza,
paga attraverso un altro conto, continuando ad eseguire questa procedura.
Questo \textbf{lapping} andò avanti per anni e fu finalmente scoperto quando uno
stato di malattia prolungato la costrinse a stare a casa.

  \Question Come evitare il problema?
\end{Exercise}

\begin{Answer} [
  ref={fr3},
  number={3}
 ]

 \Question Per risolvere questo tipo di problema è necessario introdurre un
\textbf{periodo di ferie obbligatorio}, che permette di eseguire controlli
su un certo dipendente da parte di un altro.

\end{Answer}

