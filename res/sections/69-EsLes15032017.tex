\chapter{Information security}
\label{EsInfoSec1}
Teoria disponibile al Capitolo \ref{cap:infoSec}.
% Altro esercizio

\begin{Exercise} [
  title={Gestione dei dati},
  label={infoSec1}
 ]
 
 \Question La persona responsabile per decidere chi ha accesso ai dati è:
 \begin{enumerate}
   \item Data custodian
   \item Data owner
   \item Amministratore della sicurezza
   \item Manager della sicurezza
 \end{enumerate}
\end{Exercise}


\begin{Answer} [
  ref={infoSec1},
  number={1}
  ]
  
  \Question Risposta esatta: 2
  
\end{Answer}

% Altro esercizio

\begin{Exercise} [
  title={Gestione dei dati},
  label={infoSec2}
  ]
  
  \Question Il \textit{Least Privilege} dice che:
  \begin{enumerate}
    \item Le persone dovrebbero avere l'abilità di eseguire compiti per fare i 
    loro task primari e basta
    \item I diritti d'accesso e di permessi dovrebbero essere commisurate con 
    la posizione della persona nella organizzazione: ad esempio, gli avvocati 
    minori avranno meno permessi
    \item Gli utenti non dovrebbero mai avere la password di amministratore del 
    proprio computer
    \item Le persone dovrebbero avere i permessi d'accesso solo per i loro 
    livelli di sicurezza: \textit{Confidential, Private, Sensitive}
  \end{enumerate}
\end{Exercise}


\begin{Answer} [
  ref={infoSec2},
  number={2}
  ]
  
  \Question Risposta esatta: 1
  
\end{Answer}

% Altro esercizio

\begin{Exercise} [
  title={Gestione delle informazioni personali},
  label={infoSec3}
  ]
  
  \Question Riguardo alle informazioni personali o private è sempre importante 
  tenere conto che:
  \begin{enumerate}
    \item I dati non vengano tenuti di più di quanto necessario
    \item La criptazione dei dati renda la detenzione di dati personali sicura
    \item Le informazioni private su disco non debbano mai essere portate fuori 
    dal sito
    \item I dati personali siano sempre etichettati e gestiti come critici o 
    vitali da parte dell'organizzazione
  \end{enumerate}
\end{Exercise}


\begin{Answer} [
  ref={infoSec3},
  number={3}
  ]
  
  \Question Risposta esatta: 1
  
\end{Answer}
