\label{EsFrodi2}

\begin{Exercise} [
  title={Prevenzione di una frode},
  label={fr4}
 ]

 \Question Quali sono i metodi migliori per prevenire le frodi?
\begin{enumerate}
  \item Controlli interni efficaci
  \item \textit{Fraud training program}
  \item \textit{Fraud hotline}
  \item Infliggere punizione quando una frode viene scoperta.
\end{enumerate}

\end{Exercise}

\begin{Answer} [
   ref={fr4},
   number={4}
 ]

  \Question Risposta esatta: 1

\end{Answer}

% Altro esercizio

\begin{Exercise} [
  title={Prevenzione di una frode},
  label={fr5}
 ]

 \Question Una donna in un reparto di contabilità ha creato un fornitore con le
sue iniziali, ed è stata in grado di rubare più di \$4 milioni dopo 3 anni.
L'\textit{auditor} dovrebbe essere stato in grado di trovare che:
\begin{enumerate}
 \item Il fornitore era una compagnia fasulla
 \item Gli acquisti da quel fornitore non risultavano in merce ricevuta
 \item Le iniziali del fornitore combaciavano con l'impiegata al dipartimento
di contabilità
 \item L'amministrazione non aveva autorizzato nuovi fornitori tramite una
chiamata separata
\end{enumerate}

\end{Exercise}

\begin{Answer} [
  ref={fr5},
  number={5}
 ]

 \Question Risposta esatta: 4
\end{Answer}




% Altro esercizio

\begin{Exercise} [
  title={Definizioni},
  label={fr6}
 ]

 \Question Che cosa sono: \textit{Origine, Autenticazione, Distribuzione e Verifica}?

 \begin{enumerate}
  \item Quattro passi per il rilascio di un software
  \item Autorità raccomandate per gestire l'\textit{access control}
  \item Sequenze di sviluppo di un gestore di identità biometrica (BIMS)
  \item Categorie della \textit{Segregation of Duties}
 \end{enumerate}

\end{Exercise}

\begin{Answer} [
  ref={fr6},
  number={6}
 ]

 \Question Risposta esatta: 4
\end{Answer}


% Altri esercizi

\subsection{Domande riassuntive}
\label{EsFrodi3}

\begin{Exercise} [
  title={Domande riassuntive},
  label={fr7}
 ]

 \Question Quali sono gli elementi chiave di una frode, e quali tecniche
possono essere utilizzate per contrastare questi elementi chiave?

 \Question Quali sono le tre categorie della frode?

 \Question Quali sono le considerazioni legali della frode?

 \Question Chi commette la frode? E chi commette le frodi più ``costose''?

 \Question Quali sono i \textit{red flags} di una potenziale frode?

 \Question Come accadono gli attacchi basati sul \textit{social engineering} e
come possono essere prevenuti?

 \Question Applica il concetto di \textit{segregation of duties}. %?

\end{Exercise}

\begin{Answer} [
  ref={fr7},
  number={7}
 ]

 \Question Gli elementi chiavi di una frode sono tre, e sono:
\begin{itemize}
 \item Motivazione
 \item Opportunità
 \item Giustificazione
\end{itemize}
Una delle prime cose da fare è non sottovalutare la possibile frustazione dei
lavoratori: sia di quelli con un alto grado di studio ma messi in lavori di
basso livello, sia nel caso in cui un lavoratore per troppo tempo non riceva
una promozione. Attenzione deve essere prestata a come la dirigenza si
comporta: infatti i dipendenti a livello più basso la prenderà come esempio.
Oltre a ciò rispettare la segregation of duties, sicurezza dei beni e dei
luoghi, restringendo l'accesso delle aree più critiche e fare background check
dei propri lavoratori, scoprendo in questo modo finti titoli di studio o che
dei dipendenti hanno già compiuto delle frodi.

 \Question Le categorie di una frode sono tre:
 \begin{itemize}
  \item Appropriazione indebita
  \item Corruzione
  \item Dichiarazione illecita
 \end{itemize}

 \Question 
Dal punto di vista legale una frode consiste nel fornire false informazioni a
qualcuno per fare soldi. Questa azione è intenzionale (dolosa) e causa un danno
materiale ad una vittima. 

Provare che una frode è avvenuta è molto semplice:
infatti molto spesso è evidente se qualcosa è falso, materiale e se c'è una
perdita per la vittima. Invece provare il dolo è molto complicato. Infatti, la
maggior parte delle volte la prima difesa in caso di frodi è asserire che si ha
commesso un errore, il che alcune volte corrisponde a realtà. Per questo gli
investigatori devono cercare prove e indizi a supporto dell'accusa. Alcuni
esempi lampanti sono la manipolazione di documenti o la distruzione di prove,
anche se, alcune volte, queste azioni vengono commesse per nascondere il fatto
che si ha sbagliato, per paura delle conseguenze. L'intralciare le indagini e
mentire agli investigatori sono altri campanelli d'allarme che fanno aumentare
i sospetti. Inoltre comportamenti non etici in passato mostrano qualcosa riguardo alle
tendenze dell'accusato. Oltre a ciò è necessario dimostrare che l'accusato ha avuto
dei benefici dall'avvento della frode. Nessuno di questi fattori da solo può
provare l'intenzionalità oltre ogni ragionevole dubbio. In assenza di una
confessione dell'accusato, l'intento può essere stabilito attraverso una lista
di campanelli d'allarme.

 \Question Le frodi solitamente sono commesse da personale aziendale di alto 
rango (come ad esempio dirigenti). Le frodi pi\`u ``costose'' sono solitamente 
eseguite dagli uomini. Questo dato \`e originato dal fatto che la componente 
maschile \`e maggiormente presente negli alti ranghi delle aziende.

 \Question I \textit{red flag} di una frode solitamente sono:
 \begin{itemize}
  \item Difficoltà finanziare
  \item Problemi legali
  \item Se ha già commesso questo atto in passato (eseguire un
  \textit{background check})
  \item Ego: voglia di ``rompere'' le regole (ad esempio ``bucare'' un
  firewall, entrare in sistemi non autorizzati, ecc..)
  \item Problemi in famiglia (ad esempio divorzio)
\end{itemize}

 \Question Gli attacchi basati su il \textit{social engineering} si basano 
sulla fiducia che l'attaccante insinua nella vittima. Ad un certo momento, 
l'attaccante usa questa fiducia ottenuta per eseguire la truffa.
Per contrastare questo tipo di attacchi possono essere messe in atto 
le seguenti \textbf{misure preventive}:
\begin{itemize}
  \item Verificare sempre che l'indirizzo e-mail sia dell'azienda
  \item Verificare che il richiedente sia un impiegato e che ricopra
  effettivamente quel ruolo
  \item Verificare l'autorizzazione richiesta
  \item Verificare i record delle transazioni
\end{itemize}

Anche a livello organizzativo è necessario effettuare delle verifiche, che
potrebbero essere:
\begin{itemize}
  \item Classificazione dei dati e definizione del loro trattamento
  \item \textit{Policies} definite per il comportamento degli impiegati
  \item Addestramento dei dipendenti riguardo ai ruoli e alle \textit{policies}
che devono essere applicate
\end{itemize}

% Applica il concetto di \textit{segregation of duties}.

 \Question L'applicazione della \textit{segregation of duties} consiste nel 
fare in modo che il potere associato ad una certa funzione non risieda nelle 
mani di una sola persona. Per una corretta applicazione della 
\textit{segregation of duties} sarebbero necessari questi tre elementi:
\begin{itemize}
 \item Origine
 \item Autorizzazione
 \item Verifica
\end{itemize}

Il numero minimo per una corretta applicazione della \textit{segregation of 
duties} tuttavia pu\`o essere di due persone.

\end{Answer}

