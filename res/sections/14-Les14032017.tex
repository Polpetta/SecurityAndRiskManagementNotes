\subsection{Business Continuity Process}

Un piano per la continuazione del business si suddivide nei seguenti passi:
\begin{enumerate}
  \item Attuazione dell'\textit{Business Impact Analysis}
  \item Privatizzazione dei servizi di supporto critici per i servizi di 
  business
  \item Determinazione del processo della modalità alternativa per i servizi 
  critici e vitali
  \item Sviluppare il piano di \textit{Disaster Recovery} per i servizi di 
  recupero IS
  \item Sviluppo di tipo BCP per le operazioni di recupero e continuazione del 
  business
  \item Test dei piani
  \item Mantenimento dei piani
\end{enumerate}

\subsection{Esercizi}

The amount of data transaction that are allowd to be lost following a computer 
failure (i.e., duration of orphan data) is the:
\begin{itemize}
  \item Recovery Time Objective
  \item Recovery Point Objective (risposta esatta)
  \item Service Delivery Objective
  \item Maximum Tolerable Outage
\end{itemize}

% Altro esercizio

When the RTO is large, this is associated with
\begin{itemize}
  \item Critical applications
  \item A sppedy alternate recovery strategy
  \item Sensative or nonsensitive services
  \item An extensive restoration plan (risposta esatta)
\end{itemize}

% Altro esercizio

Whrn the RPO is very short, the best solution is:
\begin{itemize}
  \item Cold site
  \item Data mirroring
  \item A detailed andd efficient Disaster Recovery Plan
  \item An accurate Business Continuity Plan
\end{itemize}

\section{Disaster recovery testing}

Un incidente non viene dichiarato immediatamente, ma c'è una procedura di 
chiamate prima di una dichiarazione di incidente. Quando una situazione di 
incidente è dichiarata l'incolumità fisica è la prima a essere presa in 
considerazione. Gli \textit{Stackeholders} sono subito contattati, mentre 
l'ufficio stampa si occupa di dichiarare l'incidente. Questa parte è molto 
delicata, è il danno dev'essere spiegato in maniera molto delicata.

L'ufficio IT entra in azione dopo tutte queste procedure, e comincia ad agire 
per risolvere il problema.

\subsection{Preparazione di un piano di \textit{piano di recupero}}

La vita delle persone viene messa sempre in primo piano. Viene poi definito chi 
deve fare cosa. Le copie del piano di recupero devono essere distribuite in 
diverse locazioni per evitare che essa possa esser distrutta.

\subsubsection{Responsabilità di un piano di recupero}

In ogni piano della struttura ad esempio dovrebbe necessario che una persona 
che si incarichi di controllare ogni stanza del piano in caso di una 
evacuazione.

\subsection{Documenti di un piano di recupero}

% TODO: copiare la tabella

\subsection{\textit{Mean time between failure}}

Detto anche MTBF è la somma del \textit{Mean Time to Repair} (MTTR) più il 
\textit{Mean Time To Fail} (MTTF).

% TODO: inserire schema da slide

L'affidabilità può essere misurata. Un tipica misura è quella detta del 
\textit{5 9s}, ovvero del $99.999\%$, ovvero 5 minuti e mezzo di fallimento 
all'anno.

\subsection{Esecuzione del test}

È l'esecuzione del piano per verificarne il suo funzionamento.
Il test viene effettuato in questo ordine:
\begin{itemize}
  \item Desk-Based Evaluation/Paper test \\
  Un gruppo comincia a seguire i passi della procedura a mano e li esegue a mano
  \item Preparedness Test \\
  Una parte dell'intero test viene eseguito. Parte differenti del test vengono 
  eseguite regolarmente.
  \item Full Operational Test \\
  Simulazione di un disastro completo
\end{itemize}

\subsubsection{Tipologie di test per il Business Continuity}

Esistono diverse tipologie di test per il \textit{Business Continuity}:
\begin{itemize}
  \item Checklist Review \\
  Rivisitazione per piano di copertura. Tutti i piani importanti sono coperti?
  \item Structured Walkthrough \\
  Rivisitazione di tutti gli aspetti del piano, spesso guardando tutti gli 
  scenari
  \item Simulation Test \\
  Esecuzione del piano basato specificamente per uno scenario, senza un sito 
  alternativo
  \item Parallel Test \\
  Viene azionato la \textit{facility} alternativa fuori dal sito, senza 
  interrompere il sito regolare
  \item Full-Interruption \\
  Si passa alla modalità alternativa da quella regolare, per verificare che 
  tutto sia funzionante.
\end{itemize}

\subsubsection{Obiettivi del testing}

Il testing del piano deve risultare in un recupero con successo delle 
infrastrutture e dei processi \textit{business}

\paragraph*{Vantaggi} Ci sono i seguenti vantaggi nell'eseguire il testing:
\begin{itemize}
  \item Identificazione di probabili errori
  \item Verifica delle assunzioni fatte
  \item Verifica dei tempi
  \item Allenare e coordinare lo staff
\end{itemize}

\subsubsection{Procedure del testing}

% TODO: inserire schema presenti sulle slide

I test sono inizialmente semplice e poi diventano man mano sempre più complesso 
con il tempo.

Include di solito una terza parte indipendente, per poter permetterne una 
migliore osservazione.

Dalla valutazione del test si ottengono quelli che sono i \textbf{punti di 
miglioramento}, che devono essere effettivamente messi in pratica.

\paragraph*{Passi del test} Esistono diversi passi del test. C'è la sua 
preparazione, l'esecuzione del test e la pulizia dopo-test. Importante è 
ricordarsi di eseguire una pulizia dei dati dopo l'esecuzione del test, ma di 
quelli che non sono importanti per la valutazione stessa del test (altrimenti 
il test sarebbe inutile).

\subsubsection{\textit{Gap Analysis}}

Una \textit{gap analysis} è relativa a un obiettivo che un'azienda si pone, ed 
è necessaria affinché una azienda possa migliorare se stessa. Quelli che sono 
più controllati sono i processi, perché le persone sono quelle che sono più 
difficili da ``modificare''.

\subsection{Auditing di un BCP}

Sono presenti diversi step da controllare, che possono essere:
\begin{itemize}
  \item Include la BIA?
  \item È il BCP in linea con i goals del business, sono effettivi o moderni?
  \item È chiaro chi dovrà svolgere cosa?
  \item Sono tutti competenti? Accettano volentieri questo ruolo?
  \item DRP è mantenuto e dettagliato?
  \item Le procedure di backup e di recupero sono seguite?
\end{itemize}

% Manca parte sull'assicurazione

\section{Riassunto dei controlli principali per i controlli di sicurezza BC}

\begin{itemize}
  \item RAID
  \item Diversi tipi di backup
  \item Diversi tipi di protezione sulla rete
  \item Siti alternativi in caso di problemi
  \item Applicazione di test
  \item Assicurazione
\end{itemize}

\section{Esercizi}

The FIRST thing that should be done when you discover an intruder has hacked 
into your computer system is to:
\begin{itemize}
  \item Disconnect the computer facilities from the computer network to 
  hopefully disconnect the attacker
  \item Power down the server to prevent turther loss of confidentiality and 
  data itegrity
  \item Call the manager
  \item Follow the directions of the Incident Response Plan (risposta esatta)
\end{itemize}

% Altro esercizio

During an audia of the business continuity plan, the finding of MOST convern is:
\begin{itemize}
  \item The phone tree has not been double-checked in 6 months
  \item The business impact analysis has not been updated this year
  \item A test of the  backup-recovery system is not performed regularly 
  (risposta esatta)
  \item The backup library site lacks a UPS
\end{itemize}