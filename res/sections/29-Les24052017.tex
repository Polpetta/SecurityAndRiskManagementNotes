\section{Race condition}

Questo attacco può essere molto pericoloso in alcune situazioni, come nel caso
di computazioni eseguite su architettura \textit{CUDA}, in cui era
possibile estrarre informazioni sensibili durante la computazione. In ambiente
\textit{cloud} è importante assicurarsi che ci sia un \textit{hypervisor} che
coordini tutte le computazioni.

Una situazione del genere può accadere in società di esplorazione geologica (eni
ad esempio) o in situazioni di sequenziazione genomica/proteica.

\section{Messaggi d'errore troppo verbosi}

A volte i messaggi di errore sono troppi verbosi e possono far inferire delle
informazioni. Un messaggio di errore troppo verboso è solitamente il form di
login: in alcuni servizi sono presenti messaggi come ``password scorretta'', che
permette all'attaccante di sapere che esiste un login con quell'indirizzo
e-mail.

\section{External Control of Path}

Normalmente quando si ha un sito web bisogna settare i diritti. Se dall'esterno
si può lanciare un file (es. file Word) si fornisce una backdoor per un
potenziale attaccante.

\section{Uso di software non fidato}

Il software a volte installato dall'utente non è fidato, (es. uno
\textit{spyware} che si occupa di raccogliere informazioni a scopo pubblicitario
o di profilazione).

Il software free è sicuro? Sì, ma solo se è \textit{open-source} e ha un po' di
``anzianità'' (viene utilizzato da un po' di tempo).

\section{Problemi classici di sicurezza}

\begin{verbatim}
Security() {
 String contents, environment;
 String spath = "security.dat"
 File security = new File();
 if (security.open(spath) > 0) {
  contents = security.read();
  environment = security.read();
 } else
  print("Error: Security.dat not found");
}
\end{verbatim}

Questa porzione di codice contiene diversi errori:
\begin{itemize}
\item Errori troppo verbosi;
\item Il file viene salvato in chiaro;
\item Non viene chiuso il file dopo la lettura;
\item Non viene controllata la lunghezza della lettura del file.
\end{itemize}

Un altro errore è:
\begin{verbatim}
purchaseProduct() {
 password ="N23m**2d3";
 count = form.quantity; //input
 total = count * product.cost();
 Message m = new Message(name,password,product,total);
 m.myEncrypt();
 server.send(m);
}
\end{verbatim}

Ci sono errori gravi in questo pezzo di codice:
\begin{itemize}
\item Le password vengono salvate sul codice (detto anche \textit{hardcoded}):
\item Ci potrebbe essere un \textit{overflow} (es. uno \textit{shift-bit});
\item Viene usata una routine locale per eseguire la criptazione dei dati,
mentre si dovrebbe sempre utilizzare algoritmi standard.
\end{itemize}


\paragraph*{Come evitare gli errori} Un modo per evitare questi errori è
testare adeguatamente il software che viene scritto. Purtroppo non è possibile
testare il programma con tutti gli input possibili, quindi è necessario
controllare il software con una \textit{test-suite} adeguata.
Anche il \textit{model-checking} è importante per analizzare il software.

\subsection{Esercizi}

La parte di esercizi riguardo questa sezione è disponibile in \ref{es:Controlli}.







