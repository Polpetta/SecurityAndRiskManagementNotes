
\section{Race condition}

Questo attacco può essere molto pericoloso in alcune situazioni, come nel caso 
per esempio di computazioni eseguite su architettura \textit{CUDA}, in cui era 
possibile estrarre informazioni sensibili durante la computazione. In ambiente 
\textit{cloud} è importante assicurarsi che ci sia un \textit{hypervisor} che 
coordina tutte le computazioni.

Una situazione del genere può accadere in società di esplorazione geologica (eni 
ad esempio) o in situazioni di sequenziazione genomica/proteica.

\section{Messaggi d'errore troppo verbosi}

A volte i messaggi di errore sono troppi verbosi, e possono far inferire delle 
informazioni. Un messaggio di errore troppo verboso è solitamente il form di 
login: in alcuni servizi sono presenti messaggi come ``password scorretta'', che 
permette all'attaccante di sapere che esiste un login con quell'indirizzo 
e-mail.

\section{External Control of Path}
Normalmente quando si ha un sito web bisogna settare i diritti. Se dall'esterno 
si può lanciare un file (per esempio un file word) si da una backdoor ad un 
potenziale attaccante.\todo{Sistema}



\section{Uso di software non fidato}

Il software a volte installato dall'utente non è fidato, che può essere uno 
\textit{spyware} che si occupa di raccogliere informazioni a scopo pubblicitario 
o di profilazione.

Il software free è sicuro? Sì, ma solo se è \textit{open-source} e ha un po' di 
``anzianità'' (viene utilizzato da un po' di tempo).

\section{Problemi classici di sicurezza}

Per esempio il codice:
\begin{verbatim}
Security() {
 String contents, environment;
 String spath = "security.dat"
 File security = new File();
 if (security.open(spath) > 0) {
  contents = security.read();
  environment = security.read();
 } else
  print("Error: Security.dat not found");
}
\end{verbatim}

Questa porzione di codice contiene diversi errori:
\begin{itemize}
\item Errori troppo verbosi
\item Il file viene salvato in chiaro
\item Non viene chiuso il file dopo la lettura
\item Non viene controllata la lunghezza della lettura del file
\end{itemize}

Un altro errore è:
\begin{verbatim}
purchaseProduct() {
 password ="
}
\end{verbatim}
Un problema è la password \emph{hard-coded} del codice.

Ci sono errori gravi in questo pezzo di codice:
\begin{itemize}
\item Le password vengono salvate sul codice (detto anche \textit{hardcoded})
\item Ci potrebbe essere un \textit{overflow}, che a volte potrebbe accadere uno 
\textit{swift-to-left} % correggete la mia scrittura REEEEEE
\item Viene usata una routine locale per eseguire la criptazione dei dati, 
mentre si dovrebbe sempre utilizzare algoritmi standard
\end{itemize}


\paragraph*{Come evitare gli errori} Un modo per evitare questi errori è 
adeguatamente testare il software che viene scritto. Purtroppo non è possibile 
testare il programma con tutti gli input possibili, ed è quindi necessario 
controllare adeguatamente il software con una \textit{test-suite} adeguata. 
Anche il \textit{model-checking} è importante per analizzare il software. % 
controlla mirko perchè sto guardando un video su youtube contemporaneamente

\subsection{Esercizi}

La parte di esercizi riguardo questa sezione è disponibile in \ref{es:Controlli}

\part{Controlli di sicurezza}

Questo modulo si concentra sui controlli di sicurezza del software. Per fare ciò 
dovremo vedere il ciclo di vita del software, con un occhio però alla sicurezza.
Partiremo dall'analisi dei requisiti, per andare all'analisi del rischio e per 
finire con lo sviluppo del software.
Anche l'ambiente produttivo richiede attenzione, è importante il mantenimento da 
parte di vista della sicurezza per non introdurre bug nel software.

\chapter{Validazione degli input}

\section{Validazione delle transazioni}

Alcuni punti importanti riguardo la validazioni delle transazioni sono:
\begin{itemize}
\item Sequence Check:
\item Limit o Range Check:
\item Validity Check or Table Lookup:
\item Reasonableness Check: 
Per esempio oggi come oggi il numero di pizze in america hanno poche grandi 
catene, e questo perchè? Perchè il sistema è diventato tutto informatizzato, ed 
è per questo che PizzaHat ha il monopolio in america, grazie alla sua 
applicazione. È per questo che il controllo sull'input è molto importante, 
perchè un possibile attacco di Denial Of Service sarebbe ordinare un numero 
incontrollato di pizze.
\item Existence Check:
\item Key verification: fare inserire due volte la mail (per esempio)
\item Check digit:
\item Completeness Check
\item Duplicate Check: è importante che il sistemi controlli un possibile doppio 
acquisto da parte di un utente, che potrebbe eseguire un acquisto per errore. 
Anche questa tipologia di controlli dovrebbero essere eseguiti.
\item Consistency or Logical Relationship Check: son difficili da eseguire in 
maniera automatizzata, e i controlli logici permettono di controllare per 
esempio se gli utenti stanno mentendo o meno. 
\end{itemize}
