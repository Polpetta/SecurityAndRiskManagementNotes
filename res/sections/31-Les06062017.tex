\subsubsection{Passo 1: Preparazione}

È importare eseguire una \textit{Business Impact Analisys}, per capire i danni
che ci possono essere al sistema.
Una buona guida può essere porsi le seguente domande:
\begin{itemize}
\item Quando interviene il team per la gestione dell'incidente?
\item Come possono aiutare le agenzie governative? Negli USA FBI e in Italia la
Polizia Postale.
\end{itemize}



\paragraph*{Tecnologie per il rilevamento}

Ogni organizzazione deve avere una le capacità di monitoraggio e di rilevamento
sufficiente per capire se ci sono incidenti in una misura di tempo ristretta.

\subparagraph*{Decoy} Esiste tutta un'area molto ricca (anche in termini
monetari) che fanno \textit{decoy}, ovvero fanno un falso obiettivo: son sistemi
in uso, account, servizi, ecc... che sono credibili da parte dell'organizzazione
ma che non sono usabili. Quando questi account, servizi, ... vengono usati
l'attacco viene rilevato. Questa metodologia è per certi versi diversa
dall'\textit{honeypot}, e si diversifica in quanto è un'operazione per eseguire
controlli, non una vulnerabilità fittizia aperta appositamente.

\subparagraph*{Vulnerabilità e incidenti} Gli incidenti e le vulnerabilità
legate registrate tramite i log si stanno rivelando fallimentari, e si sta
cominciando ad utilizzare l'intelligenza artificiale come modo per eseguire
analisi su grandi moli di dati.

\paragraph*{Management Participation}

Proporre alternative
\begin{itemize}
\item Include le criticità del business, rischio della proposta, costo e tempo
per il recupero, affidabilità
\item Costi di detection: NIDS/HIDS
\end{itemize}

Il management prende la decisione finale (il senior mngm deve essere convinto
che la soluzione vale i soldi investiti).

\paragraph*{Contenuti IRP}

\begin{itemize}
\item Preparazione preincidente
\item Come dichiarare il disastro
\item Procedure di evacuazione
\item Identificare le persone responsabili, informazioni e contattti
\end{itemize}

\subsubsection{Passo 2: Identificazione}
\subparagraph*{Triage} Questa operazione (detta anche \textbf{selezione}) si fa
anche negli ospedali (bianco, verde rosso giallo ecc.) e consiste nell'eseguire
una categorizzazione degli eventi. Possibili punti importanti:
\begin{itemize}
\item che tipo di incidente è avvenuto?
\item qual è la severità dell'incidente
\item chi dovrebbe essere chiamato?
\item stabilire una catena di custodia per le prove
\end{itemize}
È importante che esistano queste tipologie di procedure in quanto permette alla
gente di eseguire in maniera ``automatica'' anche quando ci sono situazione di
allarme/si è sotto attacco.

\subparagraph*{Catena di custodia delle prove} Un passo che non bisogna fare mai
è inquinare le tracce (anche biologiche) di un indicente. Questo perché
potrebbero essere utili per ricostruire l'incidente e anche utili per perseguire
chi ha commesso il reato.
In ogni momento l'evidenza deve seguire la catena della custodia. % u wot m8?


\subsubsection{Passo 3: Containment per conte: Identificazione}
\subparagraph*{Triage} Questa operazione (detta anche \textbf{selezione}) si fa
anche negli ospedali (bianco, verde rosso giallo ecc.) e consiste nell'eseguire
una categorizzazione degli eventi. Possibili punti importanti:
\begin{itemize}
\item che tipo di incidente è avvenuto?
\item qual è la severità dell'incidente
\item chi dovrebbe essere chiamato?
\item stabilire una catena di custodia per le prove
\end{itemize}
È importante che esistano queste tipologie di procedure in quanto permette alla
gente di eseguire in maniera ``automatica'' anche quando ci sono situazione di
allarme/si è sotto attacco.

\subparagraph*{Catena di custodia delle prove} Un passo che non bisogna fare mai
è inquinare le tracce (anche biologiche) di un indicente. Questo perché
potrebbero essere utili per ricostruire l'incidente e anche utili per perseguire
chi ha commesso il reato.
In ogni momento l'evidenza deve seguire la catena della custodia. % u wot m8?


\subsubsection{Passo 3: Containment}

Si attiva il team di risposta all'incidente per contenere il rischio, si
isola\todo{Completare la frase!}


Bisogna anche capire la corretta strategia per contenere i danni. Alcune
soluzioni potrebbero comprendere la disconnessione di rete, ma questo implica
dei costi che vanno valutati.



\paragraph*{Risposta}

Alcune valide risposte sono:
\begin{itemize}
\item Collezionare informazioni sul sistema
\item Analizzare i \textit{log} di sistema
\item Persecuzione\footnote{In termini legali.} dell'attaccante. Un problema è
che il personale che si occupa dell'aspetto legale di attacchi informatici è
poco.  % Momento di crisi del prof sui possibili sinonimi di 'persecuzione'
\end{itemize}






\subsubsection{Passo 4: Analisi e Eradicazione}

Determinare com'è avvenuto l'attacco: 5 w del giornalismo. Capire qual è stato
l'impatto e che danni ha causato. Rumuovere la radice per:

\begin{itemize}
\item Ricostruire il sistema
\item Parlare con l'ISP per ottenere più informazioni
\item Eseguire un'analisi di vulnerabilità
\item Migliorare le difese \todo{completa}
\end{itemize}


\paragraph*{Analisi}

\begin{itemize}
\item Cos'è successo?
\item Chi è stato coinvolto?
\item Qual è la ragione dell'attacco?
\item Da dove è partito l'attacco?
\item Quando è iniziato l'attacco?
\item Com'è successo?
\item Quali vulnerabilità ha permesso l'attacco?
\end{itemize}


\paragraph*{Rimuovere le radici del problema}

Alcune possibili soluzioni:
\begin{enumerate}
\item Se un account root viene compromesso, non è possibile recuperare per cui è
meglio resettare completamente il sistema formattando tutto e ripristinando i
dati dal backup, sperando che non siano stati compromessi precedentemente.
\item Implementare le patch recenti e gli antivirus, tutte le password devono
essere cambiate.
\end{enumerate}

\subsubsection{Passo 5: Recovery}

Ripristinare le normali operazioni e assicurare che il sistema sia completamente
funzionante e testato.

\subsubsection{Passo 6: Imparare la lezione}

Si trova nell'\textit{Incident Report}, che include cosa è andato storto,
quando, quanto è costato, ecc. Il piano deve essere presentato agli
stakeholders.


\section{Pianificazione dei processi}
\label{IRBC:pp}

\subsubsection{Training}

Nel momento in cui subiamo il training diventiamo responsabili. Si distingue tra
colpa (accade sotto la nostra responsabilità) e colpa grave (accade sotto la
nostra responsabilità e siamo tecnicamente preparati per evitare il problema, ma
non l'abbiamo fatto).

\paragraph*{Mentoring} Tipicamente c'è un \textbf{mentor}, ossia qualcuno con
una lunga esperienza che istruisce

\paragraph*{Training formale}

\paragraph*{On-the-job-training} Qualcuno che ti istruisce mentre lavori.


\paragraph*{Penetration Tests} Esistono diverse modalità per eseguire dei
penetration tests:
\begin{itemize}
\item Blind: i penetration testers non sanno nulla riguardo a possibili
\textit{decoy} o possibili \textit{honeypot}. Gli amministratori sono a
conoscenza del test
\item Double blind: come blind, ma anche gli amministratori di sistema non sanno
nulla riguardo l'attacco\footnote{Non è molto utile di solito}.
\end{itemize}

\subsection{Incident Management Metrics}
\begin{itemize}
\item Numero di incidenti riportati
\item Numero di incidenti rilevati
\item Tempo medio i risposta di un incidente
\item Numero totale di incidenti risolti
\item Misure prese proattive e preventve
\item Danno totale degli incidenti riportati e rilevati
\item \todo{Ultimo}
\end{itemize}

\subsubsection{Challenges} % Sono sottosezioni

Questo viene detto anche come risoluzione dei problemi ``a costo
zero''\footnote{Che per la terza legge della termodinamica non può essere mai
zero}. Questo di solito significa fallimento: alla persona vengono aggiunte
nuove responsabilità e nuovi oneri, che di solito non riesce a portare a termine
in quanto è presente un grosso \textit{overloading} sulla persona stessa.

\paragraph*{Comunicazione} La comunicazione all'interno di un \textit{team} è
importante, e non dev'essere né troppa né troppo poca.

\subsection{Esercizi}

Gli esercizi sono disponibili in \ref{esIRBC:pp}

\chapter{Computer Forensics}\todo{Copia grafico}
\label{IRBC:cf}

Raccolgono tutti gli indizi che possono essere utili per la risoluzione di un
incidente.

Le prove devono passare i test sulla:
\begin{enumerate}
\item Autenticità
\item \todo{Missing in Action}
\end{enumerate}

Bisogna tenere conto che le prove devono essere esposte solitamente ad avvocati,
e quindi a volte è necessario che il lessico usato non sia tecnico e fruibile da
tutti.

\section{Preparare una prova}

Ci sono alcune prove che non possono essere portate in tribunale, a causa di
quella che si chiama \textbf{corrispondenza privilegiata}, che sono delle e-mail
di tipo classificate che vengono controllate da un delegato speciale che
conferma o meno se hanno a che fare con il caso. È quindi importante prestare
attenzione a come le prove vengono gestite.


Il processo di identificare, analizzare e \todo{completare}

\section{Creare una copia forense}

Questo assicura che i contenuti non siano cambiati. Per eseguire questo non
vengono eseguiti tramite una copia normale, ma tramite la creazione di
un'immagine del disco\footnote{Ad esepmio con dd}.

La fase di \textbf{indexing} dei dati dev'essere eseguita tramite algoritmi
\textit{smart}, che permette una ricerca più veloce di particolari stringhe di
testo.

\section{Rapporto Legale}

Descrivere i dettagli dell'incidente accuratamente, essere comprensibile e non
ambiguo, offrire valide conclusioni, opinioni o raccomandazioni, descrivere come
le conclusioni sono state raggiunte, essere puntuale e preciso e facile da
riferire


\todo{Add title}

\begin{itemize}
\item Numero di caso

\end{itemize}

Investigation report
\begin{itemize}
\item Name and contact info for investigators
\item Numero del caso
\item Data di investigazione
\item Dettagli dell'interrogatorio o comunicazioni
\item Dettagli del device

\end{itemize}


\section{Esercizi}

Gli esercizi sono disponibili in \ref{esIRBC:cf}
