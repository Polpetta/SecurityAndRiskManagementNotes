\chapter{Audit}
\label{EsAudit}

La teoria riguardante questa serie di esercizi si trova nella
parte \ref{audit}.

\begin{Exercise} [
  title={Quiz},
  label={audit1}
  ]

  \Question Lo scopo primario di un software di audit generico (GAS) è quello 
di:
\begin{enumerate}
 \item Trovare transsazioni fraudolenti
 \item Determinare medie di esempio con medie della popolazione
 \item Estrarre informazioni per un \textit{Substantive Test}
 \item Organizzare un report di audit
\end{enumerate}

\end{Exercise}


\begin{Answer} [
  ref={audit1},
  number={1}
  ]

  \Question Risposta esatta: 3

\end{Answer}


\begin{Exercise} [
  title={Quiz},
  label={audit2}
  ]

  \Question Un controllo compensativo è definito come:
  \begin{enumerate}
   \item Due controlli forti puntano allo stesso errore
   \item Un errore è indirizzato da un controllo debole e uno forte in un'altra 
area
   \item Un controllo indirizza a un problema specifico
   \item Un controllo che aggiusta il problema dopo che è stato trovato
  \end{enumerate}

\end{Exercise}


\begin{Answer} [
  ref={audit2},
  number={2}
  ]

  \Question Risposta esatta: 2

\end{Answer}


\begin{Exercise} [
  title={Quiz},
  label={audit3}
  ]

  \Question Un \textit{IS auditor} dovrebbe pianificare il suo approccio 
all'audit basandosi su:
\begin{enumerate}
 \item Materialità
 \item Raccomandazioni del \textit{management}
 \item Raccomandazioni ISACA
 \item Rischio
\end{enumerate}
  
\end{Exercise}


\begin{Answer} [
  ref={audit3},
  number={3}
  ]

  \Question Risposta esatta: 4

\end{Answer}

\begin{Exercise} [
  title={Quiz},
  label={audit4}
  ]

  \Question Un hash totale è mantenuto per ogni file per assicurarsi che 
nessuna transazione venga persa. Questo è un esempio di un:
\begin{enumerate}
 \item \textit{Preventive Control}
 \item \textit{Detective Control}
 \item \textit{Compensating Control}
 \item \textit{Corrective Control}
\end{enumerate}

\end{Exercise}


\begin{Answer} [
  ref={audit4},
  number={4}
  ]

  \Question Risposta esatta: 2

\end{Answer}

\begin{Exercise} [
  title={Quiz},
  label={audit5}
  ]

  \Question Il primo passo per un auditor dovrebbe essere quello di:
  \begin{enumerate}
   \item Perparare gli obiettivi dell'audit e il suo scopo
   \item Imparare riguardo l'organizzazione
   \item Studiare le raccomandazioni ISACA sull'audit per l'area funzionale
   \item Eseguire una valutazione del rischio
  \end{enumerate}

\end{Exercise}


\begin{Answer} [
  ref={audit5},
  number={5}
  ]

  \Question Risposta esatta: 2

\end{Answer}


\begin{Exercise} [
  title={Quiz},
  label={audit6}
  ]

  \Question Un audit che considera come le informazioni finanziarie sono 
generate da entrambi i processi di business e \textit{IS} è anche conosciuto 
come:
\begin{enumerate}
 \item \textit{Financial audit}
 \item \textit{Operational audit}
 \item \textit{Administrative audit}
 \item \textit{Integrated audit}
\end{enumerate}

\end{Exercise}


\begin{Answer} [
  ref={audit6},
  number={6}
  ]

  \Question Risposta esatta: 4

\end{Answer}

\begin{Exercise} [
  title={Quiz},
  label={audit7}
  ]

  \Question Un auditor sovra-testa (ovvero testa una percentuale più grande di 
quello che in verità esiste) esempi di cui si aspetta essere più 
rischiosi\footnote{Testo originale in inglese: \texttt{An auditor over-tests 
(tests a greater percent than actually exist) samples that are expected to be 
most risky:}}:
\begin{enumerate}
 \item \textit{Variable Sampling}
 \item \textit{Attribute Sampling}
 \item \textit{Statistical Sampling}
 \item \textit{Non-statistical Sampling}
\end{enumerate}

\end{Exercise}


\begin{Answer} [
  ref={audit7},
  number={7}
  ]

  \Question Risposta esatta: 4

\end{Answer}


\begin{Exercise} [
  title={Quiz},
  label={audit8}
  ]

  \Question La possibilità che un router non rilevi un indirizzo IP 
\textit{spoofed} è anche conosciuto come:
\begin{enumerate}
 \item Rischio intrinseco
 \item Controllo del rischio
 \item Individuazione del rischio
 \item Rischio esterno
\end{enumerate}  

\end{Exercise}


\begin{Answer} [
  ref={audit8},
  number={8}
  ]

  \Question Risposta esatta: 2

\end{Answer}

\begin{Exercise} [
  title={Quiz},
  label={audit9}
  ]

  \Question Testare un \textit{firewall} per assicurarsi che questo permetta 
traffico web nella DMZ si chiama:
\begin{enumerate}
 \item \textit{Compliance Test}
 \item \textit{Substantive Test}
 \item \textit{Detection Test}
 \item \textit{Preventive Test}
\end{enumerate}
\end{Exercise}


\begin{Answer} [
  ref={audit9},
  number={9}
  ]

  \Question Risposta esatta: 1

\end{Answer}

\begin{Exercise} [
  title={Quiz},
  label={audit10}
  ]

  \Question Un rischio inerente \todo{Controllare la traduzione} per una scuola 
sarebbe:
\begin{enumerate}
 \item Gli studenti che provano ad entrare nel sistema per cambiarsi i voti
 \item Un \textit{firewall} che non intercetta gli indirizzi IP \textit{spoofed}
 \item Un audit che non trova una frode che in verità esiste
 \item Persone che non cambiano la password constantemente
\end{enumerate}
  
\end{Exercise}


\begin{Answer} [
  ref={audit10},
  number={10}
  ]

  \Question Risposta esatta: 1

\end{Answer}
