
Al livello operazionale deve essere dettagliata

\paragraph*{Standard IT Balanced Scorecard} 
Le \textit{dashboard} sono tutti quelli strumenti di gestione medio alti che 
fanno capire i vari livelli (di liquidità, di sicurezza aziendale). Sono 
indicatori semplici che sintetizzano strutture aziendali molto più complesse.

Ma come fanno le aziende a prendere le decisioni? Si sta andando sempre di più 
su decisioni basati sui dati. I rapporti contengono i dati, ma molto spesso 
vengono visualizzati solamente i grafici su questi report.

Misure: Statistiche costumer satisfaction (facile) e efficienza operazionale, 
che però non è facile da misurare. Per esempio il numero di ticket chiusi in un 
call center, o l'utilizzo di uno scontrino virtuale.

\todo{Aggiungere l'immagine???}

Bisogna eccellere nei processi relativi al core business. Per controllare i 
processi core del business si va a vedere l'analisi dei rischi, perché viene 
fatta su questi processi.

\subsection{Organizzazione della sicurezza}

\subsubsection{Board of Directors}
In cima c'è il \textit{Board of Directors}, che fa la \textit{review} del 
\textit{risk assessment} e del \textit{Business Impact Analysis}; definiscono 
anche la penalità per il non aderimento alle policy.

A livello più basso c'è il Management Esecutivo (Executive Management), ossia 
chi definisce gli obiettivi di sicurezza e instituisce l'organizzazione della 
sicurezza.

\subsubsection{Managers}
Il manager è colui che muove la società, che si sporca le mani azionando le leve 
all'interno delle società.
Esistono differenti linee di management, sostanzialmente ce ne sono tre:
\begin{enumerate}
\item Collaboratore
\item Lite Manager
\item Lite Manager di secondo livello. Di solito i manager di questo livello 
gestiscono in maniera effettiva da 5 a 7 persone.
\item Lite manager di terzo livello (detto anche \textit{executive})
\end{enumerate}

\subsection{Steering commitee}
Lo \textit{Steering commitee} è un comitato permamente composto da 
rappresentatnti senior delle funzionalità del business. Il suo ruolo è 
assicurare l'allineamento del programma di sicurezza con gli obiettivi del 
business.

CISO è il passo penultimo prima di arrivare a quello che è l'obiettivo, ossia 
diventare CIO. Altre posizioni sono il \textit{Chief Risk Officer} e il 
\textit{Chief Compliance Officer}.

\paragraph*{Main Concerns}

\begin{itemize}
\item Prende decisioni sulla centralizzazione/decentralizzazione dell'IT, 
assegna le responsabilità
\item Fa raccomandazioni per i piani strategici
\item approva l'architettura IT
\item fa la review e approva i piani, i budget e le priorità IT
\end{itemize}

Cambiare architettura IT ottimizza il tempo di esecuzione dei processi, aiutando 
il business (tempo di presa di decisioni minori, esecuzioni più rapide, ecc...)

\subsubsection{IT Governance Committees}
IT Strategic Committee: si concentra su la direzione e la strategia, consiglia 
il board su le strategie IT e gli allineamenti, ottimizza i costi e i risk 
dell'IT.

\paragraph{Main concerns}

\begin{itemize}
\item Allineamento dell'IT con il business
\item contributo dell'IT al business
\item esposizione e contenimento dei rischi
\item ottimizzazione dei costi IT
\item raggiungimento \todo{Finiscimi}
\end{itemize}


\subsubsection{Executive MGMT info security concerns}

\begin{itemize}
\item Ridurre civil and legal liability related to privacy, la nuova normativa 
sulla privacy europea prevede multe fino al 5\% del fatturato (non dei 
profitti).
\item Fornisce le policy e gli standard
\item Controlla che i rischi a livelli accettabili
\item Bisogna avere l'approvazione del business e avere una giusta 
\textit{leadership}
\item Basa le decisioni su informazioni accurate, solo che molte volte non sono 
cosí accurate. Le decisioni vanno prese quando si hanno abbastanza informazioni. 
Quando ho l'80\% dell'informazione prendo la decisione, perchè aspettare il 90, 
il 100 ci si mette troppo.
\item \todo{Qui che va?}
\end{itemize}


Le decisioni vengono continuamente prese in condizioni di incertezza, ma 
comunque basandosi sui dati\footnote{Il capo di Amazon ha affermato che prende 
decisioni con un po' di incertezza, all'incirca quando ha l'80\% 
dell'informazione, ovvero quando si hanno abbastanza dati ma c'è un certo 
margine d'incertezza.}.


\subsection{Esercizi}

Gli esercizi possono essere trovati nella sezione \ref{esPG:SPP}

\paragraph{Road Map for Security}
\todo{Copy Image?}

\paragraph{Security Relationships}

\todo{Copia l'immagine!}

RFP tipico di un CISO è disaster e recovery. 


% Guardati un video-meme: https://youtu.be/uvZXFvvjTQ0


\paragraph{Security Positions} 
Security Administrator
\todo{Aggiungi i punti dalla slide}

\begin{itemize}
\item 
\item 
\item
\item 
\end{itemize}
Security Architect è una posizione senior, ci vuole tanta esperienza e viene 
pagata meglio del security Administrator, in quanto per fare il Security 
Architect bisogna prima aver fatto l'Amministratore. 
Security Architect piú ad alto livello. Ci sarebbe anche il CTSO che capisce di 
una certa teconologia e fa da tramite con il business
\todo{Aggiungi i punti dalla slide}
\begin{itemize}
\item 
\item 
\item
\item 
\end{itemize}
% Vai mirko crediamo in te
\todo{Inserisci disegnino fatto alla lavagna}
\paragraph{Security Architect: Control Analysis}

Se si vuole fare il \textit{security architect} il \textit{control analysis} è 
importante.

\todo{Inserisci disegnino del pentagono}