\section{Controlli di Sicurezza}
\label{esCs}

\subsection{Application Controls}
\label{esCs:ac}

La teoria riguardante questi esercizi si trova in \ref{cs:ac}


\begin{Exercise} [
  title={Quiz},
  label={esCs1}
  ]

  \Question Un hash totale dei numeri del cliente è un input al programma di
vendita. Questo programma genera il suo totale e lo confronta con quello 
fornito in input. Qual è lo scopo di questa procedura?
\begin{enumerate}
 \item Assicurarsi che i numeri del cliente siano accurati
 \item Trovare transazioni modificate o perse
 \item Trovare errori nelle transazioni delle vendite
 \item Assicurarsi che ogni transazione di vendita sia completa
\end{enumerate}

\end{Exercise}

\begin{Answer} [
  ref={esCs1},
  number={1}
  ]

  \Question Risposta esatta: 2
\end{Answer}


\begin{Exercise} [
  title={Quiz},
  label={esCs2}
  ]

  \Question Il \textit{batch balancing} \`e usato per
  \begin{enumerate}
   \item Assicurarsi che i dati di test corrispondano accuratamente ai dati
reali quando vengono considerati i tipi di transazione
   \item Trovare transazioni perse o cambiate durante la loro elaborazione
   \item Trovare errori nelle transazioni delle vendite
   \item Verificare che il lotto totale abbia senso
  \end{enumerate}
\end{Exercise}

\begin{Answer} [
  ref={esCs2},
  number={2}
  ]

  \Question Risposta esatta: 2
\end{Answer}

\begin{Exercise} [
  title={Quiz},
  label={esCs3}
  ]

  \Question Le somme dei batch (\textit{batch totals}) potrebbero non 
corrispondere quando delle transazioni erronee vengono rimosse. 
Il processo che verifica che il trattamento delle transazioni sia
avvenuto correttamente e contemporaneamente rende conto degli errori 
viene chiamato\footnote{Testo originale: \texttt{Batch totals may not
match when error transactions are removed. The process that verifies 
full processing did occur correctly, while accounting for errors is
called:}}:

\begin{enumerate}
\item Audit trail
\item Validation
\item Batch balancing
\item Reconciliation
\end{enumerate}


\end{Exercise}

\begin{Answer} [
  ref={esCs3},
  number={3}
  ]

  \Question Risposta esatta: 4
\end{Answer}


\subsection{Application Audit}
\label{esCs:aa}

La teoria riguardante questi esercizi si trova in \ref{cs:aa}


\begin{Exercise} [
  title={Quiz},
  label={esCs4}
  ]

  \Question I moduli integrati di audit sono associati per la maggior parte con:
\begin{enumerate}
\item Audit Hooks
\item Snapshots
\item Batch processing
\item Parallel operation
\end{enumerate}



\end{Exercise}

\begin{Answer} [
  ref={esCs4},
  number={4}
  ]

  \Question Risposta esatta: 1
\end{Answer}


\begin{Exercise} [
  title={Quiz},
  label={esCs5}
  ]

  \Question Questa tecnica fornisce informazioni statistiche riguardanti dati
normali dati in input da file, per determinare se il file \`e sufficentemente
vario per l'auditor:
\begin{enumerate}
\item test data
\item snapshots
\item system control audit and review file
\item transaction selection program
\end{enumerate}

\end{Exercise}

\begin{Answer} [
  ref={esCs5},
  number={5}
  ]

  \Question Risposta esatta: 3
\end{Answer}


\begin{Exercise} [
  title={Quiz},
  label={esCs6}
  ]

  \Question Combinare dati reali e di testing durante un audit \`e viene
definito come:
\begin{enumerate}
\item parallel operation
\item integrated testing facilities
\item batch processing
\item embedded audit modules
\end{enumerate}

\end{Exercise}

\begin{Answer} [
  ref={esCs6},
  number={6}
  ]

  \Question Risposta esatta: 2
\end{Answer}


\begin{Exercise} [
  title={Match definizioni},
  label={esCs7}
  ]

  \Question

  Fare il match delle seguenti parole chiave
  \begin{itemize}
   \item Duplicate Check
   \item Existence Check
   \item Reasonableness Check
   \item Limit Check
   \item Key Verification
   \item Sequence Check
  \end{itemize}

  con le seguenti definizioni:
  \begin{enumerate}
   \item Sequence number use causes out-of-sequence and duplicate 
   numbers to be rejected.
   \item Valid numbers are below a maximum value.
   \item Values entered are plausible
   \item Required fields are entered correctly.
   \item Input is double checked via second person OR all digits are entered
twice.
   \item Transactions with duplicate IDs are checked for and rejected.
  \end{enumerate}

\end{Exercise}

\begin{Answer} [
  ref={esCs7},
  number={7}
  ]

  \Question Soluzione:
  \begin{itemize}
   \item Duplicate Check: 5
   \item Existence Check: 4
   \item Reasonableness Check: 3
   \item Limit Check: 2
   \item Key Verification: 6
   \item Sequence Check: 1
  \end{itemize}


\end{Answer}
