\part{Frode e Business Continuity}

\chapter{Frodi}
\label{Frodi}

Solo nell'ultimo anno ci sono state perdite per 650 milioni di dollari
a causa delle frodi. Le frodi ultimamente sono aumentate,
in quanto con l'informazione di massa è più semplice avere accesso a
questo tipo di crimine.

Le frodi solitamente sono grosse: in media le compagnie più piccole vengono
derubate di più in valore assoluto rispetto a quelle grandi, a causa di una
presenza meno massiccia di controlli.

Cos'è una frode? È la violazione della fiducia del datore di lavoro da parte
dell'impiegato.

A causa delle frodi, si hanno perdite in termini diretti o
indiretti (es. immagine o reputazione).

\section{Comportamento delle frodi}

Le frodi si suddividono in diverse tipologie:

% Riportare tipologia di frodi
\begin{itemize}
  \item Appropriazione indebita (91\%)
  \item Corruzione (31\%): consiste in una ``spintarella'', tipicamente a causa di un conflitto di
  interessi. I conflitti di interessi accadono quando, in una azienda di
  consulenza, si hanno consulenti che lavorano per aziende \textit{competitor}.
  È necessario quindi impedire che consulenti di aziende in competizione possano
  accedere a   certe tipologie di informazioni\footnote{Il conflitto di interessi è una
  problematica fondamentale nel campo dell'ICT.}.
  \item Dichiarazione illecita (10\%): un problema che capita spesso con le banche è l'\textbf{overstating assets}.
  Accade quando una banca creditrice mette nel bilancio totale questi crediti.
  Se i debitori risultano insolventi è come aver dichiarato di più di quello
  che effettivamente si ha. Questo è un problema grosso soprattutto riguardo ai
  controlli.
\end{itemize}
